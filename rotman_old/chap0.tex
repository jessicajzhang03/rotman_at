\documentclass[/main.tex]{subfiles}
%\documentclass{article} 
%\usepackage{jjz} 
%\usetikzlibrary{cd}
%\title{{\sffamily Chapter 0: Introduction}} 
%\author{{\sffamily Jessica Zhang}} 
%\date{{\sffamily \today}} 

\begin{document} 
\maketitle
\begin{exercise}
As per the hint, observe that if $y\in G$, then we have $y=r(y)+(y-r(y))$. Obviously, we have $r(y)\in H$. Moreover, we know that \[r(y-r(y))=r(y)-r(r(y))=0,\] and so $y-r(y)\in\ker r$. Thus $G\subseteq H\oplus\ker r$. 

The reverse is obviously true, since $H$ and $\ker r$ are both subgroups of $G$. 
\end{exercise} 

\begin{exercise}
Suppose instead that $f:D^1\to D^1$ has no fixed point. Then consider the continuous map $g:D^1\to S^0$ given by \[g(x)=\begin{cases}1&~\text{if}~f(x)<x\\-1&~\text{if}~f(x)>x\end{cases}.\] Notice that because $f(x)\ne x$ for all $x$, the function $g$ is well-defined. 

Moreover, we know that $f(-1)\ne-1$, since $f$ has no fixed point, and so $f(-1)>-1$. Thus $g(-1)=-1$. Similarly, we have $g(1)=1$. 

Thus we have $g(D^1)=S^0$, which is disconnected. This is a contradiction, so $f$ must have had a fixed point. 
\end{exercise} 

\begin{exercise}
Suppose that $r$ is such a retract. Then we have the following commutative diagram: 
\[\begin{tikzcd}
&S^n\arrow[rd,"r"]&\\
S^{n-1}\arrow[ru,"i"]\arrow[rr,"1"]&&S^{n-1}. 
\end{tikzcd}\] 
Applying $H_{n-1}$, we get another commutative diagram: 
\[\begin{tikzcd}[row sep=huge] 
&H_{n-1}(S^n)\arrow[rd,"H_{n-1}(r)"]&\\
H_{n-1}(S^{n-1})\arrow[ru,"H_{n-1}(i)"]\arrow[rr,"H_{n-1}(1)"]&&H_{n-1}(S^{n-1}).
\end{tikzcd}\]
We know that $H_{n-1}(S^n)=0$, however, implying that $H_{n-1}(1)=0$. This contradicts the fact that $H_{n-1}(S^{n-1})=\ZZ\ne0$. Thus the retraction $r$ could not have existed. 
\end{exercise} 

\begin{exercise}
Suppose $g:D^n\to X$ is a homeomorphism. Then we know that $g^{-1}\circ f\circ g$ is a continuous map from $D^n$ to itself, and so it has a fixed point $x$. Then we know that $g^{-1}(f(g(x)))=x$, and so it follows that $f(g(x))=g(x)$. Thus $g(x)\in X$ is a fixed point of $f$. 
\end{exercise} 
\end{document} 