\documentclass[../../solutions.tex]{subfiles}

\setcounter{section}{8}

\begin{document}
\section{Chapter 9: Natural Transformations}
\subsection{Definitions and Examples}
\begin{exercise} \leavevmode
This is obvious from Lemma~4.8.
\end{exercise}

\begin{exercise} \leavevmode
This is exactly the statement of \Cref{4.10}.
\end{exercise}

\begin{exercise} \leavevmode
The commutative diagram in \Cref{4.13} is exactly the statement that the map is natural.
\end{exercise}

\begin{exercise} \leavevmode
Commutativity of the diagram
\[
\begin{tikzcd}
H_n(X,A)\ar[r,"f_*"]\ar[d,"\partial",swap] & H_n(Y,B)\ar[d,"\partial"]\\
H_{n-1}(A,\emptyset)\ar[r,"(f|A)_*",swap] & H_{n-1}(B,\emptyset)
\end{tikzcd}
\]
follows from the exact sequence in Theorem~5.8, since $H_{n-1}(A,\emptyset)=H_{n-1}(A)$.
\end{exercise}

\begin{exercise} \leavevmode
This is again precisely the statement from \Cref{6.8}.
\end{exercise}

\begin{exercise} \leavevmode
\begin{enumerate}
\item
Suppose $\sig:F\to G$ and $\tau:G\to H$ are natural.
Then we can ``stack'' the commutative diagrams:
\[
\begin{tikzcd}
F(C)\ar[r,"Ff"]\ar[d,"\sig_C",swap] & F(D)\ar[d,"\sig_D"]\\
G(C)\ar[r,"Gf"]\ar[d,"\tau_C",swap] & G(D)\ar[d,"\tau_D"]\\
H(C)\ar[r,"Hf"] & H(D)
\end{tikzcd}
\]
Hence it follows that $\tau\sig=(\tau_C\sig_C)$ gives a natural transformation.

\item
Reflexivity is due to the commutativity of the following diagram:
\[
\begin{tikzcd}
F(C)\ar[r,"Ff"]\ar[d,"1_{F(C)}",swap] & G(C)\ar[d,"1_{G(C)}"]\\
F(C)\ar[r,"Ff"] & G(C)
\end{tikzcd}
\]
To see symmetry, simply choose $\tau_C^{-1}$ for each object $C$.
This can be done because each $\tau_C$ is an equivalence.
Finally, transitivity follows from the previous part and the fact that the composition of equivalences is an equivalence.
\end{enumerate}
\end{exercise}

\begin{exercise} \leavevmode
\begin{enumerate}
\item
If $\phi\in\Nat(\Hom(~,A),F)$, then $\phi_A$ is a map from $\Hom(A,A)$ to $F(A)$.
Since $1_A\in\Hom(A,A)$, it follows that $\phi_A(1_A)\in F(A)$.
Thus $y$ is a well-defined function.
\item
We must check that $\tau\in\Nat(\Hom(~,A),F)$ whenever $\mu\in F(A)$.
First, observe that $\tau_X$ is indeed a morphism from $\Hom(X,A)$ to $F(X)$.
After all, if $f:X\to A$, then $Ff:FA\to FX$.
Hence $\tau_X(f)=(Ff)(\mu)$ is an element of $F(X)$.

To see that $\tau$ is natural, we must show that the following diagram commutes for all $f:X\to Y$.
\[
\begin{tikzcd}[column sep=huge]
\Hom(X,A)\ar[d,"\tau_X",swap] & \Hom(Y,A)\ar[l,"\Hom(f{,}A)",swap]\ar[d,"\tau_Y"] \\
F(X) & F(Y)\ar[l,"Ff"]
\end{tikzcd}
\]
But for each $g\in\Hom(X,A)$, we know that
\[(Ff)\circ\tau_Y(g)=(Ff)(Fg(\mu))=F(g\circ f)\mu,\]
while we have $\Hom(f,A)=f^*$, so that
\[\tau_X\circ\Hom(f,A)(g)=\tau_X\circ f^*(g)=\tau_X(g\circ f)=F(g\circ f)\mu.\]
These are equal, so $\tau$ is a natural transformation.

\item
First, we will show that $y\circ y':F(A)\to F(A)$ is the identity.
Let $\mu\in F(A)$.
Then we know that
\[y'(\mu)=\{\tau_X:f\mapsto(Ff)(\mu)\},\]
and so we have that
\[y(y'(\mu))=(y'(\mu))_A(1_A)=F(1_A)(\mu).\]
But $F$ is a functor, so $F(1_A)=1_{F(A)}$, and so this is exactly equal to $1_{F(A)}(\mu)=\mu$, which proves that $y\circ y'$ is the identity on $F(A)$.

Now to check $y'y$, suppose $\phi\in\Nat(\Hom(~,A),F)$.
Then we know that
\[y'(y(\phi))=\{\tau_X:f\mapsto(Ff)(\phi_A(1_A)).\]
We would like to show that
\[(Ff)(\phi_A(1_A))=\phi_Xf,\]
since this will imply that $y'(y(\phi))=\phi$.
But we know that the following diagram commutes:
\[
\begin{tikzcd}[row sep=large]
\Hom(X,A)\ar[d,"\phi_X",swap] & \Hom(A,A)\ar[l,"f^*",swap]\ar[d,"\phi_A"] \\
F(X) & F(A)\ar[l,"Ff"]
\end{tikzcd}
\]
Thus we know, in particular, that
\[Ff\circ\phi_A(1_A)=\phi_Xf^*(1_A)=\phi_X(1_A\circ f)=\phi_x\circ f.\]
This is what we wanted.

\item
If $\phi:\Hom(~,A)\to\Hom(~,B)$ is natural, then we have the following commutative diagram:
\[
\begin{tikzcd}[row sep=huge, column sep=2.5cm]
\Hom(X,A)\ar[d,"\phi_X",swap] & \Hom(Y,A)\ar[l,"\Hom(f{,}A)",swap]\ar[d,"\phi_Y"] \\
\Hom(X,B) & \Hom(Y,B)\ar[l,"\Hom(f{,}B)"]
\end{tikzcd}
\]
Let $F$ be the functor $\Hom(~,B)$.
Then we know that
\begin{align*}
\phi_X(f)&=y'(y(\phi))_X(f)\\
&=(Ff)(\phi_A(1_A))\\
&=\Hom(f,B)(\phi_A(1_A))\\
&=\phi_A(1_A)\circ f,
\end{align*}
where $f:X\to A$.
Thus $\phi_X(f)=\mu f$, as desired.


\item
Same proof.
\end{enumerate}
\end{exercise}

\begin{exercise} \leavevmode
We must verify the properties of a category.
To see that the family of $\Hom(F,G)$'s, where $F$ and $G$ are functors $\cat C\to\cat A$, is disjoint, notice that this means that there exists some $\tau=(\tau_C:F(C)\to G(C))$ and $\sig=(\sig_C:F'(C)\to G'(C))$ which are equal.
Hence $F(C)=F'(C)$ and $G(C)=G'(C)$ for all $C$, since $\tau_C=\sig_C$ is in both $\Hom(F(C),G(C))$ and $\Hom(F'(C),G'(C))$.
Since this is true for all $C\in\cat C$, it follows that $F=F'$ and $G=G'$.

Composition of natural transformations reduces to composition of morphisms, which is associative.

Finally, note that $1_A\in\Hom(F,F)$ given by
\[1_A=\{(1_A)_C=1_{F(C)}\}\]
works as an identity morphism.
\end{exercise}

\begin{exercise} \leavevmode
\begin{enumerate}
\item
We shall verify the properties of a contravariant functor.
The functor gives us a complex
\[
\begin{tikzcd}
\dots\ar[r] & C_{n+1}\ar[r,"\partial_{n+1}"] & C_n\ar[r,"\partial_n"] & C_{n-1}\ar[r] & \dots\ar[r] & C_0\ar[r] & C_{-1}\ar[r] & \dots
\end{tikzcd}
\]
Since $C_n$ is abelian, we know that $n\in\ZZ$ implies that $C(n)=C_n\in\Ab$.

The only morphisms in $\ZZ$ are $\iota^x_y$ when $x\le y$.
Note that $C(\iota^x_y)$ is the composition $\partial_{x+1}\circ\dots\circ\partial y:C_y\to C_x$.
We must verify that composition is reversed and identities are respected.
But this is clear from the definition:
\[]iota_z^y\circ\iota_y^x=\iota_z^x=\partial_{x+1}\circ\dots\circ\partial_z\]
is exactly $C(\iota^x_y)\circ C(\iota^y_z)$, and $C(\iota^x_x)$ is the composition of an empty set of differentiation operators, and thus is the identity.

\item
The chain map condition is exactly the condition of commutativity.
\end{enumerate}
\end{exercise}

\subsection{Eilenberg--Steenrod Axioms}
No exercises!

\subsection{Chain Equivalences}
\begin{exercise} \leavevmode
To prove (i) implies (ii), note that $ps=1_C$ implies $s$ is injective.
Then the same argument as in Corollary~9.2 implies that $B=\ker p\oplus\im s$.
Of course, we have $\ker p=\im i$ and $C'=\im s=s(C)\cong C$.
Since $p(C')=C$, this proves the first implication.

The second implication is clear.
In particular, consider $q:B\to A$ defined by $(i(x),c)\mapsto x$.
Then $qi(a)=q(i(a))=a$.

Finally, to show (iii) implies (i), define $s(c)$ as
\[s(c)=p^{-1}(c)-iqp^{-1}(c).\]
To see that this is well-defined, pick $b\in\ker p=\im i$, so $b=i(a)$.
Thus $b-iq(b)=i(a)-iqi(a)=0$.
Hence $p(b)=p(b')$ means that $b-iqb=b'-iqb'$, proving well-definedness.
To see that this choice of $s$ gives a split exact sequence, simply verify that
\[ps(c)=p(p^{-1}(c)-iqp^{-1}(c))=c-piqp^{-1}(c).\]
Since $pi=0$, this is equal to $c$.
\end{exercise}

\subsection{Acyclic Models}
\begin{exercise} \leavevmode
First we show that the diagram given by Rotman commutes, i.e., that
\[\partial_n(t_n-t;_n-s_{n-1}d_n)=0.\]
We know that
\begin{align*}
\partial_nt_n-\partial_nt'_n-\partial_ns_{n-1}d_n&=t_{n-1}d_n-t'_{n-1}d_n-\partial_ns_{n-1}d_n.
\end{align*}
The inductive hypothesis implies that
\[\partial_ns_{n-1}=t_{n-1}-t'_{n-1}-s_{n-2}d_{n-1}.\]
Plugging this value in and canceling gives us
\[\partial_nt_n-\partial_nt'_n-\partial_ns_{n-1}d_n=s_{n-2}d_{n-1}d_n=0,\]
because $dd=0$.

Thus the diagram commutes.
In particular, we know that
\[\im(t_n-t_n'-s_{n-1}d_n)\subseteq\ker\partial_n=\im\partial_{n+1},\]
where the final equality comes from the fact that $E_*$ is an acyclic complex.
This means that we can rewrite the diagram as follows:
\[
\begin{tikzcd}[column sep=large, row sep=large]
& F_n\ar[d,"t_n-t'_n-s_{n-1}d_n",swap] & \\
E_{n+1}\ar[r,"\partial_{n+1}",swap] &
\parbox[t]{3cm}{\centering $\im(t_n-t'_n-s_{n-1}d_n)$\\{\small$=\im\partial_{n+1}=\ker\partial_n$}}\ar[r,"\partial_n=0",swap] & 0
\end{tikzcd}
\]
Thus Theorem~9.1 implies that we can find $s_n$ with the desired properties.
\end{exercise}

\begin{exercise} \leavevmode
We have $F(g)=F(0+g)=F(0)+F(g)$, so $F(0)$ acts as the 0 element.
If $A$ is the zero group, then its identity is the zero homomorphism.
Hence $1_{F(A)}=F(1_A)$ is the zero homomorphism, so $F(A)=0$.
\end{exercise}

\begin{exercise} \leavevmode
\begin{enumerate}
\item
We'll prove the covariant case.
By \Cref{9.10}, we have a morphism $q:A\to B$ with $qi=1_A$.
Note that $(Fp)\circ(Fs)=F(p\circ s)=F(1_C)=1_{F(C)}$, and similarly for $q$ and $i$, so that we still have a split sequence, as long as it is exact.
Moreover, these imply that $Fp$ is surjective and $Fi$ is injective.

It now suffices to check that $\im Fi=\ker Fp$.
But notice that $B\cong iq(B)\oplus sp(B)$ implies that $F(B)$ is equal to the functored version of the right side, thus making the center of the short functored sequence exact.

\item
This simply uses induction on $|I|=n+1$ and the following short exact sequence:
\[
\begin{tikzcd}
0\ar[r] & \sum_{i=1}^nA_i\ar[r,"i"] & \sum_{i=1}^{n+1}A_i\ar[r,"p"] & A_{n+1}\ar[r] & 0.
\end{tikzcd}
\]
Note that this is split exact with $s:a_{n+1}\mapsto(0,\dots,0,a_{n+1})$.
Thus the previous part applies, and \Cref{9.10} implies that
\[F\left(\sum_{i=1}^{n+1}A_i\right)\cong F\left(\sum_{i=1}^nA_i\right)\oplus F(A_{n+1})=\sum_{i=1}^{n+1}F(A_i),\]
where the last equality follows from the inductive hypothesis.
\end{enumerate}
\end{exercise}

\begin{exercise} \leavevmode
\begin{enumerate}
\item
If $\partial_n\partial_{n+1}=0$, then $F(\partial_n\partial_{n+1})=0$ thanks to additivity.
This proves that the functored complex is a chain complex too.

\item
Note that $f_{n-1}\partial_n=\partial'_nf_n$ implies that $F(f_{n-1}\partial_n)=F(\partial'_nf_n)$.
Since functors respect composition, this proves the result.

\item
Note that additive functors respect homotopy because they rerspect both composition and addition.
Hence if $g:B_*\to A_*$ makes $f$ an equivalence, i.e., if $g\circ f\simeq 1_{A_*}$ and $f\circ g\simeq 1_{B_*}$, then it follows that
\[Fg\circ Ff\simeq F1_{A_*}=1_{FA_*},\]
and similarly for $B$.
Hence $Fg$ is an inverse for $Ff$, so $Ff$ is a chain equivalence.
\end{enumerate}
\end{exercise}

\begin{exercise} \leavevmode
\begin{enumerate}
\item
This simply involves applying Corollary~9.13(ii).
In particulars, we know that $F_p$ and $S_p$ arer both free with basis in $\mathcal M=\{\Dlt^p\}$.
We want to show that $\Dlt^p$ is totally $S$- and $F$-acyclic.
But notice that $\tilde H_n(S_*(\Dlt^k))=0$ because $\Dlt^k$ is contractible, and similarly for $F_p$, since it coincides with $C_p$ ono $\Dlt$.
This proves acyclicity, and so the two are naturally chain equivalent.

\item
Theorem~9.8 implies that singular and large simplicial homology are the same, while Theorem~7.22 implies that normal simplicial and singular homology are the same.
\end{enumerate}
\end{exercise}

\subsection{Lefschetz Fixed Point Theorem}
\begin{exercise} \leavevmode
Notice that $1_G$ induces $1_{G/tG}:x+tG\mapsto x+tG$.
Hence, with any basis $\{x_1,\dots,x_n\}$ of $G/tG$, we have $1_{G/tG}$ equal to the identity matrix whose dimension is $\rank G/tG$.
\end{exercise}

\begin{exercise} \leavevmode
A basis $\{x_1,\dots,x_k\}$ of $G'/tG'$ can be extended to $\{x_1,\dots,x_n\}$ of $G/tG$.
Since $G''$ is just $G/G'$ and $f''(g+G')=pf(g)=f(g)+G'$.
Thus $f''(x_i+G')=f(x_i)+G'$ for $i=k+1,\dots,n$.
Thus the matrix of $\bar f$ is diagonal, of the form shown on p.~259 of the textbook, which implies the result.
\end{exercise}

\begin{exercise} \leavevmode
If $f:S^n\to S^n$, then $f_{0*}$ and $f_{n*}$ are maps $\ZZ\to\ZZ$.
Note that $f_{0*}$ is the identity, and thus has trace 1.
If $\tr f_{n*}=1$ as well, then the whole map is homotopic to either the identity or the antipodal map, implying that $f$ is a homotopy equivalence.
Thus $\tr f_{n*}=0$, and so $\la(f)=1\ne0$.
The Lefschetz fixed point theorem implies the result.
\end{exercise}

\subsection{Tensor Products}
\begin{exercise} \leavevmode
Note that
\begin{align*}
a\otimes0+a'\otimes b'&=a\otimes0+(a\otimes b'+(a'-a)\otimes b')\\
&=a\otimes b'+(a'-a)\otimes b'\\
&=a'\otimes b',
\end{align*}
and similarly for $0\otimes b$.
\end{exercise}

\begin{exercise} \leavevmode
We would like to show that $m(a,b)\sim(ma,b)$ for $m\in\ZZ$.
It is true for $m>0$ by induction, true for $m=0$ by \Cref{9.19}, and true for $m<0$ by inverses.
\end{exercise}

\begin{exercise} \leavevmode
The hint gives the full solution.
If $a\in A$ then there exists some $m>0$ so that $ma=0$.
Hence $a\otimes q=ma\otimes(q/m)=0$.
Since this is true fora all generators of $A\otimes\QQ$, the result follows.
\end{exercise}

\begin{exercise} \leavevmode
Let $m$ be the order of $a\in A$ and $n$ the order of $b\in B$.
Then  we know that $\gcd(m,n)=1$, so that there exist integers $x,y$ with $mx+ny=1$.
Hence we have that
\begin{align*}
a\otimes b&=a\otimes(mx+ny)b\\
&=(mx+ny)(a\otimes b)\\
&=mx(a\otimes b)+ny(a\otimes b)\\
&=(mxa\otimes b)+(a\otimes nyb)=0.
\end{align*}
\end{exercise}

\begin{exercise} \leavevmode
This is the exact same argument as Corollary~9.27.
\end{exercise}

\begin{exercise} \leavevmode
\begin{enumerate}
\item
Use Theorem~9.25(ii) with the fact that $A\times B\cong B\times A$.

\item
To see this, simply consider the following commutative diagram:
\[
\begin{tikzcd}
A\otimes B\ar[d]\ar[r,"1_A\otimes f"] & A\otimes C\ar[d]\\
B\otimes A\ar[r,"f\otimes1_A",swap] & C\otimes A
\end{tikzcd}
\]
\end{enumerate}
\end{exercise}

\begin{exercise} \leavevmode
Note that $T_A(f+g)=1_A\otimes f+1_A\otimes g$, since both maps complete the diagram
\[
\begin{tikzcd}
A\times B\ar[dr,"\phi",swap]\ar[rr] & & A\otimes B\ar[dl,dashed] \\
& A\otimes B &
\end{tikzcd}
\]
where $\phi(a,b)=(a,(f+g)(b))$.
But note that $1_A\otimes f+1_A\otimes g$ is just $T_A(f)+T_A(g)$, proving additivity.
\end{exercise}

\begin{exercise} \leavevmode
This is clear, since we have
\begin{align*}
1_A\otimes f:A\otimes B&\to A\times B\\a\otimes b&\mapsto a\otimes fb=a\otimes mb=m(a\otimes b).
\end{align*}
\end{exercise}

\begin{exercise} \leavevmode
\begin{enumerate}
\item
This is easy to show directly.
In particular, we show that $a\mapsto1\otimes a$ is an isomorphism.
We would like to show that $1\otimes a=n\otimes b$ if $nb=a$.
But $n\otimes b=n(1\otimes b)=1\otimes(nb)=1\otimes a$, as desired.
Hence this map is surjective.
It is injective because, otherwise, every $1\otimes a$ would be 0, which would violate the universal property of tensor products given by Theorem~9.25.
Hence this is an isomorphism.

\item
We must show that the following commutes:
\[
\begin{tikzcd}[column sep=large]
\ZZ\otimes A\ar[r,"1_A\otimes f"]\ar[d,"\tau_A",swap] & \ZZ\otimes B\ar[d,"\tau_B"] \\
A\ar[r,"f",swap] & B
\end{tikzcd}
\]
This commutes because
\[\tau_B\circ(1_A\otimes f):(n,a)\mapsto(n,f(a))\mapsto nf(a),\]
while
\[f\circ\tau_A:(n,a)\mapsto na\mapsto f(na),\]
and $nf(a)=f(na)$ since $f$ is a homomorphism.
\end{enumerate}
\end{exercise}

\subsection{Universal Coefficients}
\begin{exercise} \leavevmode
\begin{enumerate}
\item 
We can write $F=\sum A_j$ where $A_j=\ZZ x_j$, and $F'=\sum A'_k$ where $A'_k=\ZZ x'_k$.
Then $F\otimes F'$ is just
\begin{align*}
F\otimes F'&=F\otimes\sum A'_k=\sum(F\otimes A'_k)\\
&=\sum\left(\sum A_j\otimes A'_k\right)\\
&=\sum_{j,k}A_j\otimes A'_k.
\end{align*}
But it is easy to verify that $A_j\otimes A'_k=\ZZ(x_j\otimes x'_k)\cong\ZZ$, which proves the result.

\item 
This is obvious from the previous part since
\[\rank F\otimes F'=|J\times K|=|J||K|=\rank F\rank F'\]
\end{enumerate}
\end{exercise}

\begin{exercise} \leavevmode
This is simply an application of Theorem~9.28 and Corollary~9.30, along with \Cref{9.27}.
We end up with
\[A\otimes B=\ZZ/2\ZZ\oplus\ZZ/3\ZZ\oplus\ZZ/3\ZZ\oplus\ZZ/5\ZZ\oplus\ZZ/5\ZZ\oplus\ZZ/5\ZZ.\]
\end{exercise}

\begin{exercise} \leavevmode
\begin{enumerate}
\item 
Using coordinate-wise addition and scalar multiplication of the form
\[p\sum(q_i,g_i)=\sum(pq_i,g_i)\]
shows that $\QQ\otimes G$ is a $\QQ$-vector space.
Hence $\dim\QQ\otimes G$ is defined.

\item
This follows immediately from the $\Tor$ exact sequence, along with the fact that $\Tor(\QQ,B)=0$ for all $B$.
\end{enumerate}
\end{exercise}

\begin{exercise} \leavevmode
This is simply a calculation using the properties of Tor.
We end up with
\[\ZZ/3\ZZ\oplus\ZZ/5\ZZ.\]
\end{exercise}

\begin{exercise} \leavevmode
Using \Cref{9.30} with the short exact sequence
\[0\to F\to G\to G/F\to0\]
gives us
\[\dim\QQ\otimes G=\dim\QQ\otimes F+\dim\QQ\otimes G/F.\]
But $\dim\QQ\otimes G/F=0$ by \Cref{9.21} and the fact that $G/F$ is torsion.
Moreover, we know that $\QQ\otimes F$ has basis $(1,x_i)$, where $x_i$ is a generator of $F$, so $\dim\QQ\otimes F=\rank F=\rank G$, which proves the result.
\end{exercise}

\begin{exercise} \leavevmode
Note that [Tor 1] and [Tor 5] imply that there is an exact sequence
\[0\to\Tor(B',A)\to\Tor(B,A)\to\Tor(B'',A)\to B'\otimes A\to B\otimes A\to B''\otimes A\to0,\]
since $B\otimes A\cong A\otimes B$ by \Cref{9.24}.
But if $A$ is torsion-free, then $\Tor(B'',A)=0$ by [Tor 2], which gives us the desired exact sequence.
\end{exercise}

\begin{exercise} \leavevmode
This is false!
Consider, for example, when $F=\ZZ$ and $H=\ZZ/2\ZZ$, and $a=2$, $h=1$.
In general, we need the condition that if $a=\sum m_jx_j$, where $\{x_j\}$ is a basis for $F$, then $m_jh\ne0$ for at least some $j$.
After all, we need that
\[a\otimes h=(m_jx_j\otimes h)_j=(m_jh)\ne0.\]
\end{exercise}

\begin{exercise} \leavevmode
Let $\alpha$ be the map $(\cls z)\otimes g\mapsto\cls(z\otimes g)$.
Then the Universal Coefficients Theorem implies that
\[
\begin{tikzcd}
0\ar[r] & H_n(X)\otimes G\ar[r,"\alpha"] & H_n(X;G)\ar[r] & \Tor(H_{n-1}(X),G)\ar[r] & 0
\end{tikzcd}
\]
is exact.
Of course, since $G$ is torsion-free, we know that $\Tor(H_{n-1}(X),G)=0$.
Hence $\alpha$ is an isomorphism.
\end{exercise}

\begin{exercise} \leavevmode
Use the second part of the Universal Coefficients Theorem.
In particular, it gives us that
\[H_n(X;\ZZ/m\ZZ)\cong(H_n(X)\otimes\ZZ/m\ZZ)\oplus H_{n-1}(X)[m],\]
since
\[\Tor(H_{n-1}(X),\ZZ/m\ZZ)=H_{n-1}(X)[m]\]
by [Tor 4].
If $H_{n-1}(X)$ is torsion-free, the second term is zero, which gives the conclusion.
\end{exercise}

\subsection{Eilenberg--Zilber Theorem and the K\"unneth Formula}
\begin{exercise} \leavevmode
This is a straightforward calculation.
In particular, we find that 
\begin{align*}
(\la\otimes\mu)_{n-1}D_n(c_i\otimes e_j)&=(\la\otimes\mu)_{n-1}(dc_i\otimes e_j+(-1)^ic_i\otimes\partial e_j)\\
&=(\la_{i-1}\otimes\mu_j)(dc_i\otimes e_j)+(\la_i\otimes\mu_{j-1})((-1)^ic_i\otimes\partial e_j)\\
&=\la_{i-1}dc_i\otimes\mu_je_j+(-1)^i\la_ic_i\otimes\mu_{j-1}\partial e_j.
\end{align*}
A similar calculation gives
\begin{align*}
D'_n(\la\otimes\mu)_n(c_i\otimes e_j)&=D'_n(\la_i\otimes\mu_j)(c_i\otimes e_j)\\
&=D'_n(\la_ic_i\otimes\mu_je_j)\\
&=d\la_ic_i\otimes\mu_j e_j+(-1)^i\la_ic_i\otimes\partial(\mu_je_j).
\end{align*}
Of course, we know that $d\la=\la d$ and $\partial \mu=\mu\partial$, which implies the result.
\end{exercise}

\begin{exercise} \leavevmode
Note that it suffices to prove the hint, since transitivity will finish the proof.
The proof of the hint is a routine, if long, computation.
\end{exercise}

\begin{exercise} \leavevmode
Suppose $\la:C_*\to C_*'$ and $\la':C_*'\to C_*$ with $\la\circ\la'\simeq1_{C'_*}$ and $\la'\circ\la\simeq1_{C_*}$.
Similarly define $\mu$ and $\mu'$.
Then \Cref{9.38} implies that
\[\la\otimes\mu:C_*\otimes E_*\to C'_*\otimes E'_*,\]
and similarly for $\la'\otimes\mu')$.
But
\[(\la\otimes\mu)\circ(\la'\otimes\mu')=(\la\la')\otimes(\mu\mu')\simeq 1_{C'_*}\otimes 1_{E'_*}=1_{C'_*\otimes E'_*}.\]
The same calculation holds for the other composition, which proves chain equivalence.
\end{exercise}

\begin{exercise} \leavevmode
Each $n$ (i.e., each $0\to S'_n\to S_n\to S''_n\to0$) works because $E_*$ is a chain complex, hence $E_n$ is free.
\end{exercise}

\begin{exercise} \leavevmode
For $n\ge1$, we know that $H_n(X)=0=H_n(Y)$.
Hence the K\"unneth formula implies that
\[H_n(X\times Y)\cong\sum_{i+j=n}H_i(X)\otimes H_j(Y)\oplus\sum_{p+q=n-1}\Tor(H_p(X),H_q(Y)).\]
But the first term is 0 since one of $i,j$ is at least 1, and thus one of $H_i(X),H_j(Y)$ is 0.
The second term is zero since the only way for $H_p(X)$ and $H_q(Y)$ to both be nonzero is if $p=q=0$, in which case both homology groups are free.
Hence the torsion $\Tor(H_0(X),H_0(Y))$ is zero in that case too.
\end{exercise}

\begin{exercise} \leavevmode
For path-connected $X$ and $Y$, we have
\[H_1(X\times Y)=H_0(X)\otimes H_1(Y)\oplus H_1(X)\otimes H_0(Y)\oplus\Tor(H_0(X),H_0(Y)).\]
But $H_0(X)=H_0(Y)=\ZZ$, and so using \Cref{9.27} gives us that the first two terms are $H_1(Y)$ and $H_1(X)$, respectively, while [Tor 2] implies that the last term is 0.
This gives the first equation.

For $H_2$, notice that the $\Tor$ terms have either $H_0(X)$ or $H_0(Y)$, so [Tor 2] implies that they are 0.
Hence
\[H_2(X\times Y)=[H_0(X)\otimes H_2(Y)]\oplus[H_1(X)\otimes H_1(Y)]\oplus[H_2(X)\otimes H_0(Y)].\]
Using $H_0(X)=H_0(Y)=\ZZ$ again gives the result.
\end{exercise}

\begin{exercise} \leavevmode
This splits into multiple cases and is slightly annoying.
We end up with the following:
\[
H_p(K\times\RR P^n)=
\begin{cases}
0 & p\ge n+2 \\
\ZZ\oplus\ZZ/2\ZZ & p=n+1,n~\text{odd} \\
\ZZ/2\ZZ & p=n+1,n~\text{even} \\
\ZZ\oplus\ZZ/2\ZZ & p=n,n~\text{odd} \\
\ZZ/2\ZZ\oplus\ZZ/2\ZZ & p=n,n~\text{even} \\
\ZZ/2\ZZ\oplus\ZZ/2\ZZ & 1<p<n \\
\ZZ/2\ZZ\oplus\ZZ/2\ZZ\oplus\ZZ/2\ZZ & p=1,p\ne n \\
\ZZ & p=0
\end{cases}    
\]
\end{exercise}

\begin{exercise} \leavevmode
We once again have many, many cases.
\[
H_p(\RR P^n\times S^m)=
\begin{cases}
\ZZ & p=0,m\ne0 \\
\ZZ\oplus\ZZ & p=m=0 \\
\ZZ/2\ZZ & p~\text{odd},p<\min(m,n) \\
\ZZ\oplus\ZZ/2\ZZ & m~\text{odd},p=m \\
\ZZ/2\ZZ & m~\text{odd},p~\text{odd between}~m~\text{and}~n \\
\ZZ & m~\text{odd},p~\text{even},p\le m+n \\
\ZZ & m~\text{even},p=m \\
\ZZ\oplus\ZZ/2\ZZ & m~\text{even},p~\text{odd between}~m~\text{and}~n \\
\ZZ & m~\text{even},p~\text{odd},p\le m+n \\
0 & \text{otherwise}
\end{cases}
\]
(Something like that, I can't quite read my work anymore.)
\end{exercise}

\begin{exercise} \leavevmode
This is the exact same idea, but I'll admit I didn't work it all out.
\end{exercise}

\begin{exercise} \leavevmode
It turns out that the machinery we have (i.e., fundamental groups) isn't sufficient to distinguish $S^1\vee S^2\vee S^3$ from $S^1\times S^2$, as they both have fundamental group $\ZZ$.
In fact, this seems to require cohomology (see \Cref{12.21}).
\end{exercise}

\begin{exercise} \leavevmode
We use the K\"unneth formula here.
Note that the homology groups of $S^1$ are all cyclic or zero, so the $\Tor$ terms are zero.
Hence
\[H_n(S^1\times S^1)=\sum_{i+j=n}H_i(S^1)\otimes H_j(S^1).\]
If $n>2$, then one of $H_i(S^1)$ and $H_j(S^1)$ is zero, so
\[H_n(S^1\times S^1)=0\quad n>2.\]
When $n=0$, then we have $i=0,j=0$, so
\[H_0(S^1\times S^1)=\ZZ\otimes\ZZ=\ZZ\]
If $n=1$, then we have $(i,j)=(0,1),(1,0)$, and so
\[H_1(S^1\times S^1)=\ZZ\otimes\ZZ\oplus\ZZ\otimes\ZZ=\ZZ\oplus\ZZ.\]
Finally, if $n=2$, then we only have to consider $(i,j)=(1,1)$, so that
\[H_2(S^1\times S^1)=\ZZ\otimes\ZZ=\ZZ.\]

Now recall that
\[H_n(K_1\vee K_2)\cong H_n(K_1)\oplus H_n(K_2),\]
so we know that
\[H_n(S^2\vee S^1\vee S^1)\cong H_n(S^2)\oplus H_n(S^1)\oplus H_n(S^1).\]
We know the homology groups of $S^1$ and $S^2$, and working them out gives the same homology groups as those of $S^1\times S^1$.

(Interestingly, the fundamental groups of these two spaces are different from one another, which one can show using Seifert--Van Kampen.
I wonder if Rotman mixed up this problem with \Cref{9.46}.)
\end{exercise}

\begin{exercise} \leavevmode
\begin{enumerate}
\item
This is straightforward using the fact that the homology of wedges is the direct sum of homology groups.
Hence both homology groups are $\ZZ$ when $n=0,3$, $\ZZ/2\ZZ$ when $n=1$, and 0 otherwise.

\item
According to a cursory search online, this requires universal coverings.

\item
This seems to be another mistake on Rotman's part, as he seems to have thought that $\RR P^3$ and $\RR P^2\vee S^3$ had different fundamental groups.
In fact, they both have $\ZZ/2\ZZ$ as their fundamental group, and so it is obvious that $\RR P^3\times\RR P^2$ and $(\RR P^2\vee S^3)\times\RR P^2$ have the same homology groups and fundamental group.
\end{enumerate}
\end{exercise}

\begin{exercise} \leavevmode
Since the homology groups of $S^1$ are all cyclic or zero, the Tor terms in the K\"unneth formula don't count.
Suppose that
\[H_n(T^{r-1})=\ZZ^{\binom{r-1}n}.\]
Note that this is true for $r=1$.
Then we have that
\[H_n(S^1\times T^{r-1})\cong\sum_{i+j=n}H_i(S^1)\otimes H_j(T^{r-1})=H_n(T^{r-1})\oplus H_{n-1}(T^{r-1}).\]
But of course we know that
\[\binom{r-1}{n}+\binom{r-1}{n-1}=\binom rn,\]
and so it follows that
\[H_n(T^r)=H_n(S^1\times T^{r-1})=\ZZ^{\binom rn}.\]
\end{exercise}

\end{document}