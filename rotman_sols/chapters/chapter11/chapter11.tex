\documentclass[../../solutions.tex]{subfiles}

\setcounter{section}{10}

\begin{document}
\section{Homotopy Groups}
\subsection{Function Spaces}
No exercises!

\subsection{Group Objects and Cogroup Objects}
\begin{exercise} \leavevmode
\begin{enumerate}
\item
By definition of a product, there is a unique morphism $\theta:(X,q_1,q_2)\to(C_1\times C_2,p_1,p_2)$ in $\cat C$ making the diagram commute, namely $\theta=(q_1,q_2)$.

\item
The objects are ordered triples $(X,k_1,k_2)$ where $X$ is a set and $k_i:C_i\to X$ are functions.
Morphisms $\theta:(X,k_1,k_2)\to(Y,\ell_1,\ell_2)$ are functions $\theta:X\to Y$ making the following commute:
\[
\begin{tikzcd}
& X \ar[dd,"\theta"] & \\
C_1 \ar[ur,"k_1"] \ar[dr,"\ell_1",swap] & & C_1 \ar[ul,"k_2",swap] \ar[dl,"\ell_2"] \\
& Y &
\end{tikzcd}
\]
\end{enumerate}
\end{exercise}

\begin{exercise} \leavevmode
We first tackle $\Ab$.

The map $\theta:X\to G_1\oplus G_2$ in the product diagram is given by $\theta(g)=(q_1g,q_2g)$.
Commutativity follows from the fact that $p_i(\theta(g))=q_i(g)$.
Uniqueness of $\theta$ follows from the fact that any other $\theta'$ must satisfy $\theta'(g)=(g_1,g_2)$ where $g_i=p_i(g_1,g_2)=q_i(g)$.
Hence $\theta'=\theta$.

The map $\eta:G_1\oplus G_2\to X$ in the coproduct diagram is given by $(g,h)\mapsto k_1(g)+k_2(h)$, where $+$ denotes the operation in the abelian group $X$.
We can easily check commutativity and uniqueness using the fact that $\eta$ must be a group homomorphism.

Now, for $\Grp$, note that the free product property on p.~173 is exactly the coproduct property.
The same argument as in the abelian case shows that direct product is the product in $\Grp$.
\end{exercise}

\begin{exercise} \leavevmode
\begin{enumerate}
\item
We will show this for $\Top_*$.
Suppose we have $((X,x),k_1,k_2)$.
It is obvious that the map $\theta:(A_1\vee A_2,*)\to(X,x)$, if it exists, must take $*$ to $x$, and $*\ne a_i\in j_i(A_i)$ to $k_i(a_i)$.
We need only show that this map $\theta$ is continuous.
(In contrast, the proof has already been completed for $\Set_*$; commutativity of the relevant diagram is obvious from the definition of $\theta$.)

Suppose $U\subseteq X$ is closed.
Note that $\theta^{-1}(U)\cap A_i=k_i^{-1}(U)$.
(This statement is clear if $*\not\in U$.
If $*\in U$, then
\[\theta^{-1}(U)\cap A_i=(\theta^{-1}(U\setminus\{x\})\cap A_i)\cup\{*\}=k_i^{-1}(U\setminus\{x\})\cup\{a_i\}=k_i^{-1}(U),\]
which proves the statement anyway.)
The definition of the topology of the wedge (see Example~8.9) implies that $\theta^{-1}(U)$ is closed.
Hence $\theta$ is continuous, completing the proof.

\item
Call this subset $S$.
The map $f:A_1\vee A_2\to S$ which takes $a\in A_i$ (or, more accurately, $a\in j_i(A_i)$) to $(a,a_2)$ if $i=1$ and to $(a_1,a)$ if $i=2$ is continuous by the previous argument.
It is clearly bijective and closed, since a closed set $F$ in $A_1\vee A_2$ is still closed in $A_1\times A_2$.
Thus it is a homeomorphism.
\end{enumerate}
\end{exercise}

\begin{exercise} \leavevmode
Commutativity follows from the interchanging of $C_1$ and $C_2$ in the definition.
To see associativity, consider the following diagram:
\[
\begin{tikzcd}
& & (C_1\times C_2)\times C_3\ar[dl,bend right]\ar[ddrr,bend left] & & \\
& C_1\times C_2\ar[dl]\ar[dr] & & & \\
C_1 & & C_2 & & C_3 \\
& & X\ar[ull,bend left=5]\ar[u]\ar[urr,bend right=5]\ar[uul,bend left=10] & &
\end{tikzcd}
\]
There is a unique map $X\to(C_1\times C_2)\times C_3$ making this diagram commute.

Now define $p_1:(C_1\times C_2)\times C_3\to C_1$ to be the composition of the red arrows below.
Furthermore, the product property of $C_2\times C_3$ implies the existence of the following blue and green arrows:
\[
\begin{tikzcd}
& & (C_1\times C_2)\times C_3\ar[dl,bend right,red]\ar[ddrr,bend left]\ar[dr,blue] & & \\
& C_1\times C_2\ar[dl,red]\ar[dr] & & C_2\times C_3\ar[dr,green]\ar[dl,green] & \\
C_1 & & C_2 & & C_3 \\
& & X\ar[ull,bend left=5]\ar[u]\ar[urr,bend right=5]\ar[uul,bend left=10] & &
\end{tikzcd}
\]
Let $p_2$ be the blue arrow.
The fact that there is still the same unique map $X\to(C_1\times C_2)\times C_2$ making this commute, then, implies that $(C_1\times C_2)\times C_3$ is the product of $C_1$ and $C_2\times C_3$, thus proving associativity.
\end{exercise}

\begin{exercise} \leavevmode
\begin{enumerate}
\item
We would like to find $f_1\times f_2$ making the following commute:
\[
\begin{tikzcd}[column sep=large,row sep=large]
C_2\ar[r,"f_2"]\ar[d,leftarrow] & D_2\ar[d,leftarrow] \\
C_1\times C_2\ar[r,"f_1\times f_2"]\ar[d] & D_1\times D_2\ar[d]\\
C_1\ar[r,"f_1"] & D_1
\end{tikzcd}
\]
But the existence of maps $C_1\times C_2\to C_i\to D_i$ implies, by the product property of $D_1\times D_2$, a unique map $f_1\times f_2$ into $D_1\times D_2$ making the diagram commute.

\item
Same idea.
\end{enumerate}
\end{exercise}

\begin{exercise} \leavevmode
\begin{enumerate}
\item
Note that $\Dlt_X$ is the unique map making the red part of the diagram commute, while $q_1\times q_2$ is the unique map making the blue part commute:
\[
\begin{tikzcd}
& D_1\times D_2\ar[dl,blue]\ar[dr,blue] & \\
D_1 & & D_2 \\
& X\times X\ar[uu,dashed,blue,"q_1\times q_2"]\ar[dl,red]\ar[dr,red] & \\
X\ar[uu,blue] & & X\ar[uu,blue] \\
& X\ar[uu,red,dashed,"\Dlt_X"]\ar[ul,red]\ar[ur,red] &
\end{tikzcd}
\]
But of course, since the maps $q_i\circ 1_X=X\to X\to D_i$ are equal to simply the maps $q_i:X\to D_i$, we know that the unique map $X\mapsto D_1\times D_2$ making this entire diagram commute is $(q_1,q_2)$.
Uniqueness implies that $(q_1,q_2)$ must be equal to $(q_1\times q_2)\Dlt_X$.

\item
This is the same idea.

\item
We already showed the first statement.
For the second, notice that $\nabla_B(f\times g)=(f,g)$.
But $(f,g)\Dlt_A(a)=(f(a),g(a))=(f+g)(a)$ because $A\oplus B=A\amalg B$.
\end{enumerate}
\end{exercise}

\begin{exercise} \leavevmode
\begin{enumerate}
\item
Everything follows from the hint, except that we must verify that $1_{X\times Z}$ and $\theta\la$ complete the given diagram.
Commutativity of the left triangle is obvious in both cases.
To see that $q1_{X\times Z}=t$, note that $Z$ being terminal implies that $q=t$.
To show that $q\theta\la=t$, note that $q\theta\la:X\times Z\to X\to X\times Z\to Z$.
Thus $Z$ being terminal again implies the result.

Now $\theta$ and $\la$ are inverses, and so $X\times Z$ and $X$ are equivalent.

\item
This is the dualized version of the previous part.
\end{enumerate}
\end{exercise}

\begin{exercise} \leavevmode
We will use the definition of a group object.
If $G$ is a group object, then the terminal object $Z$ is the one-element group $\{z\}$.
With standard notation, let $e\in G$ be $\ep(z)$.
Note that $\mu(g,e)=\mu(e,g)=g$.
Now the fact that $\mu$ is a homomorphism implies that
\[\mu(g_1,g_2)=\mu(g_1,e)\mu(e,g_2)=g_1g_2,\]
so that $\mu$ must be the multiplication operation of $G$.
Using this, we can show that $\eta$ is indeed the inverse operation: $\eta(g)=g^{-1}$.
In particular, we know that
\[g\cdot\eta(g)=(\mu\circ(1,\eta))(g)=e\]
for any $g$.

Now we know that $\eta$ must be a homomorphism.
Thus
\[g^{-1}h^{-1}=\eta(g)\eta(h)=\eta(gh)=(gh)^{-1}=h^{-1}g^{-1}.\]
Obviously this proves that $G$ is abelian.
\end{exercise}

\begin{exercise} \leavevmode
The initial object $A$ in both cases is the empty set.
The existence of a morphism $e:C\to A$ implies that $C=\emptyset$.
It is easy to verify that $\emptyset$ is a cogroup object, which completes the proof.
\end{exercise}

\begin{exercise} \leavevmode
This time we use the co-identity property.
Let $x\in C$.
Then $m(x)$ is either in the first coordinate or the second (or it is the basepoint $*$).
Thus either $1\amalg e$ or $e\amalg 1$ will take $m(x)$ to $e(C)=*\in A\subseteq C\amalg A$.

Now we compare this with the maps in the co-identity triangles.
In particular, if $x\ne*$ is an element of $C$, then the maps $C\to C\amalg A$ and $C\to A\amalg C$ take $x$ to itself, not $x\mapsto*$.
This contradicts commutativity, so $C=\{*\}$.
\end{exercise}

\begin{exercise} \leavevmode
\begin{enumerate}
\item
We prove this only for group objects; the result for cogroup objects simply involves oppositely oriented arrows.

Identities follow from the commutativity of the following:
\[
\begin{tikzcd}
G\times G\ar[d,"\mu",swap]\ar[r,"1_G\times1_G"] & G\times G\ar[d,"\mu"] \\
G\ar[r,"1_G",swap] & G.
\end{tikzcd}
\]
Associativity follows from associativity of $\cat C$.
Composition follows from commutativity of the following:
\[
\begin{tikzcd}
G\times G\ar[r,"f\times f"]\ar[d] & H\times H\ar[r,"g\times g"]\ar[d] & J\times J\ar[d] \\
G\ar[r,"f",swap] & H\ar[r,"g",swap] & J,
\end{tikzcd}
\]
as well as the fact that
\[(g\times g)\circ(f\times f)=gf\circ gf.\]

\item
The first statement follows from Theorem~11.4.

For the second statement, we must show that
\[f_*(M_X^G(p,q))=M_X^H(f_*(p),f_*(q)),\]
where $p,q\in\Hom(X,G)$.
But we know that
\[M_X^G(p,q)=\mu^G(p,q)\in\Hom(X,G),\]
so that
\[f\circ M_X^G(p,q):x\mapsto f(\mu^G(p(x),q(x))).\]
On the other hand we know that
\[M_X^H(f_*(p),f_*(q))=\mu^H(fp,fq)\]
is the map taking
\[x\mapsto\mu^H(fp(x),fq(x)).\]
It thus suffices to show that
\[f(\mu^G(p(x),q(x))=\mu^H(fp(x),fq(x)).)\]
But following $(p(x),q(x))\in G\times G$ in the special diagram implies the result.
\end{enumerate}
\end{exercise}

\begin{exercise} \leavevmode
That every abelian group is a group object is clear by \Cref{11.8}.
To see that it is a cogroup object, define $e:g\mapsto a$ where $A=\{a\}$, $m:g\mapsto(g,g)$, and $h:g\mapsto -g$.
The axioms are easy to check.
\end{exercise}

\begin{exercise} \leavevmode
We will show that $\Hom(F,-)$ takes values in groups, where $F$ is a finitely generated free group.
Let $\{x_1,\dots,x_n\}$ be a basis for $F$.
Now consider the following function $P_G:\Hom(F,G)\times\Hom(F,G)\to\Hom(F,G)$:
\[P_G:(f,g)\mapsto (x_i\mapsto f(x_i)g(x_i)).\]
We will show that this gives $\Hom(F,G)$ a group structure.

Note that an element of $\Hom(F,G)$ is completely determined by where it sends each $x_i$.
Thus $P_G$ is well-defined.
Now suppose $\phi:G\to H$, so that $\phi_*:\Hom(F,G)\to\Hom(F,H)$.
Then we need to show that
\[\phi_*(P_G(f,g))=P_H(\phi_*(f),\phi_*(g)).\]
The left side takes
\[x_i\mapsto f(x_i)g(x_i)\mapsto\phi(f(x_i)g(x_i)).\]
On the other hand, the right side takes
\[x_i\mapsto(\phi f(x_i),\phi g(x_i))\mapsto\phi f(x_i)\cdot\phi g(x_i).\]
But these are equal because $\phi$ is a homeomorphism.
\end{exercise}

\begin{exercise} \leavevmode
This is easy; we can even use the same functions/morphisms.
\end{exercise}

\subsection{Loop Space and Suspension}
\begin{exercise} \leavevmode
We would like to show that the following commutes for all $f:A'\to A$:
\[
\begin{tikzcd}
\Hom(A\otimes Y,C)\ar[r,"(f\otimes 1)^*"]\ar[d,"\tau_{AC}",swap] & \Hom(A'\otimes Y,C)\ar[d,"\tau_{A'C}"] \\
\Hom(A,\Hom(Y,C))\ar[r,"f^*",swap] & \Hom(A',\Hom(Y,C)),
\end{tikzcd}
\]
where $\tau_{AC}(\phi)=\phi^\#$.

First, we look at the lower path $f^*\circ\tau_{AC}$.
If $\phi:A\otimes Y\to C$ takes $(a,y)$ to $\phi(a,y)$, then $\tau_{AC}(\phi)=\phi^\#$ takes $a\in A$ to the map $\phi_a\in\Hom(Y,C)$ defined by $\phi_a(y)=\phi(a,y)$.
Thus $\phi^\#f:A'\to\Hom(Y,C)$ is defined by
\[\phi^\#f:f'\mapsto a\mapsto\phi_a,\]
where $a=f(a')$.

On the other hand, the upper path takes $\phi$ to the map
\[[\phi(f\otimes 1)]^\#:A'\to\Hom(Y,C)\]
defined by taking
\[a'\mapsto[\psi_{a'}:y\mapsto\phi(f(a'),1(y))].\]
Of course, these are the same since $f(a')=a$, proving commutativity.

The second square is similar.
\end{exercise}

\begin{exercise} \leavevmode
We'll show the first square, namely commutativity of
\[
\begin{tikzcd}
\Hom(GA,C)\ar[r,"(Gf)^*"]\ar[d,"\tau_{AC}",swap] & \Hom(GA',C)\ar[d,"\tau_{A'C}"] \\
\Hom(A,C)\ar[r,"f^*",swap] & \Hom(A',C),
\end{tikzcd}
\]
where $\tau_{AC}$ takes $\phi:GA\to C$ to $\phi|_A$.
But commutativity is obvious, since both paths end up taking $\phi$ to $\phi f:A'\to C$, where the maps are as sets.
\end{exercise}

\begin{exercise} \leavevmode
We will show this for $G$; the statement for $F$ amounts to dualizing the following argument.

Adjointness implies that there is a bijection $\tau_{AC}$ between $\Hom(FA,C)$ and $\Hom(A,GC)$.
Hence consider the two diagrams below; the left one is in $\cat C$ and the right one is in $\cat A$:
\[
\begin{tikzcd}
& FX\ar[dl,"\tilde q_1",swap]\ar[dd,dashed,"\tilde\theta"]\ar[dr,"\tilde q_2"] & & & & X\ar[dl,"q_1",swap]\ar[dd,dashed,"\theta"]\ar[dr,"q_2"] & \\
C_1 & & C_2 & & GC_1 & & GC_2 \\
& C\ar[ul,"p_1"]\ar[ur,"p_2",swap] & & & & GC\ar[ul,"Gp_1"]\ar[ur,"Gp_2",swap] &
\end{tikzcd}
\]
Here, we let $\theta:X\to GC$ be the morphism corresponding to $\tilde\theta$ under $\tau_{XC}$, and we let $\tilde q_i$ be the morphism corresponding to $q_i$ under the bijection $\tau_{XC_i}$.
We claim that $\theta$ completes the diagram on the right.
To see this, use the fact that $(Gg)_*\tau=\tau g_*$.
Now if $g=p_1$, then
\[\tau g_*(\tilde\theta)=\tau(p_1\tilde\theta)=\tau(\tilde q_1)=q_1.\]

Now we must show that $\theta$ is the unique map making the product diagram on the right commute.
Suppose $\eta$ were another possible map.
Define $\tilde\eta=\tau^{-1}(\eta)$.
We will show that $\tilde\eta=\tilde\theta$, so the product diagram in $\cat C$ and the fact that $\tau$ is a bijection will imply that $\eta=\theta$.

But notice that
\[((Gp_1)_*\circ\tau)(\tilde\eta)=(Gp_1)*(\eta)=(Gp_1)*(\theta)=((Gp_1)_*\circ\tau)(\tilde\theta)=(\tau\circ(p_1)_*)(\tilde\theta).\]
But naturality implies that
\[((Gp_1)_*\circ\tau)(\tilde\eta)=(\tau\circ(p_1)_*)(\tilde\eta).\]
It thus follows that
\[\tau(p_1\tilde\theta)=\tau(p_1\tilde\eta).\]
Since $\tau$ is a bijection, it follows that
\[p_1\tilde\theta=p_1\tilde\eta=\tilde q_1.\]
Thus $\tilde\eta$ completes the product diagram in $\cat C$, so that $\tilde\eta=\tilde\theta$, proving the result.
\end{exercise}

\begin{exercise} \leavevmode
This is exactly stereographic projection (or the reverse of it).
\end{exercise}

\begin{exercise} \leavevmode
Note that $J^n$ is homeomorphic to $\II^n\setminus\{N\}$, where $N$ is some fixed point.
Hence $(J^n)^\infty\approx\II^n$.
\end{exercise}

\begin{exercise} \leavevmode
Consider the map taking $A$ to $\infty$ and taking $x\in X\setminus A$ to itself.
This is obviously a homeomorphism.
\end{exercise}

\begin{exercise} \leavevmode
We have the following:
\[S^m\wedge S^n=(\RR^m)^\infty\wedge(\RR^n)^\infty=(\RR^{m+n})^\infty=S^{m+n}.\]
\end{exercise}

\begin{exercise} \leavevmode
We have
\[\II^n\wedge\II=(J^n)^\infty\wedge J^\infty=(J^{n+1})^\infty=\II^{n+1}.\]
\end{exercise}

\subsection{Homotopy Groups}
\begin{exercise} \leavevmode
Let $F:\beta\simeq y_0$ be a homotopy.
We would like to show that
\begin{align*}
\beta_*:\pi_n(X,x_0)&\to\pi_n(Y,y_0)\\
[\alpha]&\mapsto[\beta\circ\alpha].
\end{align*}
To do so, we must show that $\beta\circ\alpha$ is nullhomotopic $\rel\dot\II^n$.
Consider the map
\begin{align*}
F\circ(\alpha\times\id_\II):\II^n\times\II&\to Y\\
(u,t)&\mapsto F(\alpha(u),t).
\end{align*}
Obviously, this is a homotopy between $\beta(\alpha(u))$ and the constant map at $y_0$.
To see that this is $\rel\dot\II^n$, simply note that $u\in\dot\II^n$ implies that $F(\alpha(u),t)=F(x_0,t)=y_0$, since $\alpha$ and $F$ are pointed maps.
\end{exercise}

\begin{exercise} \leavevmode
We have the following chain of equalities (note that some equalities are up to isomorphism or homotopy, depending on the category):
\begin{align*}
\pi_n(X\times Y)&=[S^n,X\times Y] \\
&=\Omega(X\times Y) \\
&=\Omega X\times\Omega Y \\
&=[S^n,X]\times[S^n,Y] \\
&=[S^n,X]\oplus[S^n,Y]=\pi_n(X)\oplus\pi_n(Y).
\end{align*}
Since $\pi_n(S^1)=0$, it follows that $\pi_n(T)=\pi_n(S^1)\times\pi_n(S^1)=0$.
\end{exercise}

\begin{exercise} \leavevmode
This follows from Theorem~11.29 and the fact that $S^n$ covers $\RR P^n$.
\end{exercise}

\begin{exercise} \leavevmode
Note that Theorem~10.54(i) applies because locally path-connected and contractible implies connected.
Thus $X$ is a covering space for $X/G$, and so Theorem~11.29 implies that $\pi_n(X)\cong\pi_n(X/G)$.
But $\pi_n(X)=0$ for $n\ge2$ since $X$ is contractible.
\end{exercise}

\begin{exercise} \leavevmode
This follows almost immediately from the hint.
To see that $*$ and $\circ$ coincide, note that
\[f*g=(f\circ e)*(e\circ g)=(f*e)\circ(e*g)=f\circ g.\]
To see commutativity, we need only check that $f*g=g\circ f$.
But this follows because
\[f*g=(e\circ f)*(g\circ e)=(e*g)\circ(f*e)=g\circ f,\]
as desired.
\end{exercise}

\begin{exercise} \leavevmode
\begin{enumerate}
\item
We follow the path laid out in the hints.
First, note that, if $q\in Q$, then
\[\mu(f,e)(q)=\mu(f(q),p_0)=(\mu(-,p_0)\circ f)(q).\]
Thus it follows that
\[[f]*[e]=[\mu(f,e)]=[\mu(-,p_0)\circ f]=[1_P\circ f]=[f],\]
where we use the property of an $H$-space.
Similarly, we can show that $[e]*[f]=[f]$.

Now we must show that $[f]\circ[e]=[f]$.
But $[f]\circ[e]=[(f,e)m]$, and $(f,e)m$ takes $q$ to $f(q)$ if $m(q)$ is in the first coordinate of $Q\vee Q$, and takes $q$ to $p_0$ if $m(q)$ is in the second coordinate.
Letting $q_1$ be as in the definition of an $H'$-group, i.e., letting $q_1$ be the projection to the first coordinate, we see that $q_1m$ takes $q$ to $q$ if $m(q)$ is in the first coordinate and takes $q$ to $q_0$ otherwise.
Thus $fq_1m$ takes $q$ to either $f(q)$ or $p_0$, depending on the coordinate of $q$, and so it follows that $fq_1m=(f,e)m$.
But of course $q_1m\simeq 1_Q$, from which the conclusion follows.

To show the second condition of \Cref{11.27}, note first that
\[([f]\circ[h])*([g]\circ[j])=[\mu((f,h)m,(g,j)m)].\]
The map on the right side takes $q$ to either $\mu(fq,gq)$ or $\mu(hq,jq)$, depending on where $m(q)$ is in the first or second coordinate of $Q\vee Q$.
On the other hand, we have
\[([f]*[g])\circ([h]*[j])=[(\mu(f,g),\mu(h,j))m],\]
which takes
\[q\mapsto m(q)\mapsto\begin{cases}\mu(fq,gq)\\\mu(hq,jq)\end{cases},\]
which is the exact same.
Thus condition (ii) is satisfied, and the previous exercise proves the result.

\item
Note that $[\Sig^2X,Y]=[\Sig X,\Omega Y]$ since $\Sig$ and $\Omega$ are adjoint functors.
Now since $\Sig X$ is an $H'$-group and $\Omega Y$ is an $H$-group in the category, hence an $H$-space, it follows from the previous part that $[\Sig X,\Omega Y]$ is abelian.
\end{enumerate}
\end{exercise}

\begin{exercise} \leavevmode
First, we must show that this is well-defined.
Suppose $F:f\simeq g\rel\{s_n\}$.
We claim that $\Sig f\simeq\Sig g$ with the map $G:(a,b,t)\mapsto (F(a,t),b)$.
But this is easy to verify because $G(a,b,0)=(F(a,0),b)=(f(a),b)=(\Sig f)(a,b)$, and similarly for $G(a,b,1)$.

Now, we will show that it is a homomorphism.
Let $m_n:S^n\to S^n\vee S^n$ be comultiplication.
Then $[f][g]=[(f,g)m]$.
We would like to show that
\[[\Sig((f,g)m_n)]=[\Sig f][\Sig g]=[(\Sig f,\Sig g)m_{n+1}].\]
To see this, note that the left side takes $(a,b)$ to $(f(a),b)$ or $(g(a),b)$, depending on which $S^n\wedge S^1$-component $m_{n+1}(a,b)$ belongs to in $(S^n\wedge S^1)\vee(S^n\wedge S^1)$.
On the other hand, we know that $\Sig((f,g)m_n)$ takes $(a,b)$ to $(((f,g)m_n)(a),b)$.
This first coordinate $((f,g)m_n)(a)$ is $f(a)$ or $g(a)$, depending on which ``half'' $m_n$ takes $a$ to.
Using a rotation to make sure the two halves which $m_n$ and $m_{n+1}$ determine line up (after projecting $S^{n+1}$ down to the equator, which is $S^n$), it is easy to show that these maps are homotopic.
\end{exercise}

\begin{exercise} \leavevmode
Any map $Y\to X$ is homotopic to some simplicial approximation $Sd^qL\to K$.
Obviously, there are only countably many simplicial approximations, since there are only finitely many vertices of each $\Sd^qL$ and of $K$.
Hence $[Y,X]$ is countable.
Thus $\pi_n(X)=[S^n,X]$ must be countable.
\end{exercise}

\subsection{Exact Sequences}
\begin{exercise} \leavevmode
This is the exact same argument as part (ii) of \Cref{5.14}.
\end{exercise}

\begin{exercise} \leavevmode
The same diagram chase remark applies, just changing $H$ to $\pi$.
\end{exercise}

\begin{exercise} \leavevmode
We have the following long exact sequence:
\[
\begin{tikzcd}[column sep=small]
\dots\ar[r] & \pi_{n+1}(X,X)\ar[r,"d"] & \pi_n(X)\ar[r,"\id"] & \pi_n(X)\ar[r,"j_*"] & \pi_n(X,X)\ar[r,"d"] & \pi_{n-1}(X)\ar[r,"\id"] & \dots
\end{tikzcd}
\]
Now we know that $\ker j_*=\im\id=\pi_n(X)$, so that $\im j_*=0$.
Hence $\ker d=0$.
But $\im d=\ker\id=0$, and so $\ker d=\pi_n(X,X)$.
The result now follows.
\end{exercise}

\subsection*{Fibrations}
\begin{exercise} \leavevmode

\end{exercise}

\end{document}