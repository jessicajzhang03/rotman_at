\documentclass[../../solutions.tex]{subfiles}

\setcounter{section}{10}

\begin{document}
\section{Homotopy Groups}
\subsection{Function Spaces}
No exercises!

\subsection{Group Objects and Cogroup Objects}
\begin{exercise} \leavevmode
\begin{enumerate}
\item
By definition of a product, there is a unique morphism $\theta:(X,q_1,q_2)\to(C_1\times C_2,p_1,p_2)$ in $\cat C$ making the diagram commute, namely $\theta=(q_1,q_2)$.

\item
The objects are ordered triples $(X,k_1,k_2)$ where $X$ is a set and $k_i:C_i\to X$ are functions.
Morphisms $\theta:(X,k_1,k_2)\to(Y,\ell_1,\ell_2)$ are functions $\theta:X\to Y$ making the following commute:
\[
\begin{tikzcd}
& X \ar[dd,"\theta"] & \\
C_1 \ar[ur,"k_1"] \ar[dr,"\ell_1",swap] & & C_1 \ar[ul,"k_2",swap] \ar[dl,"\ell_2"] \\
& Y &
\end{tikzcd}
\]
\end{enumerate}
\end{exercise}

\begin{exercise} \leavevmode
We first tackle $\Ab$.

The map $\theta:X\to G_1\oplus G_2$ in the product diagram is given by $\theta(g)=(q_1g,q_2g)$.
Commutativity follows from the fact that $p_i(\theta(g))=q_i(g)$.
Uniqueness of $\theta$ follows from the fact that any other $\theta'$ must satisfy $\theta'(g)=(g_1,g_2)$ where $g_i=p_i(g_1,g_2)=q_i(g)$.
Hence $\theta'=\theta$.

The map $\eta:G_1\oplus G_2\to X$ in the coproduct diagram is given by $(g,h)\mapsto k_1(g)+k_2(h)$, where $+$ denotes the operation in the abelian group $X$.
We can easily check commutativity and uniqueness using the fact that $\eta$ must be a group homomorphism.

Now, for $\Grp$, note that the free product property on p.~173 is exactly the coproduct property.
The same argument as in the abelian case shows that direct product is the product in $\Grp$.
\end{exercise}

\begin{exercise} \leavevmode
\begin{enumerate}
\item
We will show this for $\Top_*$.
Suppose we have $((X,x),k_1,k_2)$.
It is obvious that the map $\theta:(A_1\vee A_2,*)\to(X,x)$, if it exists, must take $*$ to $x$, and $*\ne a_i\in j_i(A_i)$ to $k_i(a_i)$.
We need only show that this map $\theta$ is continuous.
(In contrast, the proof has already been completed for $\Set_*$; commutativity of the relevant diagram is obvious from the definition of $\theta$.)

Suppose $U\subseteq X$ is closed.
Note that $\theta^{-1}(U)\cap A_i=k_i^{-1}(U)$.
(This statement is clear if $*\not\in U$.
If $*\in U$, then
\[\theta^{-1}(U)\cap A_i=(\theta^{-1}(U\setminus\{x\})\cap A_i)\cup\{*\}=k_i^{-1}(U\setminus\{x\})\cup\{a_i\}=k_i^{-1}(U),\]
which proves the statement anyway.)
The definition of the topology of the wedge (see Example~8.9) implies that $\theta^{-1}(U)$ is closed.
Hence $\theta$ is continuous, completing the proof.

\item
Call this subset $S$.
The map $f:A_1\vee A_2\to S$ which takes $a\in A_i$ (or, more accurately, $a\in j_i(A_i)$) to $(a,a_2)$ if $i=1$ and to $(a_1,a)$ if $i=2$ is continuous by the previous argument.
It is clearly bijective and closed, since a closed set $F$ in $A_1\vee A_2$ is still closed in $A_1\times A_2$.
Thus it is a homeomorphism.
\end{enumerate}
\end{exercise}

\begin{exercise} \leavevmode
Commutativity follows from the interchanging of $C_1$ and $C_2$ in the definition.
To see associativity, consider the following diagram:
\[
\begin{tikzcd}
& & (C_1\times C_2)\times C_3\ar[dl,bend right]\ar[ddrr,bend left] & & \\
& C_1\times C_2\ar[dl]\ar[dr] & & & \\
C_1 & & C_2 & & C_3 \\
& & X\ar[ull,bend left=5]\ar[u]\ar[urr,bend right=5]\ar[uul,bend left=10] & &
\end{tikzcd}
\]
There is a unique map $X\to(C_1\times C_2)\times C_3$ making this diagram commute.

Now define $p_1:(C_1\times C_2)\times C_3\to C_1$ to be the composition of the red arrows below.
Furthermore, the product property of $C_2\times C_3$ implies the existence of the following blue and green arrows:
\[
\begin{tikzcd}
& & (C_1\times C_2)\times C_3\ar[dl,bend right,red]\ar[ddrr,bend left]\ar[dr,blue] & & \\
& C_1\times C_2\ar[dl,red]\ar[dr] & & C_2\times C_3\ar[dr,green]\ar[dl,green] & \\
C_1 & & C_2 & & C_3 \\
& & X\ar[ull,bend left=5]\ar[u]\ar[urr,bend right=5]\ar[uul,bend left=10] & &
\end{tikzcd}
\]
Let $p_2$ be the blue arrow.
The fact that there is still the same unique map $X\to(C_1\times C_2)\times C_2$ making this commute, then, implies that $(C_1\times C_2)\times C_3$ is the product of $C_1$ and $C_2\times C_3$, thus proving associativity.
\end{exercise}

\begin{exercise} \leavevmode
\begin{enumerate}
\item
We would like to find $f_1\times f_2$ making the following commute:
\[
\begin{tikzcd}[column sep=large,row sep=large]
C_2\ar[r,"f_2"]\ar[d,leftarrow] & D_2\ar[d,leftarrow] \\
C_1\times C_2\ar[r,"f_1\times f_2"]\ar[d] & D_1\times D_2\ar[d]\\
C_1\ar[r,"f_1"] & D_1
\end{tikzcd}
\]
But the existence of maps $C_1\times C_2\to C_i\to D_i$ implies, by the product property of $D_1\times D_2$, a unique map $f_1\times f_2$ into $D_1\times D_2$ making the diagram commute.

\item
Same idea.
\end{enumerate}
\end{exercise}

\begin{exercise} \leavevmode
\begin{enumerate}
\item
Note that $\Dlt_X$ is the unique map making the red part of the diagram commute, while $q_1\times q_2$ is the unique map making the blue part commute:
\[
\begin{tikzcd}
& D_1\times D_2\ar[dl,blue]\ar[dr,blue] & \\
D_1 & & D_2 \\
& X\times X\ar[uu,dashed,blue,"q_1\times q_2"]\ar[dl,red]\ar[dr,red] & \\
X\ar[uu,blue] & & X\ar[uu,blue] \\
& X\ar[uu,red,dashed,"\Dlt_X"]\ar[ul,red]\ar[ur,red] &
\end{tikzcd}
\]
But of course, since the maps $q_i\circ 1_X=X\to X\to D_i$ are equal to simply the maps $q_i:X\to D_i$, we know that the unique map $X\mapsto D_1\times D_2$ making this entire diagram commute is $(q_1,q_2)$.
Uniqueness implies that $(q_1,q_2)$ must be equal to $(q_1\times q_2)\Dlt_X$.

\item
This is the same idea.

\item
We already showed the first statement.
For the second, notice that $\nabla_B(f\times g)=(f,g)$.
But $(f,g)\Dlt_A(a)=(f(a),g(a))=(f+g)(a)$ because $A\oplus B=A\amalg B$.
\end{enumerate}
\end{exercise}

\begin{exercise} \leavevmode
\begin{enumerate}
\item
Everything follows from the hint, except that we must verify that $1_{X\times Z}$ and $\theta\la$ complete the given diagram.
Commutativity of the left triangle is obvious in both cases.
To see that $q1_{X\times Z}=t$, note that $Z$ being terminal implies that $q=t$.
To show that $q\theta\la=t$, note that $q\theta\la:X\times Z\to X\to X\times Z\to Z$.
Thus $Z$ being terminal again implies the result.

Now $\theta$ and $\la$ are inverses, and so $X\times Z$ and $X$ are equivalent.

\item
This is the dualized version of the previous part.
\end{enumerate}
\end{exercise}

\end{document}