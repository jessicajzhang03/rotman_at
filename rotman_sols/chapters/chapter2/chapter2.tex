\documentclass[../../solutions.tex]{subfiles} 

\setcounter{section}{1}

\begin{document}

\section[Simplexes]{Simplexes\footnote{I usually use \emph{simplices} as the plural of simplex, but Rotman doesn't; no matter.}}
\subsection{Affine Spaces}
\begin{exercise} \leavevmode
Note that there is a maximal affine independent subset $S$ of $A$. This is directly implied by the fact that any set of greater than $n+1$ elements is not affine independent. Hence we can take an affine independent subset of $A$ with maximum size (because the empty set is affine independent). 

Write $S=\{p_0,\dots,p_m\}$. Then let $p_{m+1}\in A\setminus S$. By maximality of $S$, we know that $S\cup\{p\}$ is not affine independent. Hence there exist $s_i$ not all 0 such that \[\sum_{i=0}^{m+1}s_ip_i=0,\quad\sum_{i=0}^{m+1}s_i=0.\] Note that the second equation implies $\sum_{i=0}^ms_i\ne0$ for some $i<m+1$. It follows then that \[\sum_{i=0}^m\left(\frac{s_i}{\sum_{i=0}^ms_i}p_i\right)=p_{m+1}.\] But we know that \[\sum_{i=0}^m\frac{s_i}{\sum_{i=0}^ms_i}=1,\] and so it follows that $p_{m+1}$ is in fact in the affine span of $S$. 
\end{exercise} 

\begin{exercise} \leavevmode
Let $\phi$ be the isomorphism from $\RR^n$ to a subset of $\RR^k$. Suppose $A\subseteq\RR^n$ is an affine set containing $X$. Then $\phi(X)\subseteq\phi(A)\subseteq\RR^k$. 

Moreover, we claim that $\phi(A)$ is affine. After all, for any $\phi(x),\phi(x')\in\phi(A)$ and any $t\in\RR$, the point $t\phi(x)+(1-t)\phi(x')=\phi(tx+(1-t)x')\in\phi(A)$ because $A$ is affine. 

This implies that the intersection of all affine sets in $\RR^n$ containing $X$ must contain the intersection of all affine sets in $\phi(\RR^n)$ containing $\phi(X)$. Because $\phi$ is an isomorphism, using $\phi^{-1}$ gives the reverse inclusion. Thus the affine set spanned by $X$ in $\RR^n$ is precisely the same as that spanned by $X$ in $\RR^k$. 
\end{exercise} 

\begin{exercise} \leavevmode
This is evident in the case $n=0$. 

Suppose it is true for $n-1$ and consider the canonical injection $\iota:S^{n-1}\hookrightarrow S^n$ which takes $(x_0,\dots,x_{n-1})$ to $(x_1,\dots,x_{n-1},0)$. It is obvious that we can pick $n+1$ affine independent points $p_0,\dots,p_n$ in this embedding. 

Now consider the point $p_{n+1}=(0,\dots,0,1)\in S^n$. Notice that the last coordinate of each $p_i$ for $i\ne n+1$ is zero. Thus suppose we have $s_i$ with $\sum s_ip_i=0$ and $\sum s_i=0$. Then $s_{n+1}=0$, and so this reduces to the $n-1$ case. Affine independence of $\{p_0,\dots,p_n\}$ proves the result. 
\end{exercise} 

\subsection{Affine Maps}
\begin{exercise} \leavevmode
Consider the map $T'(x)=T(x)-T(0)$. We claim that $T'$ is a linear map. 

Observe that $S=\{e_i\}\cup\{0\}$ spans $\RR^n$. Thus we can write any point as the affine sum of elements of $S$. Note that the coefficient of the zero vector is flexible, and so we have effectively no restrictions on the sum of the coefficients. 

Consider arbitrary elements $\sum r_ie_i+r\cdot0$ and $\sum s_ie_i+s\cdot0$ in $\RR^n$, where $r=1-\sum r_i$ and similarly for $s$. Let $R,S\in\RR$. Then note that \begin{align*}T'\left(R\sum r_ie_i+S\sum s_ie_i\right)&=T'\left(\sum(Rr_i+Ss_i)e_i\right)\\&= T\left(\sum(Rr_i+Ss_i)e_i+\left(1-\sum(Rr_i+Ss_i)\right)\cdot0\right)-T(0)\\&=R\sum r_iT(e_i)+S\sum s_iT(e_i)-R\sum r_iT(0)-S\sum s_iT(0).\end{align*} Considering the $R$-terms first, simply observe that we can add and subtract $RT(0)$ to give us that \[R\sum r_iT(e_i)-R\sum r_iT(0)=R\left(T\left(\sum r_i T(e_i)+r\cdot0\right)-T(0)\right).\] This is simply $RT'\left(\sum r_ie_i\right)$. A similar result holds for the $S$-terms, from which we conclude that \[T'\left(R\sum r_ie_i+S\sum s_ie_i\right)=RT'\left(\sum r_ie_i\right)+ST'\left(\sum s_ie_i\right),\] proving linearity. 
\end{exercise} 

\begin{exercise} \leavevmode
This is obvious from the previous exercise and continuity of linear maps. 
\end{exercise} 

\begin{exercise} \leavevmode
Given two $m$-simplexes $[p_0,\dots,p_m]$ and $[q_0,\dots,q_m]$, the map $f$ taking $p_i$ to $q_i$ for every $i$ is a homeomorphism. Bijectivity is obvious by the definition. Continuity is clear by how we extend $f$ from $\{p_i\}$ to $[p_i]$. Finally, the inverse is of the same form as $f$, only with the $q_i$'s taking the place of the $p_i$'s and vice versa; thus $f^{-1}$ is also continuous. 
\end{exercise} 

\begin{exercise} \leavevmode
The following map works: \[f:x\mapsto\frac{t_2-t_1}{s_2-s_1}(x-s_1)+t_1.\] 
\end{exercise} 

\begin{exercise} \leavevmode
Pick arbitrary $T(x),T(x')\in T(X)$ and observe that \[tT(x)+(1-t)T(x')=T(tx+(1-t)x')\in T(X).\] Thus $T(X)$ is affine if $X$ is affine, and convex if $X$ is convex. The second statement of the exercise follows by noting that $\ell$ is convex. 
\end{exercise} 

\begin{exercise} \leavevmode
Without loss of generality, we delete $p_0$. Now suppose that \[\sum_{i=1}^ms_ip_i+sb=0,\quad\sum_{i=1}^ms_i+s=0.\] Then we know by definition of the barycenter $b$ that \[\sum_{i=1}^ms_ip_i+\frac s{m+1}\sum_{i=0}^mp_i=0.\] Moreover, letting $s'_i$ be the coefficient of $p_i$ in the above equation, it is obvious that $\sum_{i=0}^ms'_i=s+\sum_{i=1}^ms_i=0$. Thus $s'_i=0$ for all $i$ because $\{p_0,\dots,p_m\}$ was affine independent. But then we conclude that $0=s'_0=\frac s{m+1}$, and so $s=0$. For every $i\in\{1,\dots,m\}$, we have $0=s'_i=\frac s{m+1}+s_i$. Thus $s=0$ implies $s_i=0$ for every $i$,a nd so it follows that $\{b,p_1,\dots,p_m\}$ is affine independent, as desired.
\end{exercise} 

\begin{exercise} \leavevmode
Once again, suppose without loss of generality that $i=0$. Then the map taking $\sum t_ip_i\in[p_0,p_1,\dots,p_m]$ to $\left(\sum_{i=1}^mt_ip_i,t_0\right)$ works. Note that this actually requires the affine independence of the $p_i$'s, as well as the fact that the coefficients $t_i$ are all between 0 and 1. 
\end{exercise} 

\begin{exercise} \leavevmode
Notice that $[0,e_1,\dots,e_n]$, where $e_i$ are the standard basis vectors in $\RR^n$, is an $n$-simplex. Thus there is a homeomorphism $[p_0,\dots,p_n]\to[0,e_1,\dots,e_n]$. If we translate the image by $\vec v=(-\frac14,\dots,-\frac14)$, then we can map the result to $D^n$ by taking a radial mapping. In particular, this map will take \begin{align*}p_0&\mapsto\frac{\vec v}{\norm{\vec v}}\\p_i&\mapsto\frac{e_i+\vec v}{\norm{e_i+\vec v}}~\text{for}~i\ne0.\end{align*} Note that this extends to a homeomorphism. 
\end{exercise}

\end{document} 
