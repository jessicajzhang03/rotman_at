\documentclass[../../solutions.tex]{subfiles}

\setcounter{section}{6}

\begin{document}
\section{Simplicial Complexes}
\subsection{Definitions}
\begin{exercise} \leavevmode
\begin{figure}[htbp]
\centering
\includegraphics[width=0.5\textwidth]{7-1.pdf}
\label{fig:7-1}
\end{figure}
\end{exercise}

\begin{exercise} \leavevmode
Consider some (nondegenerate) triangle with vertices $P,x_0,y_0$ in $\RR^2$.
Then define $x_i$ to be the midpoint of $P$ and $x_{i-1}$, and similarly define $y_i$.
Then the union $X$ of the triangle with all the line segments $x_iy_i$ is compact and connected.

We claim that it is not a polyhedron.
Otherwise, there exists some simplicial complex $K$ admitting a homeomorphism $h:|K|\to X$.
But observe that $K$ must have an infinite vertex set.

To see this, for each $i$, define $s_i$ to be
\[s_i=\bigcap_{h^{-1}(x_i)\in s}s,\]
where $s$ ranges over all simplices of $K$.
Note that this intersection is over a nonempty set because $\bigcup s=|K|$, so there must exist some $s$ containing $h^{-1}(x_i)$.
Moreover, there are only finitely many simplices, so the intersection exists.
Condition (ii) implies that $s_i$ is a common face of $s$, and thus is a simplex.
It must be 0-dimensional since the segment $Px_i$, $x_iy_i$, and $x_0x_i$ cannot all be part of the same 1-simplex.
In other words, $x_i$ must be a common face of two 1-simplices, and so it must be a point.

Hence there are infinitely many vertices of $K$, a contradiction.
\end{exercise}

\begin{exercise} \leavevmode
Note that the upper right and lower right triangles are the same.
\end{exercise}

\begin{exercise} \leavevmode
\begin{enumerate}
\item
The forwards direction is just the definition of the subspace topology.
To see the backwards direction, suppose $F\cap s$ is closed in $s$ for every $s\in K$.
Each $s$ is closed in $|K|$, so $F\cap s$ is closed in $|K|$.
Since there are finitely many $s$ and $\bigcup s=|K|$, it follows that we can take the union of all $F\cap s$.
In particular, we have
\[F=\bigcup_{s\in K}(F\cap s)\]
is the finite union of closed sets, hence is itself closed in $|K|$.

\item
This is obviously true if $K$ has dimension 0.

If $K$ (and hence $s$) has dimension $>1$, then consider the complement of $s^\circ$:
\[(s^\circ)^c=(|K|-s)\cup\dot s.\]
Then notice that
\[\big[(|K|-s)\cup\dot s\big]\cap s=\dot s,\]
which is closed in $s$.
Suppose $t\in K$ is not equal to $s$.
Then consider
\[A_t=\big[(|K|-s)\cup\dot s\big]\cap t.\]
If $s\cap t=\emptyset$, then $A_t=\emptyset$ is closed in $t$.
Otherwise, we know that $s\cap t$ is a face of $t$.
Since $s$ is of highest dimension, we know that either $s=t$, which we already took care of above, or $s\cap t$ is part of $\dot s$, in which case we know that
\[\dot s\cap t=s\cap t,\quad(|K|-s)\cap t=t-s\cap t.\]
Hence $A_t=t$, which is still closed in $t$.

The previous part proves the result.
\end{enumerate}
\end{exercise}

\begin{exercise} \leavevmode
We begin by showing $s^\circ\cap t^\circ=\emptyset$ when $s\ne t$.
Note that
\[s^\circ\cap t^\circ=(s-\dot s)\cap(t-\dot t)=s\cap t-\dot s\cap t-s\cap\dot t.\]
But $s\cap t$ is a face of both $s$ and $t$.
It can't be equal to both $s$ and $t$ since $s\ne t$.
Thus $s\cap t$ is a \textit{proper} face of at least one of $s$ and $t$, say $s$.
This means that $s\cap t$ is part of $\dot s$, and thus is in $\dot s\cap t$.
This proves disjointness.

To see that $\bigcup s^\circ=|K|$, simply do this in the case of $K$ as a simplex, and take unions.
(To do this when $K$ is a single simplex, use induction.)
\end{exercise}

\begin{exercise} \leavevmode
The backwards direction is obvious by the definition of $\st$.
For the forwards direction, suppose
\[x\in\st(p)=\bigcup_{p\in\Vt(t)}t^\circ.\]
Then we know that $x\in t^\circ$ for some $t$ having $p$ as a vertex.
Uniqueness implies that $s=t$, so $p\in\Vt(s)$.
\end{exercise}

\begin{exercise} \leavevmode
\begin{enumerate}
\item
Obviously the union is $|K|$ because every $s\in K$ has at least one vertex, hence is contained in at least one star.
To see that $\st(p)\subseteq|K|$ is open, notice that
\[(\st(p))^c=\bigcup_{p\not\in\Vt}s^\circ.\]
Intersect this with $t\in K$.
If $p\not\in\Vt(t)$, then this intersection is equal to $t$ since no simplex of $\dot t$ can have $p$ as a vertex.
If $p\in\Vt(t)$, then write $t=[p,p_1,\dots,p_k]$.
The intersection can be seen to simply be $\{p_1,\dots,p_k\}$, which is obviously closed.
Thus \Cref{7.4} implies the result.

\item
If $x\in\st(p)$, then $x\in s^\circ$ for some $s$ with $p\in\Vt(s)$.
Since $x,p\in s$ and $s$ is convex, it follows that the line segment is also contained in $\st(p)$.
\end{enumerate}
\end{exercise}

\begin{exercise} \leavevmode
The forwards direction is because $[p_0,\dots,p_n]$ is in the intersection.
The backwards direction is because there must exist some simplex $[p_0,\dots,p_n,q_0,\dots,q_m]\in K$.
Since any face of a simplex in $K$ is also in $K$, it follows that $[p_0,\dots,p_n]$ is a simplex in $K$.
\end{exercise}

\begin{exercise} \leavevmode
In the forwards direction, suppose $\phi$ is a simplicial map.
If $\bigcap\st(p_i)\ne\emptyset$, then there exists a simplex in $K$ with vertices $[p_i]$.
The definition implies that there must exist a simplex with vertices $[\phi(p_i)]$, proving this direction.
The backwards direction follows directly from \Cref{7.8}.
\end{exercise}

\begin{exercise} \leavevmode
Suppose $\phi$ is a simplicial approximation to $f$, and suppose $x\in|K|$ with $f(x)\in s^\circ$.
Write $x\in t^\circ$ for $t\in K$, and write $t=[p_1,\dots,p_n]$.
Then we know that $x\in\st(p_i)$ implies that $f(x)\in\st(\phi(p_i))$, so that $s^\circ\subseteq\st(\phi(p_i))$.
Thus $s$ has $\phi(p_i)$ as a vertex for each $i=1,\dots,n$.

Hence $|\phi|(x)$, which is determined by $\phi(p_i)$, is in $s$ by affineness.

Now suppose that $f(x)\in s^\circ$ implies $|\phi|(x)\in s$.
Let $p$ be some vertex of $K$ so that $x\in\st(p)$.
Then $f(x)\in s^\circ$, so $|\phi|(x)\in s$.
Hence $\phi(p)$ is a vertex of $s$ by affineness and the definition of $|\phi|$, from which it follows that
\[f(x)\in s^\circ\subseteq\st(\phi(p)).\]
We can take the union over all $x\in\st(p)$:
\[\bigcup_{x\in\st(p)}f(x)\subseteq\st(\phi(p)).\]
Of course, this left side is exactly $f(\st(p))$, and so we're done.
\end{exercise}

\begin{exercise} \leavevmode
Suppose $\phi:K\to L$ is a simplicial approximation.
Consider the obvious homotopy:
\[H(t,x)=(1-t)|\phi|(x)+tf(x).\]
We can do this because $|\phi|(x)$ and $f(x)$ are, by \Cref{7.10}, in the same simplex.
\end{exercise}

\begin{exercise} \leavevmode
\begin{enumerate}
\item
This is true because it's true for simplices.

\item
Order the vertices of $K$, and define $\phi(b^s)$ to be the smallest vertex of $s$ under this order.
We claim that this gives a simplicial approximation to the identity.
Consider a vertex $b^s$ of $\Sd(K)$.
Then we know that
\[f(\st(b^s))=s^\circ\subseteq\st(\phi(b^s))\]
by the definition of $\phi(b^s)$, where $f$ is the identity.

\item
There exists a homeomorphism $g:|L|\to X$.
If $g(v)$, then we are done.
Otherwise, we know that $x\in g(s^\circ)$ for some unique $s\in L$.
Consider the subdivision $K$ of $L$ obtained by drawing lines from $s$ to every vertex of $s$.
This gives a function $h:|K|\to X$ which is equal to $g$, and thus is a homeomorphism, as desired.
\end{enumerate}
\end{exercise}

\begin{exercise} \leavevmode
Suppose that $\sum\la_ib^{s_i}=0$.
Since $s_0<\dots<s_q$, we knwo that there exists some vertex $p_q$ which only appears in $b^{s_q}$, so $\la_q=0$.
But then there is a vertex $p_{q-1}$ which only appears in $b^{s_{q-1}}$, so $\la_{q-1}=0$, and so on.
Thus $\la_i=0$ for all $i$, proving independence.
\end{exercise}

\begin{exercise} \leavevmode
Every point of $\Sd K$ is contained in a unique open simplex of $K$, so it follows that an open simplex of $\Sd K$ can be contained in at most one open simplex of $K$.
To see that there is at least one such simplex, note that $[b^{s_0},\dots,b^{s_q}]^\circ$ is contained in $s_q^\circ$.
\end{exercise}

\begin{exercise} \leavevmode
This follows from the triangle inequality:
\[|x-y|\le|x-p|+|p-y|\le2\mu,\]
because $x$ and $p$ are in one simplex, and $y$ and $p$ are in another.
\end{exercise}

\begin{exercise} \leavevmode
Write $s=[b^{s_0},\dots,b^{s_q}]$, where $s_0<\dots<s_q$.
Then $\diam s=\sup||b^{s_i}-b^{s_j}||$.
If $i<j$, then we know that
\[||b^{s_i}-b^{s_j}||\le\frac{n_j}{n_j+1}\diam s_j,\]
where $n_j=\dim s_j$.
But $\diam s_j\le\mesh K$ since $s_j\in K$, and $\frac{n_j}{n_j+1}\le\frac{n}{n+1}$, since $n_j\le n$.
Hence it follows that
\[\diam s\le\frac n{n+1}\mesh K,\]
and so $\mesh\Sd K\le(n/n+1)\mesh K$.
Induction implies the general result.
\end{exercise}

\begin{exercise} \leavevmode
If $s\in K^{(q)}$, then $s=[p_0,\dots,p_r]$ for some $r\le q$.
Thus $\phi(s)=[\phi(p_0),\dots,\phi(p_r)]\in L^{(q)}$, as desired.
\end{exercise}

\begin{exercise} \leavevmode
Let $b$ be the barycenter of the $(n+1)$-simplex, and consider
\[f(x)=\frac{x-b}{||x-b||}+b.\]
This is the desired homeomorphism.
\end{exercise}

\begin{exercise} \leavevmode
In general, there are $\binom{n+2}{q+1}$ total $q$-simplices in an $(n+1)$-simplex.
Since $S^n$ is the $n$-skeleton of such a simplex, it follows that we must simply evaluate
\[\chi(S^n)=\sum_{q=0}^n\binom{n+2}{q+1}(-1)^q=\sum_{q=0}^{n+2}\binom{n+2}q(-1)^{q+1}+\binom{n+2}0+\binom{n+2}{n+2}=2\]
when $q$ is even.
When $q$ is odd, the last term is negative, and we find that $\chi(S^n)=0$.
\end{exercise}

\begin{exercise} \leavevmode
Here, we have $\alpha_2=18$, $\alpha_1=27$, and $\alpha_0=9$.
Thus $\chi(T)=18-27+9=0$.
\end{exercise}

\begin{exercise} \leavevmode
Note that $i$ is obviously an injection.
Moreover, since the element $\sum{b\in B_1}m_bb+\sum_{c\in B_2}m_cc\in F(b)$ is equal to
\[\sum{b\in B_1}m_bb+\sum_{c\in B_2}m_cc\in F(b)=p\left(\sum m_bb,\sum-m_cc\right),\]
we see that $p$ is surjective.
Finally, note that
\begin{align*}
\ker p&=\left\{\left(\sum m_bb,\sum m_cc\right):\sum m_bb=\sum m_cc\right\}\\
&=\{(x,x):x\in F(B_1)\cap F(B_2)\}\\
&=\{(x,x):x\in F(B_1\cap B_2)\}=\im i,
\end{align*}
which completes the proof of exactness.
\end{exercise}

\begin{exercise} \leavevmode
For $q\ge1$, the complexes are the same.
If $q=0$, we use the same argument as in Theorem 5.17, in particular, by restricting our attention to the ending:
\[
\begin{tikzcd}
0\ar[r] & \ker\tilde\partial_0\ar[r,hookrightarrow] & C_0(K)\ar[r,"\tilde\partial_0"] & C_{-1}(K)\ar[r] & 0
\end{tikzcd}.
\]
\end{exercise}

\begin{exercise} \leavevmode
This is simply because $\ker\tilde\partial_{-1}=C_{-1}(K)$.
\end{exercise}

\begin{exercise} \leavevmode
\begin{enumerate}
\item
We can simply use the straight line homotopy between $\phi(p)$ and $\psi(p)$ for all vertices $p$ of $K$;
the rest of the point follow by affineness.
The reason this works is simply because $\phi(p)$ and $\psi(p)$ belong to the same simplex, which is convex.
\item
Since $|\phi|\simeq|\psi$, we know that $|\phi|_*=|\psi|_*$, which in turn implies that $\phi_*=\psi_*$ by Theorem 7.22.
\end{enumerate}
\end{exercise}

\begin{exercise} \leavevmode
Let $L$ be a line segment, along with its endpoints and its midpoints.
Thus it is composed of two 1-simplices, and three 0-simplices.
Then let $\phi_1$ map a 1-simplex to the left side of $L$, and $\phi_2$ map it to the right side of $L$.
Finally, if $\phi_3$ maps the 1-simplex to the midpoint, it follows that $\phi_1\sim\phi_3\sim\phi_2$, but obviously $\phi_1\not\sim\phi_2$.
\end{exercise}

\begin{exercise} \leavevmode
\begin{enumerate}
\item
This is clear by mapping the base points together, and mapping a given equivalence class to the corresponding equivalence class.
For example, we have some point $x\in X$, then the homeomorphism would take $[[x]]\in(X\vee Y)\vee Z$ to $[x]\in X\vee(Y\vee Z)$.
Similarly, it would take $[[y]]\mapsto[[y]]$ and $[z]\mapsto[[z]]$.

\item
For $i=1,2$, there exists a simplicial complex $L_i$ and a homeomorphism $h_i:|L_i|\to K_i$.
Fix some vertex $x_i\in\Vt(L_i)$.
Then let $L=L_1\vee L_2$.
Then, identifying each $L_i$ with the natural corresponding set in $L$, we can apply Theorem 7.17 to find the exact senuence
\[
\begin{tikzcd}[column sep=small]
\dots\ar[r] & H_n(L_1\cap L_2)\ar[r] & H_n(L_1)\oplus H_n(L_2)\ar[r] & H_n(L)\ar[r] & H_{n-1}(L_1\cap L_2)\ar[r] & \dots.
\end{tikzcd}
\]
Of course, we have $L_1\cap L_2$ is a singleton, so the homology groups are 0.
Thus, if $n\ge2$, then we know that $H_n(L_1\cap L_2)=H_{n-1}(L_1\cap L_2)=0$, and so $H_n(L)\cong H_n(L_1)\oplus H_n(L_2)$, as desired.
Otherwise, we can simply use the tail:
\[
\begin{tikzcd}[column sep=small]
\dots\ar[r] & H_1(L)\ar[r] & H_0(L_1\cap L_2)\ar[r] & H_0(L_1)\oplus H_0(L_2)\ar[r] & H_0(L)\ar[r] & 0.
\end{tikzcd}
\]
If $L_i$ has $c_i$ components, then notice that $L$ has $c_1+c_2-1$ components.
Since the map $H_0(L_1)\oplus H_0(L_2)\to H_0(L)$ is surjective, it follows that its kernel is $\ZZ$ (or, more accurately, a free abelian group of rank 1).
Hence the image of $H_0(L_1\cap L_2)\to H_0(L_1)\oplus H_0(L_2)$ is $\ZZ$.
The fact that $H_0(L_1\cap L_2)=\ZZ$ implies that this map is an isomorphism, thus with empty kernel.
Finally, we conclude that the image of $H_1(L)\to H_0(L_1\cap L_2)$ is trivial, and so we again have the exact sequence
\[
\begin{tikzcd}[column sep=small]
0\ar[r] H_1(L_1)\oplus H_1(L_2)\ar[r] & H_1(L)\ar[r] & 0
\end{tikzcd}.
\]
The result follows.

\item
Use Corollary 7.19.
In particular, let $K_q$ consist of all proper faces of an oriented $(q+1)$-simplex.
Then the corollary implies that $H_q(K_q)=\tilde H_q(K_q)=\ZZ$ and $H_r(K_q)=0$ for any $r\ne q$.
(Note that reduced homology matches the regular homology since $q\ge 1$.)
Thus the previous part shows that the space
\[\bigvee_{q=1}^n\bigvee_{i=1}^{m_q}K_q,\]
where the wedge occurs at some identified vertex, satisfies the desired properties.
\end{enumerate}
\end{exercise}

\begin{exercise} \leavevmode
\begin{enumerate}
\item
This follows directly from the five lemma and Theorem 7.22, namely by looking at the following commutative diagram with exact rows:
\[
\begin{tikzcd}[column sep=small]
\dots\ar[r] & H_n(L)\ar[d]\ar[r] & H_n(K)\ar[d]\ar[r] & H_n(K,L)\ar[d]\ar[r] & H_{n-1}(L)\ar[d]\ar[r] & H_{n-1}(K)\ar[d]\ar[r] & \dots\\
\dots\ar[r] & H_n(|L|)\ar[r] & H_n(|K|)\ar[r] & H_n(|K|,|L|)\ar[r] & H_{n-1}(|L|)\ar[r] & H_{n-1}(|K|)\ar[r] & \dots
\end{tikzcd}.
\]

\item
This follows from the previous part, Corollary 7.17, and Theorem 7.22.
\end{enumerate}
\end{exercise}

\begin{exercise} \leavevmode
We can simply use the straight line homotopy to $p$.
\Cref{7.7} implies that this is well-defined.
\end{exercise}

\begin{exercise} \leavevmode
In particular, we must show that
\[\left(\bigcap L_{\alpha_i}\right)\cap\left(\bigcap L_{\beta_i}\right)\ne\emptyset.\]
But notice that $\sig_0<\sig_1<\dots<\sig_q$ implies that $\sig_0\in L_{\beta_i}$ for each $\beta_i$.
We also know that $\sig_0\in L_{\alpha_0}\cap\dots\cap L_{\alpha_q}$, and so it follows that $\sig_0$ is in the displayed intersection above.
Hence $g$ and $f$ are contiguous.
\end{exercise}

\begin{exercise} \leavevmode
We have the following exact sequence:
\[
\begin{tikzcd}[column sep=small]
H_q(M\cap L_1)\ar[r] & H_q(M)\oplus H_q(L_1)\ar[r] & H_q(M\cup L_1)\ar[r] & H_{q-1}(M\cap L_1).
\end{tikzcd}
\]
The conditions imply that $H_q(M)\oplus H_q(L_1)=H_q(M)$ and the two outermost terms are both trivial.
Thus $H_q(M)\cong H_q(M\cup L_1)$.

Now consider the following exact sequence:
\[
\begin{tikzcd}[column sep=small]
H_q((M\cup L_1)\cap L_2)\ar[r] & H_q(M\cup L_1)\oplus H_q(L_2)\ar[r] & H_q(M\cup L_1\cup L_2)\ar[r] & H_{q-1}((M\cup L_1)\cap L_2).
\end{tikzcd}
\]
But notice that
\[(M\cup L_1)\cap L_2=(M\cap L_2)\cup(L_1\cap L_2)=M\cap L_2\]
since $L_1\cap L_2\subseteq M$.
Hence the flanking terms of the exact sequence displayed above are again 0.
Since $L_2$ is acyclic, it follows that $H_q(M\cup L_1)\cong H_q(M\cup L_1\cup L_2)$.
Repeating this proves the result.
\end{exercise}

\begin{exercise} \leavevmode
Consider the Klein bottle, as in \Cref{fig:7-31}.
\begin{figure}[htbp]
\centering
\includegraphics[width=0.2\textwidth]{7-31.pdf}
\caption{The Klein bottle}
\label{fig:7-31}
\end{figure}
Let $P$ be the entire square.
Then we can define the adequate subcomplex with chains
\[E_2=\langle P\rangle,\quad E_1=\langle a\rangle\oplus\langle b\rangle,\quad E_2=\langle v\rangle.\]
We have
\begin{align*}
\partial P&=a+b+a-b\\
\partial a=\partial b&=0\\
\partial v&=0.
\end{align*}
Hence it follows that we have
\[
\begin{tabular}{l l l}
$Z_2=0$, & $Z_1=\langle a\rangle\oplus\langle b\rangle$, & $Z_0=\langle v\rangle$, \\
$B_2=0$, & $B_1=\langle2a\rangle$, & $B_0=0$.
\end{tabular}
\]
The results are obvious.
\end{exercise}

\begin{exercise} \leavevmode
This time, if we let $a$ denote each edge and $v$ denote each vertex, we have
\[\partial P=ka,\quad\partial a=0,\quad\partial v=0.\]
Thus we now have
\[
\begin{tabular}{l l l}
$Z_2=0$, & $Z_1=\langle a\rangle\oplus\langle b\rangle$, & $Z_0=\langle v\rangle$, \\
$B_2=0$, & $B_1=\langle ka\rangle$, & $B_0=0$.
\end{tabular}
\]
This gives the desired homology groups.
\end{exercise}

\begin{exercise} \leavevmode
This is true because equality is an equivalence relation.
\end{exercise}

\begin{exercise} \leavevmode
\begin{enumerate}
\item
It suffices to show that $\oo(\alpha)$ cannot be changed in a single move.
But this is clear.
In particular, using the definition, note that $\oo(\alpha)$ is $\oo(\beta)$ if $\beta\ne\emptyset$, and is $p$ if $\beta$ is empty.
The same holds for $\oo(\alpha')$, so $\oo(\alpha)$ is preserved.
Similarly, $\ee(\alpha)=\ee(\alpha')$.

\item
Again, it suffices to show this for a single elementary move.
We can further assume that $\beta=\beta'$.
Write $\alpha=\ga(p,q)(q,r)\dlt$ and $\alpha'=\ga(p,r)\dlt$.
Then
\[\alpha\beta=\ga(p,q)(q,r)\dlt\beta=\ga(p,r)\dlt\beta=\alpha'\beta'.\]
(Recall $\beta=\beta'$.)
\end{enumerate}
\end{exercise}

\begin{exercise} \leavevmode
An edge path, by definition, only goes along the 1-skeleton.
Thus $K$ being connected automatically implies that $K^{(1)}$ is.

If $K^{(1)}$ is connected, then let $x,y\in|K|$.
There are unique open simplices $s^\circ,t^\circ$ with $x\in s^\circ$ and $y\in t^\circ$.
Pick vertices $v$ and $w$ of $s$ and $t$, respectively.
Then consider the path taken by going straight line from $x$ to $v$, then along the edges to $w$, then along a straight to $y$.
Hence $|K|$ is connected (indeed, path-connected).

If $|K|$ is connected, then $|K|$ is clearly path-connected.

Finally, if $|K|$ is path-connected, then we can find edge paths between any two vertices of $K$ in the following manner:
Each time the path crosses the 1-skeleton, say along the edge between $v$ and $w$, pick either $v$ and $w$ and append that vertex (or, rather, the edge between that vertex and the previous one) to the edge path.
That this works is clear.
\end{exercise}

\begin{exercise} \leavevmode
This is exactly the proof of Theorem 3.6, with $\ga$ as the edge path from $p_0$ to $p_1$.
\end{exercise}

\begin{exercise} \leavevmode
Since an elementary move only moves across a 2-simplex, it follows that the edge path group is only dependent on the 2-skeleton.
\end{exercise}

\begin{exercise} \leavevmode
\begin{enumerate}
\item
This is clear.

\item
If $v$ and $w$ are in the same component as some point $x$, then by taking an edge path from $v$ to $x$, then from $x$ to $w$, we have an edge path between $v$ and $w$.
This proves that components are connected.

Obviously the union of the components is $K$.
To see that the unions are disjoint, suppose $v\in[x]\cap[y]$ and $w\in[x]$.
Then the path $w\to x\to v\to y$ implies that $w\in[y]$.
Since $w$ was arbitrary, and since $w\in[y]$ would similarly imply $w\in[x]$, it follows that $[x]=[y]$.
This proves disjointness.

\item
Suppose $[\alpha]\in\pi(K,x)$.
Then we claim that $[\alpha]\in\pi(L,x)$.
But this is simply because any vertex along $\alpha$ is necessarily connected to $p$ via an edge path, hence belongs to $L$.
\end{enumerate}
\end{exercise}

\begin{exercise} \leavevmode
\begin{enumerate}
\item
Write $\alpha=e_1\dots e_m$ and $\beta=e_{m+1}\dots e_{m+n}$.
Then $(\alpha\beta)^\circ:I_{m+n}\to K$ takes $v_i$ to $p_i$, where $p_i=\alpha^\circ(v_i)$ for $0\le i\le m$ and $p_i=\beta^\circ(v_{i-m-1})$ otherwise.
This is exactly $\ga$.

\item
It suffices to show this if $\alpha$ and $\beta$ are separated by one step.
But, writing $\alpha=\ga(p,q)(q,r)\dlt=\ga(p,r)\dlt=\beta$, simply note that we can use the straight line homotopy from the center of $(p,r)$ to go to $q$.
Resizing intervals as necessary, as in the previous part, gives the result.
\end{enumerate}
\end{exercise}

\begin{exercise} \leavevmode
It suffices to show that trees are contractible.
This is true for zero or one 1-simplices.
For $(n+1)$ total 1-simplices, simply pick an edge one of whose endpoints is a leaf.
Then we can contract that edge to the other vertex, which is connected to the rest of the tree.
Induction implies the result.
\end{exercise}

\begin{exercise} \leavevmode
Suppose $e_1\dots e_n$ were a circuit in $T_1\cup T_2$.
Suppose without loss of generality that $e_1\in T_1$.
Let $i$ and $j$ be the first and last indices, respectively, such that $e_i,e_j\in T_1\cap T_2$.
There is a path $\alpha$ which starts with $e_i$ and ends with $e_j$ contained in $T_1\cap T_2$.
Now notice that $e_1\dots e_{i-1}\alpha e_{j+1}\dots e_n$ is a circuit contained entirely within $T_1$, contradicting that $T_1$ is a tree.
\end{exercise}

\begin{exercise} \leavevmode
Let $G$ be any abelian group, and let $\phi:\{xF':x\to X\}\to G$.
Our goal is to show that there is a unique homomorphism $\psi:F/F'\to G$ with $\psi(xF')=\phi(xF')$ for all $xF'\in F/F'$.
(See Theorem 4.1(i).)

As in the definition of a free group, let $\tilde\phi$ be the unique homomorphism from $F$ to $G$ with $\tilde\phi(x)=\phi(xF')$ for all $x\in X$.
Now define
\begin{align*}
\psi:F/F'&\to G\\ fF'&\mapsto\tilde\phi(f).
\end{align*}
To see that this is well-defined, notice that $f\in F'$ implies that $f=g^{-1}h^{-1}gh$ for some $g,h\in F$.
Thus
\[\tilde\phi(f)=\tilde\phi(g)^{-1}\tilde\phi(h)^{-1}\tilde\phi(g)\tilde\phi(h).\]
But $G$ is abelian, so this is exactly 1, which proves well-definedness.

To see that $\psi$ does indeed satisfy that $\psi(xF')=\phi(xF')$, simply notice that $\psi(xF')=\tilde\phi(x)$, which is defined to be $\phi(xF')$.

Finally, to see that $\psi$ is the \textit{unique} homomorphism with this property, note that any other function $\psi'$ would have to have $\psi'(fF')=\tilde\phi(f)$, and thus be exactly equal to $\psi$.

Hence $F/F'$ is indeed free abelian, with the desired basis.
\end{exercise}

\begin{exercise} \leavevmode
\Cref{7.42} shows that the rank of the free group $F$ is the rank of the free abelian group $F/F'$.
But this latter rank is invariant with respect to $X$.
\end{exercise}

\begin{exercise} \leavevmode
\begin{enumerate}
\item
By picking a maximal tree $T$, and setting some edge not in the tree to be $x$, we can see that every other edge becomes either $x$, $x^{-1}$, or 1.
Hence $G_{\RR P^2,T}\cong\ZZ/2\ZZ$, and Corollary 7.37 implies the result.

\item
Hurewicz's theorem applies since $\RR P^2$ is obviously path-connected.
Moreover, since $\ZZ/2\ZZ$ is abelian, its commutator subgroup is trivial.
Thus $H_1(\RR P^2)\cong\ZZ/2\ZZ$, as desired.
\end{enumerate}
\end{exercise}

\begin{exercise} \leavevmode
\begin{enumerate}
\item
Pick points $x,y\in X$.
Then consider vertices $p$ and $q$ of the simplices containing $x$ and $y$, respectively.
Consider the following path:
Take the straight line from $x$ to $p$, then take the path mapped out by $F(p,t)$ as $t\in\II$, then the path mapped out by $F(q,1-t)$, and finally the straight line from $q$ to $y$.

\item
Let $F:X\times\II\to X$ have $F(v,0)$ for all $v\in X^{(1)}$ and $F(~,1)$ a constant function.
Then by taking the homotopy along $F$, we can go from $(p,q)$ to the constant point, then back to some arbitrary edge of $T$, where $T$ is a maximal tree of $X$.
Hence $(p,q)=1$, implying a trivial edge path group.
Thus the fundamental group is trivial too.
\end{enumerate}
\end{exercise}

\begin{exercise} \leavevmode
Since there are $n$ vertices, we know that there are $n-1$ edges of a maximal tree.
The result follows from Corollary 7.35.
\end{exercise}

\begin{exercise} \leavevmode
If $X$ has $m$ edges and $n$ vertices, then $\chi(X)=-m+n$.
Thus $1-\chi(X)=m-n+1$.
Now use Hurewicz's theorem, \Cref{7.35}, \Cref{7.42}, and Corollary 7.35 to find the result for $H_1$.
Note that $H_0(X)=\ZZ$ because $X$ is connected, and $H_q(X)=0$ for $q\ge2$ because $X$ has dimension 1.
\end{exercise}

\begin{exercise} \leavevmode
We know that $S^m$ is the boundary of an $(m+1)$-simplex.
Thus there is an edge between any two vertices, so we can fix one vertex $p$ and let $T$ be the star consisting of all edges $(p,q)$.
Now consider any other edge $(q,r)$.
Note that $\{p,q,r\}$ forms a simplex, so $(p,q)(q,r)=(p,r)$.
But in $G_{K,T}$, we know that $(p,q)=(p,r)=1$, so $(q,r)=1$ as well.
Thus $\pi(K,p)\cong G_{K,T}=1$, and so $\pi_1(S^m)=1$.
Hence $S^m$ is simply connected.
\end{exercise}

\begin{exercise} \leavevmode
\begin{enumerate}
\item 
Since every vertex is contained in $K^{(q)}$, we can pick any simplex of maximal dimension.
Its vertices are contained in $\Vt(K^{(q)})$, but it does not itself belong in the $q$-skeleton.
\item 
If a full subcomplex $L$ exists, we know that it would need to include every simplex of $K$ with vertices in $A$.
Moreover, adding any other simplex would introduce new vertices.
Thus such a subcomplex would be unique.
Note that the set thus described is indeed a subcomplex, since any faces of $s\in L$ would have to have vertices in $A$ as well.

The second part of the statement follows from the description of $L$.
\end{enumerate}
\end{exercise}

\begin{exercise} \leavevmode
Consider some element $[\alpha]=\in\pi(K,v_0)$.
Then there is some path $\alpha'\simeq\alpha$ with $\alpha'\in\pi(L,v_0)$.
Thus $i[\alpha']=[\alpha]$, proving surjectivity.

If $K$ is the 2-simplex and $L$ is its boundary, then obviously any closed edge path in $K$ is also in $L$ (and, in particular, is homotopic to a closed edge path in $L$).
But the fact that $K$ is simply connected while $L$ is not implies that there cannot be an isomorphism.
\end{exercise}

\end{document}