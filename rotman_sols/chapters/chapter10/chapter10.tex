\documentclass[../../solutions.tex]{subfiles}

\setcounter{section}{9}

\begin{document}
\section{Covering Spaces}
\subsection{Basic Properties}
\begin{exercise} \leavevmode

\end{exercise}

\begin{exercise} \leavevmode

\end{exercise}

\begin{exercise} \leavevmode

\end{exercise}

\begin{exercise} \leavevmode

\end{exercise}

\begin{exercise} \leavevmode

\end{exercise}

\begin{exercise} \leavevmode

\end{exercise}

\begin{exercise} \leavevmode

\end{exercise}

\begin{exercise} \leavevmode

\end{exercise}

\begin{exercise} \leavevmode

\end{exercise}

\begin{exercise} \leavevmode

\end{exercise}

\begin{exercise} \leavevmode

\end{exercise}

\begin{exercise} \leavevmode

\end{exercise}

\begin{exercise} \leavevmode

\end{exercise}

\begin{exercise} \leavevmode

\end{exercise}

\begin{exercise} \leavevmode

\end{exercise}

\begin{exercise} \leavevmode

\end{exercise}

\begin{exercise} \leavevmode

\end{exercise}

\subsection{Covering Transformations}
\begin{exercise} \leavevmode

\end{exercise}

\begin{exercise} \leavevmode

\end{exercise}

\begin{exercise} \leavevmode

\end{exercise}

\begin{exercise} \leavevmode

\end{exercise}

\begin{exercise} \leavevmode

\end{exercise}

\begin{exercise} \leavevmode

\end{exercise}

\subsection{Existence}
\begin{exercise} \leavevmode

\end{exercise}

\begin{exercise} \leavevmode

\end{exercise}

\begin{exercise} \leavevmode

\end{exercise}

\begin{exercise} \leavevmode

\end{exercise}

\begin{exercise} \leavevmode

\end{exercise}

\subsection{Orbit Spaces}
% 10.29
\begin{exercise} \leavevmode
We know by Theorem~10.54 that $\Cov(\tilde X/(\tilde X/H))=H$, and so we can think of $G$ as a subgroup of $\Cov(\tilde X/(\tilde X/H))$.
Now use Theorem~10.52, with $X=\tilde X/H$.
We know, in particular, that $G$ is a subgroup of $\Cov(\tilde X/X)$, and thus is exactly a covering space $(\tilde X/G,v)$ of $X=\tilde X/H$, as desired.
\end{exercise}

\begin{exercise} \leavevmode
\begin{enumerate}
\item
Suppose $gx=x$ and consider a proper neighborhood $V$ of $x$.
Then we know that $gV\cap V=\emptyset$, but $x=gx\in gV\cap V$, contradiction.

\item
If $G=\{e,g_1,\dots,g_n\}$ and $x\in X$, then, since $X$ is Hausdorff and since $g_ix\ne x$, there exists a neighborhood $V$ of $X$ which does not contain any $g_ix$.
Obviously, this $V$ is a proper neighborhood.
\end{enumerate}
\end{exercise}

\begin{exercise} \leavevmode
This is exactly the argument in the proof of Theorem~10.2, namely in the first full paragraph on p.~276.
\end{exercise}

\begin{exercise} \leavevmode
\begin{enumerate}
\item
The group $\ZZ/p\ZZ$ acts on $S^3$ via $m\bullet(z_0,z_1)=(\zeta^mz_0,\zeta^{mq}z_1)$.
This action is proper because part (ii) of \Cref{10.30} obviously applies.

\item
Note that $S^3/(\ZZ/p\ZZ)$ is exactly $L(p,q)$.
Thanks to the previous part, Theorem~10.54(ii) applies, which implies that
\[\pi_1(L(p,q))=\pi_1(S^3/(\ZZ/p\ZZ))=\ZZ/p\ZZ.\]

\item
We know that $L(p,q)$ inherits the local properties of $S^3$, since there is a local homeomorphism between them.
Thus $L(p,q)$ is a 3-manifold.

If $\mathcal U$ is an open cover of $L(p,q)$, then $p^{-1}(\mathcal U)$ is an open cover of $S^3$.
Hence finitely many elements of $p^{-1}(\mathcal U)$, say $p^{-1}(U_i)$ for $i=1,\dots,n$, cover $S^3$.
But then $\{U_1,\dots,U_n\}$ is a finite subcover of $\mathcal U$ which covers $L(p,q)$, proving compactness.

Finally, note that $A\subseteq L(p,q)$ clopen implies that $p^{-1}(A)$ is clopen in $S^3$.
Hence $p^{-1}(A)=\emptyset,S^3$, and so $A=\emptyset,L(p,q)$.
Thus $L(p,q)$ is connected too.
\end{enumerate}
\end{exercise}

\begin{exercise} \leavevmode
Notice that $T\to T/G$ is a universal covering space since $T$ is simply connected.
Moreover, since $T/G$ is a connected 1-complex, we know by Corollary~7.35 that $\pi_1(T/G)$ is free.
But Theorem~10.54(iii) implies that $\pi_1(T/G)\cong G$, and so $G$ is free.
\end{exercise}

\end{document}