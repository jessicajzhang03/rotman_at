\documentclass[../../solutions.tex]{subfiles}

\setcounter{section}{9}

\begin{document}
\section{Covering Spaces}
\subsection{Orbit Spaces}
% 10.29
\begin{exercise} \leavevmode
We know by Theorem~10.54 that $\Cov(\tilde X/(\tilde X/H))=H$, and so we can think of $G$ as a subgroup of $\Cov(\tilde X/(\tilde X/H))$.
Now use Theorem~10.52, with $X=\tilde X/H$.
We know, in particular, that $G$ is a subgroup of $\Cov(\tilde X/X)$, and thus is exactly a covering space $(\tilde X/G,v)$ of $X=\tilde X/H$, as desired.
\end{exercise}

\begin{exercise} \leavevmode
\begin{enumerate}
\item
Suppose $gx=x$ and consider a proper neighborhood $V$ of $x$.
Then we know that $gV\cap V=\emptyset$, but $x=gx\in gV\cap V$, contradiction.

\item
If $G=\{e,g_1,\dots,g_n\}$ and $x\in X$, then, since $X$ is Hausdorff and since $g_ix\ne x$, there exists a neighborhood $V$ of $X$ which does not contain any $g_ix$.
Obviously, this $V$ is a proper neighborhood.
\end{enumerate}
\end{exercise}

\begin{exercise} \leavevmode
This is exactly the argument in the proof of Theorem~10.2, namely in the first full paragraph on p.~276.
\end{exercise}

\begin{exercise} \leavevmode
\begin{enumerate}
\item

\end{enumerate}
\end{exercise}

\end{document}