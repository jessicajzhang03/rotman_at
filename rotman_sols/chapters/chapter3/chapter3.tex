\documentclass[../../solutions.tex]{subfiles}

\setcounter{section}{2}

\begin{document} 

\section{The Fundamental Group}
\subsection{The Fundamental Groupoid} 
\begin{exercise} \leavevmode
The homotopy $H:X\times\II\to Z$ given by \[H:(x,t)\mapsto\begin{cases}g_0(F(x,2t))&\text{if}~t\le\frac12,\\G(f_1(x),2t-1)&\text{if}~t\ge\frac12\end{cases}\] works. Continuity follows because $g_0(F(x,1))=G(f_1(x),0)$. 

Moreover, this homotopy is indeed $\rel A$. For a detailed argument why this is so, simply suppose that $a\in A$ and $t\in I$. If $t\le\frac12$, then $F(a,2t)=f_0(a)$ by definition of $F$. Hence $H(a,t)=g_0(f_0(a))$. 

Similarly, we can show that if $t\ge\frac12$, then $H(a,t)=g_1(f_1(a))$. This follows because $f_1(a)\in B$ and $G$ is a homotopy $\rel B$. 

It thus suffices to show that $g_0(f_0(a))=g_1(f_1(a))$. But this is obvious because $f_0$ and $f_1$ agree on $A$, and $g_0$ and $g_1$ agree on $B\supseteq f_0(A)$. 
\end{exercise} 

\begin{exercise} \leavevmode
\begin{enumerate}
\item First, note that $f'$ is well-defined because $f(0)=f(1)$. It is obvious by continuity of $f$ and $\ln$ that $f'$ is continuous. 

Moreover, consider the map \[H':\left(e^{2\pi i\theta},t\right)\mapsto H(\theta,t).\] This is clearly continuous, for the same reasons that $f'$ was continuous. If $t=0$, clearly $H'(e^{2\pi i\theta},t)=H(\theta,0)=f(\theta)=f'(e^{2\pi i\theta})$, and similarly for $t=1$. Thus $H$ is indeed a homotopy from $f'$ to $g'$. 

To see that it is a homotopy $\rel\{1\}$, simply note that $e^{2\pi i\theta}=1$ corresponds to $\theta=0,1$. Thus it follows that \[H'(1,t)=H(1,t)=f(1)\] for all $t$, proving the result. 

\item Theorem 3.1 implies that $f*g\simeq f_1*g_1\rel\Idot$. Using the previous part, we find that $(f*g)'\simeq(f_1*g_1)'\rel\{1\}$. Now, using the observation that $(f*g)'=f'*g'$, we find that $f'*g'\simeq f_1'*g_1'\rel\{1\}$, as desired. 
\end{enumerate} 
\end{exercise} 

\begin{exercise} \leavevmode
The forward direction is trivial. 

For the converse, note that $g'$ is a constant map, and so $f'$ is nullhomotopic. Then Theorem 1.6 implies that $f'\simeq g'\rel\{1\}$. In particular, note that $g':S^1\to X$ takes every element of $S^1$ to $g'(1)=g(0)=x_0$. Observe that $f'(1)=x_0$ as well, and so it follows that $f'\simeq g\rel\{1\}$, as desired. 
\end{exercise} 

\begin{exercise} \leavevmode
\begin{enumerate}
\item Instead of applying Theorem 1.6, I constructed an explicit homotopy. (If you are interested in a proof using Theorem 1.6, my guess would be that it relies on the fact that $\Delta^2\approx D^2$. However, I have not gone through the details.) 

The effective idea of the homotopy I constructed is to, at time $t\in[0,1]$, return the function which traverses the first $t$ units of the face opposite $e_0$, then goes along a segment to the point $t$ units away from $e_1$ on the fact opposite $e_2$, before returning back to $e_1$, as shown in the red path below. 

\begin{figure}[htbp]
\centering 
\begin{asy}
settings.outformat="pdf";
unitsize(3cm);
pair e0, e1, e2; 
e0 = (-1/2,0); 
e1 = (1/2,0); 
e2 = (0,sqrt(3)/2); 

pair te0,te2; 
real t = 0.7; 
te0 = t*e2+(1-t)*e1; 
te2 = t*e0+(1-t)*e1; 

draw(e0--e1,dotted,MidArcArrow(SimpleHead)); 
draw(e1--e2,dotted,MidArcArrow(SimpleHead)); 
draw(e0--e2,dotted,MidArcArrow(SimpleHead)); 

draw(e1--te0--te2--cycle,red,MidArcArrow(SimpleHead));

dot("$e_0$",e0,WNW); 
dot("$e_1$",e1,ENE); 
dot("$e_2$",e2,E); 
dot(te0,red);
dot(te2,red);
\end{asy}
\end{figure} 

The specific homotopy $H:\II\times\II\to X$ from $(\sigma_0*\sigma_1^{-1})*\sigma_2$ to the constant map at $e_1$ is as follows: \[H(x,t)=\begin{cases}\sigma_0(4(1-t)x)&\text{if}~x\le\frac14,\\\sigma((1-x)\ep_0(1-t)+x\ep_2(t))&\text{if}~\frac14\le x\le\frac12,\\\sigma(2tx-(2t-1))&\text{if}~x\ge\frac12.\end{cases}\]

We leave it to the reader to check that this works. 

\item One can generate a similar homotopy, which we do not do here. 

\item This time, we use the homotopy which goes up along $\gamma$ for $t$ units, before going parallel to $\beta$ and coming back down along $\delta^{-1}$. The particular formula is as follows: 
\[H(x,t)=\begin{cases}F(0,4tx)&\text{if}~x\le\frac14,\\F(4x-1,t)&\text{if}~\frac14\le x\le\frac12,\\F(1,2t(1-x))&\text{if}~\frac12\le x.\end{cases}\] Once again, we leave the details to the reader to check. 
\end{enumerate} 
\end{exercise} 

\begin{exercise} \leavevmode
Simply use the homotopy $H:\II\times\II\to X\times Y$ which takes $(s,t)$ to $(F(s,t),G(s,t))$. This is clearly a homotopy from $(f_0,g_0)$ to $(f_1,g_1)$. To see that it is still $\rel\Idot$, simply observe that $H(0,t)=(F(0,t),G(0,t))$. Because $F$ and $G$ are both $\rel\Idot$, it follows that $H(0,t)$ never changes. A similar argument shows that $H(1,t)$ is always the same, and so $H$ is indeed a homotopy $\rel\Idot$. 
\end{exercise} 

\begin{exercise} \leavevmode
\begin{enumerate}
\item It is obvious that the homotopy $H':(x,t)\mapsto H(x,1-t)$ works. 

\item This is just some slightly annoying manipulation. In particular, note that \[(f*g)(x)=\begin{cases}f(2x)&\text{if}~x\le\frac12,\\g(2x-1)&\text{if}~x\ge\frac12.\end{cases}\] By replacing $x$ with $1-x$ to get the inverse, we find that \[(f*g)^{-1}(x)=\begin{cases}f(2-2x)&\text{if}~x\ge\frac12,\\g(1-2x)&\text{if}~x\le\frac12.\end{cases}\] However, note that \begin{align*}(g^{-1}*f^{-1})(x)&=\begin{cases}g^{-1}(2x)&\text{if}~x\ge\frac12,\\f^{-1}(2x-1)&\text{if}~x\ge\frac12\end{cases}\\&=\begin{cases}g(1-2x)&\text{if}~x\le\frac12,\\f(2-2x)&\text{if}~x\ge\frac12.\end{cases}\end{align*} Thus the two are indeed the same. 

\item Take the closed path $f(t)=e^{2\pi it}$ on $S^1$. Then note that $(f*f^{-1})(\frac18)=f(\frac14)=i$, while $(f^{-1}*f)(\frac18)=f^{-1}(\frac14)=-i$. 

\item Suppose $i_p*f=f$ and $f$ is not constant. Note that continuity implies that there must exist some $0<t<1$ so that $f(t)\ne p$. Thus there exists some $k\in\NN$ so that $t<1-2^{-k}$. 

We claim, however, that $f$ must be constant on $[0,1-2^{-n}]$ for every $n\in\NN$. We prove this inductively. Clearly, it is true on $n=0$. If it is true on $n-1$, then we know that $i_p*f$ must be equal to $p$ on $[0,\frac12]$, as well as on $[\frac12,1-2^{-n}]$ (note that $1-2^{-n}$ comes from $2(1-2^{-n})-1$, which itself comes from the equation for the star operator). Thus $f$ is constant on $[0,1-2^{-n}]$, as desired. 

Thus it follows that $f(t)=p$, a contradiction. Thus $f$ must have been constant in the first place. 
\end{enumerate} 
\end{exercise} 

\begin{exercise} \leavevmode
Recall that we defined the $\sin(1/x)$ space as the union of $A=\{(0,y):-1\le y\le1\}$ and $G=\{(x,\sin(1/x)):0<x\le1/2\pi\}$. We also know that $A$ and $G$ are the path components of the $\sin(1/x)$ space. Moreover, both $A$ and $G$ are contractible, and so every path in either $A$ or $G$ is nullhomotopic. In particular, we conclude that the fundamental group at any basepoint is trivial.  
\end{exercise} 

\begin{exercise} \leavevmode 
Let $X$ be the $\sin(1/x)$ space. We know that $CX$ is contractible. But consider an open ball around the point $x=((0,0),0)$, that is, the point $(0,0)$ on the ``zeroth'' level of the cone. Consider a small neighborhood (not including the points $(t,1)$, in particular) around this point and pick some element $y=((\ep,\sin(1/\ep)),0)$ in the neighborhood. Now observe that any path between $x$ and $y$ can be projected down to a path between $(0,0)$ and $(\ep,\sin(1/\ep))$ in $X$, which we know does not exist. Hence $CX$ is contractible but not locally path connected. 
\end{exercise}

\begin{exercise} \leavevmode
Note that composition is associative because $\circ$ is. Moreover, the path class of the trivial loop based at $p$ is the identity on $p$. Thus this is a category. 

To see that each morphism in $\cat C$, simply note that the inverse path, i.e., the path $f^{-1}$ taking $t$ to $f(1-t)$, gives a path class $[f^{-1}]$ which works as an inverse to $[f]\in\Hom(p,q)$. 
\end{exercise}

\begin{exercise} \leavevmode 
We simply let $\pi_0$ take $(X,x_0)\in\Cat{Sets}_*$ to the set of all path components of $X$, with basepoint equal to the path component containing $x_0$. It takes a morphism $f\in\Hom((X,x_0),(Y,y_0))$ to the map $\pi_0(f)$ which takes each path component $A$ of $X$ to the path component $B$ of $Y$ which contains $f(A)$. 

Note that this is possible because continuous images of path connected spaces are path connected and hence contained within a single path component of $Y$. Moreover, this is indeed a pointed map because the path component containing $x_0$ must be contained in the path component containing $f(x_0)=y_0$, which is the basepoint of $\pi_0((Y,y_0))$. 

It is easy to check functoriality, completing the proof. 
\end{exercise}

\begin{exercise} \leavevmode
Evidently the only possible path is the constant path at $x_0$. Hence $\pi_1(X,x_0)$ is the trivial group, i.e., $\{1\}$. 
\end{exercise}

\begin{exercise} \leavevmode
Note that $1_S$ is a loop based at 1, i.e., an element of $\pi_1(S^1,1)$. 
Thus if $\pi_1(S^1,1)$ were trivial, then $1_S$ would be nullhomotopic. 
The hint gives the rest of the solution. 
\end{exercise}

\begin{exercise} \leavevmode
We know that $\deg u=1$. 
Since 1 is a generator for $\ZZ$, it follows that $[u]$ generates $\pi_1(S^1,1)$. 
\end{exercise}

\begin{exercise} \leavevmode
Let $\tilde\gamma(t)=m\tilde f(t)$, where $\tilde f$ is the lifting of $f$ satisfying $\tilde f(0)=0$. 
Now simply observe that \[\exp\tilde\gamma(t)=\left(\exp\tilde f(t)\right)^m=f(t)^m\] and $\tilde\gamma(0)=0$. 
Thus $\tilde\gamma$ is indeed the lifting of $f^m$ taking 0 to 0, and so we conclude that \[\deg(f^m)=\tilde\gamma(1)=m\tilde f(1)=m\deg f.\]
\end{exercise}

\begin{exercise} \leavevmode
Note that \Cref{1.3} implies that there is a homotopy $F:R_f\circ f\simeq f$, where $R_f$ is the rotation associated with $f$. 
Moreover, from the proof of that same exercise, it follows that $F$ gives a closed path at every time $t$. 
Similarly, we have $G:g\simeq R_g\circ g$. 
Thus if $H:f\simeq g$ where $H$ gives a closed path at every time $t$, then the homotopy which follows $F$, then $H$, and finally $G$ is a homotopy between $R_f\circ f$ and $R_g\circ g$. 
Thus Corollary 3.18 implies that $f$ and $g$ have the same degree. 

For the converse, simply use Corollary 3.18 to show that $\deg f=\deg g$ implies that there is a homotopy $\rel\Idot$ taking $R_f\circ f$ to $R_g\circ g$. 
Then using $F$ and $G$ defined above, it is clear that $g\simeq R_g\circ g\simeq R_f\circ f\simeq f$. 
\end{exercise}

\begin{exercise} \leavevmode
Theorem 3.7 implies that $\pi_1(T,t_0)=\ZZ\times\ZZ=\ZZ^2$. 
\end{exercise}

\begin{exercise} \leavevmode
Because $D^2$ is contractible, its fundamental group is trivial. 
Thus if there were to exist a retraction $r:D^2\to S^1$, then $r_*:\pi_1(D)\to\pi_1(S^1)$ would be a constant. 
But then, letting $i:S^1\to D^2$ be the canonical injection, we would have that $(r\circ i)_*=r_*\circ i_*$ is a constant. 
However, we also know that $r\circ i$ is the identity on $S^1$, and so $(r\circ i)_*$ is the identity on $\pi_1(S^1)$, which is \textit{not} a constant. 
This is a contradiction, from which we conclude that $S^1$ is not a retract of $D^2$, as desired. 
\end{exercise}

\begin{exercise} \leavevmode
This was proved in Theorem 0.3, which required only the fact proved in the above problem, namely that $S^1$ is not a retract of $D^2$. 
\end{exercise}

\begin{exercise} \leavevmode
\begin{enumerate}
\item Let $\tilde f$ be the unique lifting of $f$ with $\tilde f(0)=0$. 
Then if $\tilde f(1)\ge1$, the intermediate value theorem implies that every point in the interval $[0,1]\subset\RR$ is in the image of $\tilde f$. 
But this implies that $f=\exp\circ\tilde f$ must be surjective, a contradiction. 

\item Consider the map which traverses the circle once counterclockwise, reaching the point 1 at time $t=\frac12$, before looping back and making a clockwise rotation. 
Clearly it is surjective. 
However, it is composed of two loops, one of which has degree 1 and one of which has degree $-1$. 
Because $\deg(f*g)=\deg f+\deg g$, it follows that this map has degree 0. 
\end{enumerate}
\end{exercise}

\begin{exercise} \leavevmode
As per the hint, consider an arbitrary closed path $f$ in $X$ and let $\la$ be a Lebesgue number of the open cover $\{f^{-1}(U_j):j\in J\}$ of $\II$. 
Note that $\la$ exists by the Lebesgue number lemma and compactness of the unit interval. 
Picking $N\in\NN$ with $N>1/\la$, it follows that if we subdivide $I$ into $N$ equal intervals $I_k=[\frac kN,\frac{k+1}N]$, then $f(I_k)\subseteq U_{j_k}$ for some $j_k\in J$. 

Define $f_k$ as the path in $U_{j_k}$ obtained by restricting $f$ to $I_k$ and then stretching suitably so that the domain is all of $\II$.
With notation, define $f_k(t)=f\left(\frac{k+t}N\right)\in U_{j_k}$.
Because $f_k$ is a path in $U_{j_k}$, it follows that $[f'_k]=[i_{j_k}\circ f_k]\in\im(i_j)_*$. But now simply observe that $[f'_0*\dots*f'_{N-1}]=[f]$, implying that $[f]$ is contained in the group generated by the subsets $\im(i_j)_*$. This proves the result. 
\end{exercise}

\begin{exercise} \leavevmode
Let $U_1$ and $U_2$ be defined as in the hint, and let $i_k$ be the injection from $U_k$ to $S^n$ for $k=1,2$. 
Observe that, by the previous exercise, it suffices to show that $\im(i_k)_*$ is trivial for $k=1,2$. 

Without loss of generality, let $k=1$. 
But we know that $(i_1)_*$ takes a closed path $f:\II\to U_1$ to the path class $[i_1\circ f]$. 
(Note that the basepoint doesn't really matter for us as long as it is neither the north nor the south pole.Thus we omit it.)
Because $U_1\approx D^n$ and is therefore contractible, it follows that $f$ is nullhomotopic. 
In particular, we know that $i_1\circ f$ is nullhomotopic, and so $[i_1\circ f]=[1]$ for every $f$. 
Thus $\im(i_1)_*$ is trivial, and similarly for $k=2$, proving the result. 
\end{exercise}

\begin{exercise} \leavevmode
Corollary 3.11 implies that path connected spaces of the same homotopy type must have isomorphic fundamental groups. 
But obviously $\ZZ\not\cong\{1\}$, and so $S^1$ and $S^n$ do not have the same homotopy type for $n>1$. 
\end{exercise}

\begin{exercise} \leavevmode
The multiplication map $\mu$ on $G/H$ is continuous. 
After all, if we let $v$ be the natural map, then for any open set $U\subseteq G/H$, we have \[\mu^{-1}(U)=\{([x],[y]):xy\in v^{-1}(U)\}.\] 
But this set is open in $G/H\times G/H$ because the set consisting of elements $(v^{-1}([x]),v^{-1}([y]))$ for each $([x],[y])\in \mu^{-1}(U)$ is just $\mu^{-1}(v^{-1}(U))$, which is clearly open. 

For the inversion map $i:G/H\to G/H$, a very similar argument holds. 
In particular, for any open set $U\subseteq G/H$, we have \[v^{-1}(i^{-1}(U))=\{x^{-1}:x\in v^{-1}(U)\}.\]
Thus $v^{-1}(i^{-1}(U))$ is open, and so $i^{-1}(U)$ is open, proving continuity. 
\end{exercise}

\begin{exercise} \leavevmode
First, we will show that we can lift a loop $f:(\II,0)\to(G/H,1)$ to a unique continuous map $\tilde f:(\II,0)\to(G,h)$ for any $h_0\in H$, as shown below. 
\[\begin{tikzcd}[row sep=large]
    &(G,h_0)\arrow[d,"v"]\\
    (\II,0)\arrow[ur,"\tilde f"]\arrow[r,"f" swap]&(G/H,1)
\end{tikzcd}\]
In the above diagram, the map $v$ is the natural map taking $g$ to the coset $gH\in G/H$. 

First, we will find a suitable neighborhood $U$ of 1 such that the family $\{hU:h\in H\}$ is pairwise disjoint. 
Discreteness of $H$ implies that there exists an open neighborhood $V$ of 1 with $V\cap H=\{1\}$. 
It is clear that the map $\phi:(x,y)\mapsto xy^{-1}$ is the composition $\mu\circ(\id\times i)$ and is therefore continuous. 
Thus $\phi^{-1}(V)\subseteq G\times G$ is an open neighborhood of $(1,1)$. 
This implies that we can find an open neighborhood $U$ of 1 such that $U\times U\subseteq f^{-1}(V)$. 

Now suppose that there are $h_1,h_2\in H$ and $x,y\in U$ with $h_1x=h_2y$. 
But this would require that $xy^{-1}=h_1^{-1}h_2$. 
It is clear that $xy^{-1}\in\phi(U)\subseteq V$. 
Moreover, because $H$ is a subgroup, we know that $h_1^{-1}h_2\in H$, and so $xy^{-1}\in V\cap H$. 
Thus $x=y$ and $h_1=h_2$, proving that the sets $hU$ are disjoint, as desired. 
Note that any translate $U_g=gU$ of $U$ is a neighborhood of $g\in G$ and has $\{hU_g:h\in H\}$ disjoint. 

Note that $v$ is an open map, and so the set $W=v(U)\subseteq G/H$ is open. 
Moreover, because $v|_U$ is the restriction of a continuous open map to an open set, it follows that $v|_U$ is itself continuous and open. 
It is also a bijection onto $W$, and so $v|_U:U\to V$ is a homeomorphism. 

Note that the collection of sets $V[g]$ for $[g]\in G/H$ forms an open cover of $G/H$. 
Thus, if we are given some $f:(\II,0)\to(G/H,1)$, then we can consider the open cover \[\{f^{-1}(V[g]):[g]\in G/H\}\] of $\II$. 
Note that we can find a finite subcover of this open cover.
This means that we can take subsets of the sets in this open cover, given us a finite collection open overlapping subintervals which are, in order of their smaller coordinate, labeled $I_1,\dots,I_k$. 
Let the group elements $g_1,\dots,g_k$ be such that $I_j\subseteq f^{-1}(V[g_j])$. 
This is simply because $\II=[0,1]$ is connected compact. 

Now we can lift $f$ to each interval $f^{-1}(V[g])$ in this finite subcover. 
Note that $0=t_1\in I_1\subseteq f^{-1}(V[g_1])$. 
Moreover, we know that $v^{-1}(V[g_1])$ consists of disjoint unions of $U$, and so we can pick the one containing $h_0$. 
Now, for each $t\in I_1$, we let $\tilde f(t)$ to be the unique element in this copy of $U$ such that $v(\tilde f(t))=f(t)$. 
Because the intervals overlap, we know that there is some $t_2\in I_2\cap I_1$, and so we can do the same thing, all the way to $t_k$. 
This lets us define $\tilde f(t)$ for all $t\in\II$, and it is easy to show that our construction is indeed a lifting satisfying the commutative diagram above. 

Now consider the map $d:\pi_1(G/H,1)\to H$ taking a loop $[f]$ to $d([f])=\tilde f(1)$, where $\tilde f$ is the unique lifting of $f$ with $\tilde f(0)=1$. 
It is obvious that $\im d\subseteq H$ because $v(\tilde f(1))=f(1)=[1]$ implies that $d([f])=\tilde f(1)\in H$. 
Moreover, the reverse inclusion holds, showing surjectivity. 
In particular, if $h\in H$, then path connectedness of $G$ implies that there is a path $\tilde f$ from 1 to $h$. 
Taking its projection $f=v\circ\tilde f$, note that $f$ is a loop because $v(\tilde f(1))=v(h)=[1]$. 
Thus $d([f])$ is defined and equal to $h$. 
To show injectivity, simply note that $\tilde f(1)=1$ implies that $\tilde f$ is a loop in $G$. 
Because $G$ is simply connected, however, it follows that $\tilde f$, and hence $f$, is nullhomotopic. 
Thus $f\in\ker d$ implies that $[f]=[1]$. 
Finally, we must show that $d$ is indeed a homomorphism, i.e., that $d(f*g)=d(f)d(g)$. 
But this is clear if we lift $f$ to $\tilde f$ with $\tilde f(0)=1$, and if we lift $g$ to $\tilde g$ with $\tilde g(0)=d(f)$. 
This follows the same proof layout as Theorem 3.16, and proves the result. 
\end{exercise}

\begin{exercise} \leavevmode
If $S\subseteq GL(n,\RR)$ is a subgroup of $GL(n,\RR)$, then note that $\mu:S\times S\to S$ is continuous. 
After all, the product, entrywise, is simply a polynomial, and so $\mu$ is a polynomial in each of its $n^2$ entries. 
Since polynomials are continuous in $\RR^2$, it follows that each of the $n^2$ components of $\mu$ is continuous. 
Hence $\mu$ is continuous. 

To see that the inversion $i$ is continuous, observe that the determinant $\det A$ is a continuous function, since it too is a polynomial (and is never zero, by definition of $GL$). 
It thus suffices to show that the function $A\mapsto\adj A$ is continuous. 
But it is easy to see that the adjugate matrix, which is the transpose of the cofactor matrix, is also a polynomial in the entries of $A$, and so $\adj A$ is continuous too. 
Thus $i$ is continuous, and so $S$ is a topological group, as desired. 
\end{exercise}

\begin{exercise} \leavevmode
As hinted in the exercise, fix $h_0\in H$ and let $\phi:G\to H$ be the map taking $x$ to $xh_0x^{-1}h_0^{-1}$. 
Note that $xh_0x^{-1}\in H$ because $H$ is normal, and so $\phi(x)$ is indeed an element of $H$. 
Moreover, we know that $\phi$ is continuous and $\{h\}\subseteq H$ is open for each $h\in H$. 
Thus $\{\phi^{-1}(h):h\in H\}$ is an open cover of $G$ consisting of disjoint open sets. 

In particular, if there are two elements $h_1,h_2\in H$ such that $\phi^{-1}(h_i)\ne\emptyset$ for $i=1,2$, then setting $A=\phi^{-1}(h_1)$ and $B=\bigcup_{h\ne h_1}\phi^{-1}(h)$ will give us two disjoint open sets $A$ and $B$ that cover $G$. 
This implies that $G$ is disconnected, a contradiction. 
Thus for all but one element of $H$, we must have $\phi^{-1}(h)=\emptyset$, proving that $\phi$ is constant. 
But obviously, setting $x=h_0$, we find that $\phi(x)=1$. 
Thus $xh_0x^{-1}h_0^{-1}=1$ for all $x\in G$, and so $xh_0=h_0x$ for each $h_0\in H$. 
This proves the result. 
\end{exercise}

\end{document}
