\documentclass[../../solutions.tex]{subfiles}

\setcounter{section}{2}

\begin{document} 

\section{The Fundamental Group}
\subsection{The Fundamental Groupoid} 
\begin{exercise} \leavevmode
The homotopy $H:X\times\II\to Z$ given by \[H:(x,t)\mapsto\begin{cases}g_0(F(x,2t))&\text{if}~t\le\frac12,\\G(f_1(x),2t-1)&\text{if}~t\ge\frac12\end{cases}\] works. Continuity follows because $g_0(F(x,1))=G(f_1(x),0)$. 

Moreover, this homotopy is indeed $\rel A$. For a detailed argument why this is so, simply suppose that $a\in A$ and $t\in I$. If $t\le\frac12$, then $F(a,2t)=f_0(a)$ by definition of $F$. Hence $H(a,t)=g_0(f_0(a))$. 

Similarly, we can show that if $t\ge\frac12$, then $H(a,t)=g_1(f_1(a))$. This follows because $f_1(a)\in B$ and $G$ is a homotopy $\rel B$. 

It thus suffices to show that $g_0(f_0(a))=g_1(f_1(a))$. But this is obvious because $f_0$ and $f_1$ agree on $A$, and $g_0$ and $g_1$ agree on $B\supseteq f_0(A)$. 
\end{exercise} 

\begin{exercise} \leavevmode
\begin{enumerate}
\item First, note that $f'$ is well-defined because $f(0)=f(1)$. It is obvious by continuity of $f$ and $\ln$ that $f'$ is continuous. 

Moreover, consider the map \[H':\left(e^{2\pi i\theta},t\right)\mapsto H(\theta,t).\] This is clearly continuous, for the same reasons that $f'$ was continuous. If $t=0$, clearly $H'(e^{2\pi i\theta},t)=H(\theta,0)=f(\theta)=f'(e^{2\pi i\theta})$, and similarly for $t=1$. Thus $H$ is indeed a homotopy from $f'$ to $g'$. 

To see that it is a homotopy $\rel\{1\}$, simply note that $e^{2\pi i\theta}=1$ corresponds to $\theta=0,1$. Thus it follows that \[H'(1,t)=H(1,t)=f(1)\] for all $t$, proving the result. 

\item Theorem 3.1 implies that $f*g\simeq f_1*g_1\rel\Idot$. Using the previous part, we find that $(f*g)'\simeq(f_1*g_1)'\rel\{1\}$. Now, using the observation that $(f*g)'=f'*g'$, we find that $f'*g'\simeq f_1'*g_1'\rel\{1\}$, as desired. 
\end{enumerate} 
\end{exercise} 

\begin{exercise} \leavevmode
The forward direction is trivial. 

For the converse, note that $g'$ is a constant map, and so $f'$ is nullhomotopic. Then Theorem 1.6 implies that $f'\simeq g'\rel\{1\}$. In particular, note that $g':S^1\to X$ takes every element of $S^1$ to $g'(1)=g(0)=x_0$. Observe that $f'(1)=x_0$ as well, and so it follows that $f'\simeq g\rel\{1\}$, as desired. 
\end{exercise} 

\begin{exercise} \leavevmode
\begin{enumerate}
\item Instead of applying Theorem 1.6, I constructed an explicit homotopy. (If you are interested in a proof using Theorem 1.6, my guess would be that it relies on the fact that $\Delta^2\approx D^2$. However, I have not gone through the details.) 

The effective idea of the homotopy I constructed is to, at time $t\in[0,1]$, return the function which traverses the first $t$ units of the face opposite $e_0$, then goes along a segment to the point $t$ units away from $e_1$ on the fact opposite $e_2$, before returning back to $e_1$, as shown in the red path below. 

\begin{figure}[htbp]
\centering 
\begin{asy}
settings.outformat="pdf";
unitsize(3cm);
pair e0, e1, e2; 
e0 = (-1/2,0); 
e1 = (1/2,0); 
e2 = (0,sqrt(3)/2); 

pair te0,te2; 
real t = 0.7; 
te0 = t*e2+(1-t)*e1; 
te2 = t*e0+(1-t)*e1; 

draw(e0--e1,dotted,MidArcArrow(SimpleHead)); 
draw(e1--e2,dotted,MidArcArrow(SimpleHead)); 
draw(e0--e2,dotted,MidArcArrow(SimpleHead)); 

draw(e1--te0--te2--cycle,red,MidArcArrow(SimpleHead));

dot("$e_0$",e0,WNW); 
dot("$e_1$",e1,ENE); 
dot("$e_2$",e2,E); 
dot(te0,red);
dot(te2,red);
\end{asy}
\end{figure} 

The specific homotopy $H:\II\times\II\to X$ from $(\sigma_0*\sigma_1^{-1})*\sigma_2$ to the constant map at $e_1$ is as follows: \[H(x,t)=\begin{cases}\sigma_0(4(1-t)x)&\text{if}~x\le\frac14,\\\sigma((1-x)\ep_0(1-t)+x\ep_2(t))&\text{if}~\frac14\le x\le\frac12,\\\sigma(2tx-(2t-1))&\text{if}~x\ge\frac12.\end{cases}\]

We leave it to the reader to check that this works. 

\item One can generate a similar homotopy, which we do not do here. 

\item This time, we use the homotopy which goes up along $\gamma$ for $t$ units, before going parallel to $\beta$ and coming back down along $\delta^{-1}$. The particular formula is as follows: 
\[H(x,t)=\begin{cases}F(0,4tx)&\text{if}~x\le\frac14,\\F(4x-1,t)&\text{if}~\frac14\le x\le\frac12,\\F(1,2t(1-x))&\text{if}~\frac12\le x.\end{cases}\] Once again, we leave the details to the reader to check. 
\end{enumerate} 
\end{exercise} 

\begin{exercise} \leavevmode
Simply use the homotopy $H:\II\times\II\to X\times Y$ which takes $(s,t)$ to $(F(s,t),G(s,t))$. This is clearly a homotopy from $(f_0,g_0)$ to $(f_1,g_1)$. To see that it is still $\rel\Idot$, simply observe that $H(0,t)=(F(0,t),G(0,t))$. Because $F$ and $G$ are both $\rel\Idot$, it follows that $H(0,t)$ never changes. A similar argument shows that $H(1,t)$ is always the same, and so $H$ is indeed a homotopy $\rel\Idot$. 
\end{exercise} 

\begin{exercise} \leavevmode
\begin{enumerate}
\item It is obvious that the homotopy $H':(x,t)\mapsto H(x,1-t)$ works. 

\item This is just some slightly annoying manipulation. In particular, note that \[(f*g)(x)=\begin{cases}f(2x)&\text{if}~x\le\frac12,\\g(2x-1)&\text{if}~x\ge\frac12.\end{cases}\] By replacing $x$ with $1-x$ to get the inverse, we find that \[(f*g)^{-1}(x)=\begin{cases}f(2-2x)&\text{if}~x\ge\frac12,\\g(1-2x)&\text{if}~x\le\frac12.\end{cases}\] However, note that \begin{align*}(g^{-1}*f^{-1})(x)&=\begin{cases}g^{-1}(2x)&\text{if}~x\ge\frac12,\\f^{-1}(2x-1)&\text{if}~x\ge\frac12\end{cases}\\&=\begin{cases}g(1-2x)&\text{if}~x\le\frac12,\\f(2-2x)&\text{if}~x\ge\frac12.\end{cases}\end{align*} Thus the two are indeed the same. 

\item Take the closed path $f(t)=e^{2\pi it}$ on $S^1$. Then note that $(f*f^{-1})(\frac18)=f(\frac14)=i$, while $(f^{-1}*f)(\frac18)=f^{-1}(\frac14)=-i$. 

\item Suppose $i_p*f=f$ and $f$ is not constant. Note that continuity implies that there must exist some $0<t<1$ so that $f(t)\ne p$. Thus there exists some $k\in\NN$ so that $t<1-2^{-k}$. 

We claim, however, that $f$ must be constant on $[0,1-2^{-n}]$ for every $n\in\NN$. We prove this inductively. Clearly, it is true on $n=0$. If it is true on $n-1$, then we know that $i_p*f$ must be equal to $p$ on $[0,\frac12]$, as well as on $[\frac12,1-2^{-n}]$ (note that $1-2^{-n}$ comes from $2(1-2^{-n})-1$, which itself comes from the equation for the star operator). Thus $f$ is constant on $[0,1-2^{-n}]$, as desired. 

Thus it follows that $f(t)=p$, a contradiction. Thus $f$ must have been constant in the first place. 
\end{enumerate} 
\end{exercise} 

\begin{exercise} \leavevmode
Recall that we defined the $\sin(1/x)$ space as the union of $A=\{(0,y):-1\le y\le1\}$ and $G=\{(x,\sin(1/x)):0<x\le1/2\pi\}$. We also know that $A$ and $G$ are the path components of the $\sin(1/x)$ space. Moreover, both $A$ and $G$ are contractible, and so every path in either $A$ or $G$ is nullhomotopic. In particular, we conclude that the fundamental group at any basepoint is trivial.  
\end{exercise} 

\begin{exercise} \leavevmode 
Let $X$ be the $\sin(1/x)$ space. We know that $CX$ is contractible. But consider an open ball around the point $x=((0,0),0)$, that is, the point $(0,0)$ on the ``zeroth'' level of the cone. Consider a small neighborhood (not including the points $(t,1)$, in particular) around this point and pick some element $y=((\ep,\sin(1/\ep)),0)$ in the neighborhood. Now observe that any path between $x$ and $y$ can be projected down to a path between $(0,0)$ and $(\ep,\sin(1/\ep))$ in $X$, which we know does not exist. Hence $CX$ is contractible but not locally path connected. 
\end{exercise}

\begin{exercise} \leavevmode
Note that composition is associative because $\circ$ is. Moreover, the path class of the trivial loop based at $p$ is the identity on $p$. Thus this is a category. 

To see that each morphism in $\cat C$, simply note that the inverse path, i.e., the path $f^{-1}$ taking $t$ to $f(1-t)$, gives a path class $[f^{-1}]$ which works as an inverse to $[f]\in\Hom(p,q)$. 
\end{exercise}

\begin{exercise} \leavevmode 
We simply let $\pi_0$ take $(X,x_0)\in\Cat{Sets}_*$ to the set of all path components of $X$, with basepoint equal to the path component containing $x_0$. It takes a morphism $f\in\Hom((X,x_0),(Y,y_0))$ to the map $\pi_0(f)$ which takes each path component $A$ of $X$ to the path component $B$ of $Y$ which contains $f(A)$. 

Note that this is possible because continuous images of path connected spaces are path connected and hence contained within a single path component of $Y$. Moreover, this is indeed a pointed map because the path component containing $x_0$ must be contained in the path component containing $f(x_0)=y_0$, which is the basepoing of $\pi_0((Y,y_0))$. 

It is easy to check functoriality, completing the proof. 
\end{exercise}

\begin{exercise} \leavevmode
Evidently the only possible path is the constant path at $x_0$. Hence $\pi_1(X,x_0)$ is the trivial group, i.e., $\{1\}$. 
\end{exercise}

\end{document}
