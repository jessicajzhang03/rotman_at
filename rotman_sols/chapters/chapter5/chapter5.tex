\documentclass[../../solutions.tex]{subfiles}

\setcounter{section}{4}

\begin{document}
\section{\texorpdfstring{The Category $\Comp$}{The Category Comp}}
\begin{exercise} \leavevmode
These results all follow directly from the definition of exactness. 
\begin{enumerate}
\item Note that $\ker f=\im 0=0$, and so $f$ is injective. 
\item In this case, we have $\im g=\ker0=C$. 
\item By the previous two parts, we know that $f$ is bijective. 
Because $f$ is a homomorphism as well, it follows that $f$ is an isomorphism. 
\item Either observe that $0\to A\to0\to0$ is exact and apply the previous part, or note that $A\to0$ is injective while $0\to A$ is surjective, implying that $A\cong0$, i.e., that $A=0$. 
\end{enumerate}
\end{exercise}

\begin{exercise} \leavevmode
Note that $f$ is surjective if and only if $\ker g=\im f=B$. 
But $\ker g=B$ if and only if $g$ is the zero map, which is itself true exactly when $\ker h=\im g=0$. 
Since $\ker h=0$ if and only if $h$ is injective, we are done. 
\end{exercise}

\begin{exercise} \leavevmode
We know that $0\to A\xrightarrow{i}B$ implies that $i$ is an injection. 
But because $i$ is a surjection onto its image, this implies that $iA\cong A$. 
Moreover, because $\ker p=\im i=iA$, we know that $B/iA=B/\ker p\cong\im p$. 
Because $p$ is a surjection (see \Cref{5.1}), the result follows. 
\end{exercise}

\begin{exercise} \leavevmode
This amounts, effectively, to following the arrows and the equations given by exactness. 
In more detail, let $f_n:B_n\to C_n$ and $g_n:C_n\to A_{n-1}$. 
Now observe that $B_n=\im h_n=\ker f_n$. 
Thus $f_n$ is the zero map. 
Moreover, because $\ker g_n=\im f_n$, we know that $g_n$ is injective. 
Finally, we have $\im g_n=\ker h_{n-1}$. 
But $h_{n-1}$ is an isomorphism, and so its kernel is trivial. 
Thus $\im g_n=0$. 
Because $g_n$ was injective, it follows that $C_n=0$. 
\end{exercise}

\begin{exercise} \leavevmode
\begin{enumerate}
\item Let $f$ be the map from $A$ to $B$ and $g$ be the map from $B$ to $C$.
Let $\{a_\alpha\}$ and $\{c_\gamma\}$ be maximal independent sets of $A$ and $C$, respectively. 
For every $\alpha$, let $b_\alpha=f(a_\alpha)$. 
For every $\gamma$, pick some $b_\gamma\in g^{-1}(c_\gamma)$, which is possible by surjectivity of $g$.
If $\sum n_\alpha b_\alpha+\sum n'_\gamma b'_\gamma=0$, then we know that \[g\left(\sum n_\alpha b_\alpha+\sum n'_\gamma b'_\gamma\right)=0.\]
But we also know that $\im f=\ker g$, and so $g(b_\alpha)=0$. 
Thus this simply implies that $\sum n'_\gamma c_\gamma=0$, implying that $n'_\gamma=0$. 
But now we know that $\sum n_\alpha b_\alpha=0$, and so injectivity of $f$ implies that $\sum n_\alpha a_\alpha=0$ as well. 
Thus $n_\alpha=0$ for all $\alpha$ as well, and so $\{b_\alpha\}\cup\{b_\gamma\}$ is independent. 
Thus $\rank B\ge\rank A+\rank C$. 

To show the opposite inequality, it suffices to show that $\{b_\alpha\}\cup\{b_\gamma\}$ is \textit{maximally} independent. 
Note that $b\not\in f(A)$. 
Otherwise, we could take $f^{-1}$ on $\{b_\alpha\}\cup\{b\}$, which is not independent. 
Now consider $\{b_\gamma\}\cup\{b\}$. 
If $g(b)=g(b_\gamma)$ for any $\gamma$, then we know by \Cref{5.3} that $b-b_\gamma\in f(A)$. 
Obviously, we cannot have $b-b_\gamma=b_\alpha$ for any $\alpha$, otherwise that would give us our linear dependence. 
Thus $\{b-b_\gamma\}\cup\{b_\alpha\}$ is a subset of $f(A)$ with $\rank A+1$ elements. 
This is not independent, a contradiction. 

\item We prove this by induction. 
The previous part takes care of the base case. 
Consider the following commutative diagrams. 
\[\begin{tikzcd}
0\ar[r] & A_n\ar[r,"f_n"] & A_{n-1}\ar[r,"v"] & A_{n-1}/\im f_n\ar[r] & 0
\end{tikzcd}\]
\[\begin{tikzcd}
0\ar[r] & A_{n-1}/\ker f_{n-1}\ar[r,"\bar f_{n-1}"] & A_{n-2}\ar[r,"f_{n-2}"] & \dots\ar[r,"f_2"] & A_1\ar[r,"f_1"] & A_0\ar[r] & 0.
\end{tikzcd}\]
Here $v$ is the natural map and $\bar f_{n-1}$ is the well-defined map taking $x+\ker f_{n-1}$ to $f_{n-1}(x)$. 

Let $r_i$ be the rank of $A_i$. 
Then the first diagram implies that $r_n-r_{n-1}+r=0$, where $r$ is the rank of $A_{n-1}/\im f_n$. 
Because $\im f_n=\ker f_{n-1}$, the second diagram implies by induction that $r-r_{n-2}+r_{n-3}+\dots=0$. 
Thus we subtract the first from the second to find that $r_n-r_{n-1}+r_{n-2}-\dots=0$, as desired. 
\end{enumerate}
\end{exercise}

\begin{exercise} \leavevmode
If $\partial_n=0$ for all $n$, then we know that $H_n(S_*)=\ker\partial_n/\im\partial_{n+1}=S_n/\{0\}=S_n$. 
\end{exercise}

\begin{exercise} \leavevmode
If $f:S_*\to S'_*$ is an equivalence, then it has an inverse $g:S'_*\to S_*$. 
Thus at every $n$, there is a $g_n:S'_n\to S_n$ so that $g_n\circ f_n=\id_{S_n}$ and $f_n\circ g_n=\id_{S'_n}$. 
It follows that $f_n$ is an isomorphism for every $n$. 

Conversely, suppose $f_n$ is an isomorphism for every $n$. 
Then let $g=\{g_n\}$, where $g_n=f_n^{-1}$. 
It is clear that $f$ and $g$ are inverses, and so $f$ is indeed an equivalence in $\Comp$. 
\end{exercise}

\begin{exercise} \leavevmode
If the former sequence is exact in $\Comp$, then we know that $\im f=\ker g$. 
Then the terms of degree $n$ of both $\im f$ and $\ker g$ must be the same. 
In other words, we must have $\im f_n=\ker g_n$, and so the latter sequence is exact in $\Ab$. 

On the other hand, suppose that the latter sequence is exact for every integer $n$. 
Then we know that the degree $n$ terms of $\ker f$ and $\im g$ are the same. 
Moreover, we know that the differentiation operators are the same because they are defined, in both cases, simply as restrictions of the differentiation operator in $S_*$. 
Thus the two complexes are the same, as desired. 
\end{exercise}

\begin{exercise} \leavevmode
\begin{enumerate}
\item We have the following diagram, where $\bar\partial_n$ represents the map taking $s_n+S'_n\mapsto\partial_n(s_n)+S'_{n-1}$. 
\[\begin{tikzcd}
\dots\ar[r] & S_{n+1}\ar[d,"v_{n+1}"]\ar[r,"\partial_{n+1}"] & S_n\ar[d,"v_n"]\ar[r,"\partial_n"] & S_{n-1}\ar[d,"v_{n-1}"]\ar[r] & \dots \\ 
\dots\ar[r] & S_{n+1}/S'_{n+1}\ar[r,"\bar\partial_{n+1}"] & S_n/S'_n\ar[r,"\bar\partial_n"] & S_{n-1}/S'_{n-1}\ar[r] & \dots 
\end{tikzcd}\]
To show that $v$ is a chain map, we must show that $v_{n-1}\partial_n=\bar\partial_nv_n$ for every $n$. 
Pick a simplex $\sig:\Delta^n\to X$. 
We know that $v_{n-1}(\partial n\sig)=\partial_n\sig+S'_{n-1}$. 
However, we also have $\bar\partial_nv_n\sig=\bar\partial_n(\sig+S'_n)=\partial_n\sig+S'_{n-1}$. 
Thus this is indeed a chain map. 

Moreover, it is obvious that $\ker v_n=S'_n$ for every $n$. 
The definition of a subcomplex implies that the $\partial_n|\ker v_n$ is the operation in $S'_*$. 
Thus $\ker v=S'_*$, as desired. 

\item At each $n$, we know from the previous part that we have the following commutative diagram in $\Ab$. 
\[\begin{tikzcd}
S_n\ar[d,"v_n"]\ar[r,"\partial_n"] & S_{n-1}\ar[d,"v_{n-1}"] \\ 
S_n/\ker f_n\ar[r,"\bar\partial_n"] & S_{n-1}/\ker f_{n-1}
\end{tikzcd}\]
By the first isomorphism theorem for groups, however, we know that there is an isomorphism $\theta_n$ from $S_n/\ker f_n\to\im f_n$ such that $\theta_nv_n=f_n$. 

We claim that $\theta=\{\theta_n\}$ is the desired chain map. 
To see this, observe that \[\theta_{n-1}\bar\partial_n(\sig+\ker f_n)=\theta_{n-1}(\partial_n\sig+\ker f_{n-1})=f_{n-1}\partial_n\sig.\] 
On the other hand, because $\sig+\ker f_n=v_n(\sig)$, we know that \[\partial'_n\theta_n(\sig+\ker f_n)=\partial'_n(f_n(\sig))=\partial_nf_n\sig.\] 
The two final expressions in the above equations are equal, moreover, because $\{f_n\}$ is itself a chain map. 
\end{enumerate}
\end{exercise}

\begin{exercise} \leavevmode
First, note that both $S'_*/(S'_*\cap S''_*)$ and $(S'_*+S''_*)/S''_*$ are well-defined because everything is abelian. 
Now consider the map \begin{align*}\phi:S'_*&\to\frac{S'_*+S''_*}{S''_*}\\S'_n&\mapsto S'_n+S''_*.\end{align*}
Note that we have boundary maps \[\partial'_n:S'_n\to S'_{n-1}\] and \[\overline\partial_n:\frac{S'_n+S''_n}{S''_n}\to\frac{S'_{n-1}+S''_{n-1}}{S''_{n-1}},\] where $\overline\partial_n$ takes $(s'_n+s''_n)+S''_n$ to $(\partial'_ns'_n+\partial''_ns''_n)+S''_{n-1}$. 

We claim that $\phi$ is a chain map.
To see this, it suffices to show that $\phi_{n-1}\partial'_n=\overline\partial_n\phi_n$. But for any $\sig'\in S'_n$, we know that \[\phi_{n-1}\partial'_n(\sig)=\partial'_n\sig+S''_{n-1}=\overline\partial_n(\sig+S''_n)=\overline\partial_n\phi_n(\sig),\] as desired. 
Moreover, the second isomorphism theorem for groups implies that $\phi_n$ is a homomorphism with kernel $S'_n\cap S''_n$, from which it follows that $\phi$ is a chain map with $\ker\phi=S'_*\cap S''_*$. The first isomorphism theorem (\Cref{5.9}) implies the result. 
\end{exercise}

\begin{exercise} \leavevmode
Consider the sequence in the problem, namely 
\[\begin{tikzcd}0\ar[r]&T_*/U_*\ar[r,"i"]&S_*/U_*\ar[r,"p"]&S_*/T_*\ar[r]&0\end{tikzcd}.\]
Clearly, we have \[\im i_n=\{t_n+U_n:t_n\in T_n\}.\] 
Moreover, we know that $p_n(s_n+U_n)=s_n+T_n$, so \[\ker p_n=\{s_n+U_n:s_n\in T_n\}.\] Clearly these are equal. 

It now suffices to prove that $\ker i_n=0$ and $\im p_n=S_n/T_n$. 
But note that $i_n(t_n+U_n)=0$ implies that $t_n\in U_n$. 
Hence $t_n+U_n=0$ as an element of $T_n/U_n$ as well. 
Moreover, consider an arbitrary element $s_n+T_n\in S_n/T_n$. 
It is equal to $p_n(s_n+U_n)$, which proves that $p$ is surjective. 
Hence the sequence of complexes is exact. 
\end{exercise}

\begin{exercise} \leavevmode
% By definition, we have \[H_n\left(\sum S_*^\la\right)=\ker\partial_n/\im\partial_{n+1},\] where $\partial_n=\sum\partial_n^\la$. 
We claim that \[\ker\left(\sum\partial_n^\la\right)=\sum\ker\partial_n^\la.\]
If $\sum s_n^\la\in\ker\left(\sum\partial_n^\la\right)$, then by definition we must have $\partial_n^\la(s_n^\la)=0$ for each $\la$. 
The converse is also clearly true. 
Similarly, we find that \[\im\left(\sum\partial_{n+1}^\la\right)=\left\{\sum\partial_{n+1}^\la(s_{n+1}^\la)\right\}=\sum\im\partial_{n+1}^\la.\]
Thus we conclude that \[H_n\left(\sum S_*^\la\right)=\frac{\ker\partial_n}{\im\partial_{n+1}}=\frac{\sum\ker\partial_n^\la}{\sum\im\partial_{n+1}^\la}=\sum\frac{\ker\partial_n^\la}{\im\partial_{n+1}^\la}=\sum H_n(S_*^\la).\]
\end{exercise}

\begin{exercise} \leavevmode
Suppose $\sum m_\sig\sig+S_n(A)=\sum m'_\sig\sig+S_n(A)$. 
This implies that $m_\sig-m'_\sig=0$ for every $\sig$ with $\im\sig\not\subseteq A$. 
Hence we can pick the unique representative \[\sum_{\im\sig\not\subseteq A}m_\sig\sig=\sum_{\im\sig\not\subseteq A}m'_\sig\sig,\] thus showing that this is indeed the free abelian group generated by $\sig$ with $\im\sig\not\subseteq A$. 
\end{exercise}

\begin{exercise} \leavevmode
\begin{enumerate}
\item It suffices to prove that $p_n$ is surjective and that $\im i_n=\ker p_n$. 
To see that $p_n$ is surjective, consider the following segment of the long exact sequence: 
\[\begin{tikzcd}
B_n\ar[r,"p_n"] & C_n \ar[r] & A_{n-1} \ar[r,"i_{n-1}"] & B_{n-1}
\end{tikzcd}\]
Note that the map $f_n:C_n\to A_{n-1}$ has $\im f_n=\ker i_{n-1}=0$, and so $\ker f_n=C_n$.
Thus $\im f_n=C_n$, proving surjectivity. 

To see that $\im i_n=\ker p_n$, simply consider the following segment: 
\[\begin{tikzcd}
A_n\ar[r,"i_n"] & B_n \ar[r,"p_n"] & C_n
\end{tikzcd}\]
The result immediately follows. 

\item There exists a map $r$ with $r\circ i=\id_A$. 
Thus $r_*\circ i_*=\id_{H_n(A)}$, from which it follows that $i_*$ is injective. 
Theorem 5.8 gives an exact sequence 
\[\begin{tikzcd}
\dots\ar[r] & H_n(A)\ar[r,"i_n"] & H_n(X)\ar[r,"p_n"] & H_n(X,A)\ar[r,"d"] & \dots
\end{tikzcd}\]
where $p_*$ is induced by the quotient map $S_*(X)\to S_*(X)/S_*(A)$.
Since $i_*$ injective implies that $i_n$ is injective, we can apply the previous part to find an exact sequence 
\[\begin{tikzcd}
0\ar[r] & H_n(A)\ar[r,"i_n"] & H_n(X)\ar[r,"p_n"] & H_n(X,A)\ar[r] & 0
\end{tikzcd}\]
Then \Cref{5.3} implies that $H_n\oplus H_n(X,A)=H_n(X)$, as desired. 

\item We now have $i\circ r\simeq\id_X$ as well.
In particular, since $A$ and $X$ have the same homotopy type, we must have $H_n(A)\cong H_n(X)$ by Corollary 4.24.
Thus $i_n$ in the exact sequence given in the previous part must be the identity, and so $\ker p_n=\im i_n=H_n(X)$. 
But since $p_n$ is surjective, it follows that $H_n(X,A)=0$, as desired. 
\end{enumerate}
\end{exercise}

\begin{exercise} \leavevmode
We prove this in cases. We will use the follow commutative diagram, where the columns are exact: 
\[\begin{tikzcd}[row sep=large,column sep=large]
& 0\ar[d] & 0\ar[d] & 0\ar[d] & \\ 
\dots\ar[r] & S'_{n+1} \ar[r,"\partial'_{n+1}"] \ar[d,"i_{n+1}"] & S'_n \ar[r,"\partial'_n"] \ar[d,"i_n"] & S'_{n-1}\ar[r] \ar[d,"i_{n-1}"] & \dots \\ 
\dots\ar[r] & S_{n+1} \ar[r,"\partial_{n+1}"] \ar[d,"p_{n+1}"] & S_n \ar[r,"\partial_n"] \ar[d,"p_n"] & S_{n-1}\ar[r] \ar[d,"p_n"] & \dots \\ 
\dots\ar[r] & S''_{n+1} \ar[r,"\partial''_{n+1}"] \ar[d] & S''_n \ar[r,"\partial''_n"] \ar[d] & S''_{n-1}\ar[r] \ar[d] & \dots \\ 
& 0 & 0 & 0 & 
\end{tikzcd}\]

\begin{case}
$S_*$ and $S'_*$ are acyclic. 
\end{case}
We would like to show than $Z''_n=\ker\partial''_n$ is equal to $B''_n=\im\partial''_{n+1}$. 
We already know that $B''_n\subseteq Z''_n$. 
Surjectivity of $p$ implies that we can rewrite $Z''_n$ as \[Z''_n=p_n(\ker(\partial''_np_n))=p_n(\ker(p_{n-1}\partial_n)).\] 
This, in turn, can be written as \[Z''_n=\{p_ns_n:p_{n-1}\partial_ns_n=0\}.\]
On the other hand, we can rewrite $B''_n$ as \[B''_n=\im(\partial''_{n+1}p_{n+1}=\im(p_n\partial_{n+1}))=p_n\im\partial_{n+1}.\]
Since $S_*$ is acyclic, we know that $\im\partial_{n+1}=\ker\partial_n$, and so we find that \[B''_n=\{p_nz_n:\partial_nz_n=0\}.\]

Now consider an arbitrary element $p_ns_n\in Z''_n$. 
Since $p_{n-1}\partial_ns_n=0$, we know that $\partial_ns_n\in\ker p_{n-1}=\im i_{n-1}$, where again we use the fact that $S_*$ is acyclic. 
Injectivity of $i_{n-1}$ implies the existence of a unique $s'_{n-1}\in S'_{n-1}$ with $i_{n-1}s'_{n-1}=\partial_ns_n$. 
We know, however, that $\partial_{n-1}\partial_n=0$, and so \[0=\partial_{n-1}\partial_ns=\partial_{n-1}i_{n-1}s'_{n-1}=i_{n-2}\partial'_{n-1}s'_{n-1}.\]
Since $i_{n-2}$ is injective, it follows that $\partial'_{n-1}s_{n-1}=0$, and so acyclicity of $S'_*$ implies that $s_{n-1}\in\ker\partial'_{n-1}=\im\partial'_n$. 
In particular, we can write $s'_{n-1}=\partial'_ns'_n$. 

Now notice that \[\partial_ni_ns'_n=i_{n-1}\partial'_ns'_n=i_{n-1}s'_{n-1}=\partial_ns_n,\] where the last equality follows from the definition of $s'_{n-1}$. 
We know that $z_n=s_n-i_ns'_n\in Z_n$ since $\partial'_n$ is a homomorphism. 
But we also know that \[p_nz_n=p_n(s_n-i_ns'_n)=p_ns_n-p_ni_ns'_n=p_ns_n,\] where we use exactness of the columns. 
In other words, we have a $z_n$ with $\partial_nz_n=0$, such that $p_nz_n=p_ns_n$. 
Thus $p_ns_n\in B''_n$, proving that $Z''_n=B''_n$. 
To be even more explicit, this implies that $H''_n=Z''_n/B''_n=0$ for all $n$, proving that $S''_*$ is an acyclic complex as well. 

\begin{case}
$S'_*$ and $S''_*$ are acyclic. 
\end{case}

Suppose $s_n\in Z_n$, i.e., that $\partial_ns_n=0$. 
Then $\partial''_np_n=p_{n-1}\partial_n$ implies that $p_ns_n\in\ker\partial''_n=\im\partial''_{n+1}$. 
Hence write $p_ns_n=\partial''_{n+1}s''_{n+1}$. 
Since $p_{n+1}$ is surjective, we can find $s_{n+1}$ with $p_{n+1}s_{n+1}=s''_{n+1}$, and so \[p_n\partial_{n+1}s_{n+1}=\partial''_{n+1}p_{n+1}s_{n+1}=p_ns_n.\]
But then we know that $\partial_{n+1}s_{n+1}-s_n\in\ker p_n=\im i_n$. 
Thus there exists a unique $s'_n$ with $i_ns'_n=\partial_{n+1}s_{n+1}-s_n$. 
We can take $\partial_n$ of both sides to find that \[0=\partial_n\partial_{n+1}s_{n+1}-\partial_ns_n=\partial_ni_ns'_n=i_{n-1}\partial'_ns'_n,\] and so it follows that $\partial'_ns'_n=0$. 
In particular, we know that $s'_n\in\im\partial'_{n+1}$, so we can find $s'_{n+1}$ whose boundary is $s'_n$. 
Recall that we had \[s_n=\partial_{n+1}s_{n+1}-i_ns'_n.\]
But the last term is equal to $i_n\partial'_{n+1}s'_{n+1}=\partial_{n+1}i_{n+1}s'_{n+1}$, and so this is in turn equal to \[s_n=\partial_{n+1}(s_{n+1}-i_{n+1}s'_{n+1}).\] 
This proves that $s_n\in B_n$, and so $Z_n=B_n$. 

\begin{case}
$S_*$ and $S''_*$ are acyclic. 
\end{case}

This final case is handled similarly to the first two, but we lay out the details below. 
Let $s'_n\in\ker\partial'_n=Z'_n$ be arbitrary. 
Then $i_ns'_n\in\ker\partial_n=\im\partial_{n+1}$, and so \[i_ns'_n=\partial_{n+1}s_{n+1}\] for some $s_{n+1}$. 
We know that $p_{n+1}s_{n+1}\in\ker\partial''_{n+1}=\im\partial''_{n+2}$ because $p_ni_ns'_n=0$. 
Hence there exists $s''_{n+2}$ with $\partial''_{n+2}s''_{n+2}=p_{n+1}s_{n+1}$. 
But then it follows that \[p_{n+1}\partial_{n+2}s_{n+2}=\partial''_{n+2}p_{n+2}s_{n+2}=p_{n+1}s_{n+1},\] from which it follows that $s_{n+1}-\partial_{n+2}s_{n+2}\in\ker p_{n+1}=\im i_{n+1}$. 
Thus there exists $s'_{n+1}$ with $i_{n+1}s'_{n+1}=s_{n+1}-\partial_{n+2}s_{n+2}$.
We then find that \[i_n\partial'_{n+1}s'_{n+1}=\partial_{n+1}i_{n+1}s'_{n+1}=\partial_{n+1}s_{n+1}=i_ns'_n.\]
Injectivity implies $s'_n=\partial'_{n+1}s'_{n+1}\in B_n$, thus proving the final case. 
\end{exercise}

\begin{exercise} \leavevmode
To show that $f_\#(Z_n(X,A))\subseteq Z_n(X',A')$, consider $\gamma\in Z_n(X,A)$. 
Note that $\gamma\in S_n(X)$, and so Lemma 4.8 implies that \[\partial'_nf_\#\gamma=f_\#\partial_n\gamma.\] 
We know, moreover, that $\partial_n\gamma=\sum m_\sig\sig$, where the sum ranges over all $\sig$ with $\im\sig\subseteq A$. 
Thus it follows that \[\partial'_nf_\#\gamma=f_\#\left(\sum m_\sig\sig\right)=\sum_{\im(f\sig)\subseteq f(A)} m_\sig f\sig.\] 
Since $f\sig$ is a simplex into $X'$ with image contained in $A'$, it follows that this is an element of $S_{n-1}(A')$. 
Hence we conclude that $f_\#\gamma\in Z_n(X',A')$, as desired. 

The proof for boundaries is similar. 
\end{exercise}

\begin{exercise} \leavevmode
As defined, we have that $f_\#:H_n(X,A)\to H_n(X',A')$ is given by \[f_\#:\overline\ga+\im\overline\partial_{n+1}\mapsto f_\#\left(\overline\ga\right)+\im\overline{\partial'}_{n+1},\] where $\overline\partial$ and $\overline{\partial'}$ denote the boundary maps of the quotient complexes $S_*(X)/S_*(A)$ and $S_*(X')/S_*(A')$, respectively, and where \[\overline\ga\in\ker\overline\partial_n=Z_n(X,A)/S_n(A).\] 
The third isomorphism theorem gives an isomorphism \[H_n(X,A)=\frac{Z_n(X,A)/S_n(A)}{B_n(X,A)/S_n(A)}\to\frac{Z_n(X,A)}{B_n(X,A)}\] which takes \[\overline\ga+B_n(X,A)/S_n(A)\mapsto\ga+B_n(X,A).\] 
Since $f_\#(\overline\ga)=\overline{f_\#(\ga)}$, we find that, thinking of $f_\#$ as a map from $Z_n(X,A)/B_n(X,A)$ to a map from $Z_n(X',A')/B_n(X',A')$, it takes \[\ga+B_n(X,A)\mapsto f_\#(\ga)+B_n(X',A'),\] as desired.
\end{exercise}

\begin{exercise} \leavevmode
Recall the definition of $\ep_i:\Delta^{n-1}\to\Delta^n$ as taking $\{e_0,\dots,e_{n-1}\}$ to $\{e_0,\dots,\hat e_i,\dots,e_{n-1}\}$. 
Thus we have \[\partial_n\sig=\sum_{i=0}^n(-1)^i\sig\ep_i.\]
Since $\sig\ep_i:\Delta^{n-1}\to X$ has image in $A$ by hypothesis, this is in $S_{n-1}(A)$, as desired. 
\end{exercise}

\begin{exercise} \leavevmode
By Theorem 5.6, it is sufficient to show that we have a short exact sequence \[\begin{tikzcd}0\ar[r] & \widetilde S_*(A)\ar[r] & \widetilde S_*(X)\ar[r] & S_*(X,A)\ar[r] & 0.\end{tikzcd}\]
When $n\ge1$, this is clear by Theorem 5.8. 
When $n=0$, we have the sequence \[0\to\ZZ\to\ZZ\to0\to0,\] which is easily verified to be exact. 
\end{exercise}

\begin{exercise} \leavevmode
Consider the exact sequence \[\begin{tikzcd}H_1(CX)\ar[r] & H_1(CX,X)\ar[r] & \widetilde H_0(X)\ar[r] & \widetilde H_0(CX).\end{tikzcd}\]
Note that $H_1(CX)=0$ because $CX$ is contractible.
On the other hand, Corollary 5.18 implies that $\widetilde H_0(X)\cong\ZZ^4$, while $\widetilde H_0(CX)\cong0$. 
Thus the map $H_1(CX,X)\to\widetilde H_0(X)$ is surjective.
Moreover, its kernel is equal to the image of the map $H_1(CX)\to H_1(CX,X)$, which is simply 0 since $H_1(CX)=0$. 
Thus the map is also injective, from which it immediately follows that $H_1(CX,X)\cong\ZZ^4$. 
\end{exercise}

\begin{exercise} \leavevmode
Consider the exact sequence 
\[\begin{tikzcd}
    \widetilde H_1(S^0)\ar[r] & \widetilde H_1(S^1)\ar[r] & H_1(S^1,S^0)\ar[r] & \widetilde H_0(S^0)\ar[r] & \widetilde H_0(S^1),
\end{tikzcd}\] 
which is simply equal to 
\[\begin{tikzcd}
    0\ar[r] & \ZZ\ar[r] & \boxed{\phantom{\widetilde HS}}\ar[r] & \ZZ\ar[r] & 0.
\end{tikzcd}\]
Now note that the first map has $\im=0$, so the second map has $\ker=0$. 
Thus the second map has $\im\cong\ZZ$, and so the third map has $\ker\cong\ZZ$. 
Yet we also know that the last map is the zero map, and so the third map has $\im=\ZZ$, from which it follows that the $H_1(S^1,S^0)=\ZZ\times\ZZ$. 
\end{exercise}

\begin{exercise} \leavevmode
When $n=0$, this follows from the exact sequence \[\begin{tikzcd}\widetilde H_0(X)\ar[r] & \widetilde H_0(X)\ar[r] & H_0(X,X)\ar[r] & 0\end{tikzcd}.\]
After all, the first map is the identity, and so the second map is the zero map. 
But the second map is surjective, and so $H_0(X,X)=0$. 

For $n>0$, we have the exact sequence 
\[\begin{tikzcd}
    \widetilde H_n(X)\ar[r] & \widetilde H_n(X)\ar[r] & H_n(X,X)\ar[r] & \widetilde H_{n-1}(X)\ar[r] & \widetilde H_{n-1}(X).
\end{tikzcd}\]
The first map is the identity, and so the second map is everywhere zero. 
Thus the image of the second map, which is the kernel of the third map too, is equal to 0.
Since the kernel of the last map, which is 0 (the map is the identity), is equal to the image of the third map, it follows that the third map is everywhere zero. 
Hence the third map is injective, but also everywhere zero, and so $H_n(X,X)$ must have been 0 in the first place. 
\end{exercise}
\end{document}