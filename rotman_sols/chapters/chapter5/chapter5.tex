\documentclass[../../solutions.tex]{subfiles}

\setcounter{section}{4}

\begin{document}
\section{\texorpdfstring{The Category $\Cat{Comp}$}{The Category Comp}}
\begin{exercise} \leavevmode
These results all follow directly from the definition of exactness. 
\begin{enumerate}
\item Note that $\ker f=\im 0=0$, and so $f$ is injective. 
\item In this case, we have $\im g=\ker0=C$. 
\item By the previous two parts, we know that $f$ is bijective. 
Because $f$ is a homomorphism as well, it follows that $f$ is an isomorphism. 
\item Either observe that $0\to A\to0\to0$ is exact and apply the previous part, or note that $A\to0$ is injective while $0\to A$ is surjective, implying that $A\cong0$, i.e., that $A=0$. 
\end{enumerate}
\end{exercise}

\begin{exercise} \leavevmode
Note that $f$ is surjective if and only if $\ker g=\im f=B$. 
But $\ker g=B$ if and only if $g$ is the zero map, which is itself true exactly when $\ker h=\im g=0$. 
Since $\ker h=0$ if and only if $h$ is injective, we are done. 
\end{exercise}

\begin{exercise} \leavevmode
We know that $0\to A\xrightarrow{i}B$ implies that $i$ is an injection. 
But because $i$ is a surjection onto its image, this implies that $iA\cong A$. 
Moreover, because $\ker p=\im i=iA$, we know that $B/iA=B/\ker p\cong\im p$. 
Because $p$ is a surjection (see \Cref{5.1}), the result follows. 
\end{exercise}

\begin{exercise} \leavevmode
This amounts, effectively, to following the arrows and the equations given by exactness. 
In more detail, let $f_n:B_n\to C_n$ and $g_n:C_n\to A_{n-1}$. 
Now observe that $B_n=\im h_n=\ker f_n$. 
Thus $f_n$ is the zero map. 
Moreover, because $\ker g_n=\im f_n$, we know that $g_n$ is injective. 
Finally, we have $\im g_n=\ker h_{n-1}$. 
But $h_{n-1}$ is an isomorphism, and so its kernel is trivial. 
Thus $\im g_n=0$. 
Because $g_n$ was injective, it follows that $C_n=0$. 
\end{exercise}

\begin{exercise} \leavevmode
\begin{enumerate}
\item 
\end{enumerate}
\end{exercise}
\end{document}