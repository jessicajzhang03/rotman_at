\documentclass[../../solutions.tex]{subfiles}

\setcounter{section}{7}

\begin{document}
\section{CW Complexes}
\subsection{Hausdorff Quotient Spaces}
\begin{exercise} \leavevmode
For any $x,y\ne0$, let $\la=xy^{-1}$.
Then $x=\la y$, so $[x]=[y]$.
Hence $FP^0$ is just a single point.
\end{exercise}

\begin{exercise} \leavevmode
In each case, first use the fact that each space is a division ring to map $[x_0,x_1]\mapsto[1,x]=[1,x_0^{-1}x_1]$, then divide each term by $\sqrt{1+||x||^2}$ so that the result has magnitude 1.
This gives the desired homeomorphisms.
\end{exercise}

\begin{exercise} \leavevmode
Note that $U(\RR)=\{\pm1\}\homeo S^0$.
To see the homeomorphism for $\CC$, use the map $e^{i\theta}\mapsto\cos\theta+i\sin\theta$.
Finally, to see the homeomorphism for $\mathbb H$, write a given quaternion as an ordered quadruple, and divide by its magnitude.
\end{exercise}

\begin{exercise} \leavevmode
Note that the real projective plane is just the quotient of $S^2$, where antipodal points are identified.
This is in turn equal to the quotient of $\RR^3$ where points on a line through the origin are identified, i.e., $\RR P^2$.
\end{exercise}

\begin{exercise} \leavevmode
Use the map $[x]\mapsto[x/|x|]$ to get a homeomorphism $\RR P^n\mapsto S^n/\ecls$.
\end{exercise}

\begin{exercise} \leavevmode
Consider the map $f$ taking $[x_1,\dots,x_{2n+2}]\in S^{2n+1}/\ecls$ to $[z_1,\dots,z_{n+1}]\in\CC P^n$, where $z_j=x_{2j-1}+ix_{2j}$ for each $j$.
This is easily seen to be well-defined.
If $x,y\in S^{2n+1}$ with $x\sim y$, then $x=\la y$ for some $\la$ with $|\la|=1$.
If $f(x)=[z_i]$ and $f(y)=[w_i]$, then notice that $(z_i)=\la(w_i)$ as well, so $[z_i]=[w_i]$.
\end{exercise}

\begin{exercise} \leavevmode
The same argument as above holds, this time by defining $z_n=x_{4n-3}+ix_{4n-2}+jx_{4n-1}+kx_{4n}$.
\end{exercise}

\subsection{Attaching Cells}
\begin{exercise} \leavevmode
By Corollary 1.9, it suffices to show that $\alpha\amalg\beta$ is constant on the fibers of $v$.
Thus suppose $v(s)=v(t)$.
It is sufficient to suppose that $t=f(s)$ and $s\in A$, since the relation $\ecls$ is generated by all $(a,f(a))$.
But if $t=f(s)$, then we have
\[(\alpha\amalg\beta)(s)=\alpha(s)=\beta(f(s))=\beta(t)=(\alpha\amalg\beta)(t).\]
This proves the result.
\end{exercise}

\begin{exercise} \leavevmode
\begin{enumerate}
\item
It is clear that $B$ and $B^{-1}$ are contained in the equivalence relation generated by $B$.
Note that $D$ is as well due to reflexivity.
Finally, $K$ is attained by $(a,f(a))(f(a),a')=(a,a')$, where $f(a')=f(a)$.
Now note that repeating this with $(a',f(a'))$ simply takes us back to $(a,f(a))$, so there are no other elements in the equivalence relation.

\item
Simply note that
\begin{align*}
K&=\{(a,a'):f(a)=f(a')\}\\&=(f\times f)^{-1}\{(x,y)\in\im(f\times f):x=y\}\\&=(f\times f)^{-1}(\Dlt\cap\im(f\times f)).
\end{align*}
This is exactly what we wanted.
\end{enumerate}
\end{exercise}

\begin{exercise} \leavevmode
It is easy to verify that the diagram commutes.
Now suppose we have some $Z$ with functions $\alpha:X\to Z$ and $\beta:Y\to Z$ so that $\beta\circ f\alpha\circ i$.
We would like to find a function $\phi:X\amalg_fY\to Z$ making the pushout diagram commute.

We will first show that, if such a function exists, then it must be unique.
Since the maps $X\to X\amalg_fY$ and $Y\to X\amalg_fY$ are induced by $v$, it follows that $\phi\circ v|X=\alpha$ and similarly for $Y$ and $\beta$.
Hence $\phi\circ v$ would have to be equal to $\alpha\amalg\beta$, i.e., $\phi=(\alpha\amalg\beta)\circ v^{-1}$.
Note that \Cref{8.8} applies, so this is indeed a well-defined map.
\end{exercise}

\begin{exercise} \leavevmode
The only case too check is if $x\in X$ and $y\in Y$.
Note that there exists some $a\in A$, so consider the path from $x$ to $a=f(a)$ to $y$.
Hence $X\amalg_fY$ is path-connected.
\end{exercise}

\begin{exercise} \leavevmode
\begin{enumerate}
\item
The equation given in the hint follows from \Cref{8.9}.
Now, for the forwards direction, observe that $v(C)$ closed implies $v^{-1}v(C)$ closed in $X\amalg Y$.
Hence its intersection with $Y$ is closed in $Y$.
But notice that $f(C\cap A)\cap Y=f(C\cap A)$.
Furthermore, we know that $f^{-1}(f(C\cap A))$ and $f^{-1}(C\cap Y)$ are completely disjoint from $Y$.
Thus
\[v^{-1}v(C)\cap Y=(C\cap Y)\cup f(C\cap A)\]
is closed in $Y$, as desired.

Going backwards, observe that $v^{-1}v(C)\cap Y$ is closed in $Y$, using the hypothesis and the argument above.
Moreover, since $f$ is continuous, the hypothesis implies that
\[f^{-1}((C\cap Y)\cup f(C\cap A))=f^{-1}(C\cap Y)\cup f^{-1}(f(C\cap A))\]
is closed.
Since $C\cap X$ is closed in $X$ and $f(C\cap A)\cap X=\emptyset$, this implies that $v^{-1}v(C)\cap X$ is closed in $X$.
Hence $v^{-1}v(C)$ is closed in $X\amalg Y$.
Since $v$ is an identification, it follows that $v(C)$ is closed in $X\amalg_fY$.

\item
The function is clearly bijective and continuous.
To see that it is a homeomorphism, we will show that it is a closed map.
Thus suppose $C\subseteq Y$ is closed.
Obviously, $i(C)\subseteq X\amalg Y$ is closed and has empty intersection with $X$.
Then to see that $v(i(C))$ is closed in $X\amalg_fY$, simply use the previous part.
In particular, observe that
\[i(C)\cap Y=i(C),\]
which is closed in $Y$, while
\[f(i(C)\cap A)=\emptyset,\]
which is also closed, so that their union is closed.
Thus $v(i(C))$ is closed in $X\amalg_fY$, proving that the given function is a homeomorphism.

\item
Again, this is clearly bijective and continuous.
Since $X-A$ is open in $X$, if $U$ is open in $X-A$, then it is also open in $X$.
Note that $i(U)$ is open in $X\amalg Y$.
Now note that $i(U)^c$ is closed in $X\amalg Y$, and its intersection with $X$ is $U^c$, which is closed in $X$.
Moreover, we know that its intersection with $Y$ is $Y$ itself, while $f(i(U)^c\cap A)=f(A)$.
Since $Y\cup f(A)=Y$, which is closed in $Y$, part (i) implies that $v(i(U)^c)$, which, by surjectivity of $v$, is exactly $v(i(U))^c$, is closed.
Thus $v(i(U))$ is open, proving that this is an open, bijective, continuous map, thus a homeomorphism.

\item
Note that $\Phi$ takes $A\subseteq X$ to $A\subseteq X\amalg Y$, which is then exactly equal to the attached region of $X\amalg_fY$.
\end{enumerate}
\end{exercise}

\begin{exercise} \leavevmode
\begin{enumerate}
\item
Since $f$ is from a compact set to a Hausdorff set, it is closed.
Let $C$ be closed.
Then $A$ being compact implies that it is closed, so $C\cap A$ is closed in $X$.
Thus $f(C\cap A)$ is closed in $Y$.
Since $C\cap Y$ is closed in $Y$, it follows from part (i) that $v(C)$ is closed in $X\amalg_fY$.

\item
First, suppose that $z\in\im\Phi|A$.
Then there is some $x\in X$ with $v(i(x))=z$, so $i(x)$ is in the fiber.
We know that $\{z\}$ is closed because $Y$ is Hausdorff, so $v^{-1}(z)$ is also closed.
Since $v^{-1}(z)\subseteq A$, and closed subsets of compact sets are compact, it follows that $v^{-1}(z)$ is compact.

Otherwise, we know that we can use either the homeomorphism in \Cref{8.12}(ii) or the homeomorphism $\Phi|(X-A)$ to show that $v^{-1}(z)=\{z\}$, which is indeed a nonempty compact subset of $X$.
\end{enumerate}
\end{exercise}

\begin{exercise} \leavevmode
This is just invariance of boundary.
Alternatively, see the proof of Lemma 8.15.
\end{exercise}

\begin{exercise} \leavevmode
If $n=0$, this is obviously true.
Otherwise, let $e=s-\dot s\homeo D^{n-1}-S^{n-1}$ and let $Y=|K^{(n-1)}|$ be a closed subset of $|K|$ (since it's the finite union of (closed) simplices).
Then $e\cap Y=\emptyset$ and $e$ is an $n$-cell.
Hence Theorem 8.7 says that we need only exhibit a relative homeomorphism $\Phi:(D^n,S^{n-1})\to(e\cup Y,Y)$.
But letting $\Phi$ be the obvious homeomorphism from $D^n$ to $s$ works.
\end{exercise}

\begin{exercise} \leavevmode
Write $Y=\{y\}$.
Then define the relative homeomorphism $\Phi:(D^n,S^{n-1})\to(e^n\cup Y,Y)$ which takes $D^n-S^{n-1}$ to $e^n$ in the obvious way, and takes $x\in S^{n-1}$ to $y$.
Theorrem 8.7 tells us that the attachment of $D^n$ to $Y$ along $f=\Phi|S^{n-1}$ is a homeomorphisms between $D^n/\partial D^n=S^n$ and $e^n\cup Y\homeo e^n\cup e^0$.
\end{exercise}

\subsection{Homology and Attaching Cells}
\begin{exercise} \leavevmode
Note that $\chi(K)=1-2+1=0$, so $\rank H_2(K)+1=\rank H_1(K)$.
Furthermore, doing the same thing as with the torus in Example 8.7, we see that the projections are $f\alpha*f\alpha_1^{-1}$, which has degree 0, and $f\beta*f\beta_1$, which has degree 2.
Thus, since $H_1(S^1\vee S^1)\cong H_1(S^1)\oplus H_1(S^1)$, we can consider $f_*$ to be the map $x\mapsto(0,2x)$.
It has trivial kernel and image isomorphic to $\ZZ\oplus\ZZ/2\ZZ$.
Working through the exact sequence in Theorem 8.11 gives the result.
\end{exercise}

\begin{exercise} \leavevmode
There are a couple typos here:
In the first part, the wedge for $M$ is of $2h$ circles, and in the second part, we should have $\chi(M)=2-2h$, not $\chi(M)=h$.

\begin{enumerate}
\item
Note that each $(\alpha_i,\alpha_i^{-1})$ and $(\beta_i,\beta_i^{-1})$ pair gives a circle.
Since all the vertices are identified with each other, this gives us the desired wedge product.
More formally, we can define a function $\Phi$ from a polygon $P$ to $W$, and let $f=\Phi|\partial P$.
Then $f\alpha_i=(f\alpha_i^{-1})^{-1}$, and similarly for $\beta$, which gives us our $2h$ circles.
A similar argument can be done for $M'$.

\item
Note that $H_2(S^1\vee\dots\vee S^1)=0$.
Thus we have the following exact sequence:
\[
\begin{tikzcd}[column sep=small]
0\ar[r] & H_2(M)\ar[r] & H_1(S^1)\ar[r,"f_*"] & H_1(S^1\vee\dots\vee S^1)\ar[r,"i_*"] & H_1(M)\ar[r] & \ZZ\ar[r] & \ZZ^2\ar[r] & \ZZ\ar[r] & 0
\end{tikzcd},
\]
where the last few terms are just $H_0(S^1)$, $\ZZ\oplus H_0(S^1\vee\dots\vee S^1)$, and $H_0(M)$, since all three spaces are path-connected.
The fact that this sequence is exact implies that $\ZZ^2\to\ZZ$ is a surjection, so the map $\ZZ\to\ZZ^2$ is an injection.
Hence $H_1(M)\to\ZZ$ is the zero map.
Thus $i_*$ is surjective and has kernel (isomorphic to) $\ZZ^{2n}/H_1(M)$.

Looking at the maps from left to right now, observe that $H_2(M)\to H_1(S^1)=\ZZ$ is injective.
Thus $\ker f_*=H_2(M)$, so $\im f_*=H_1(S^1)/H_2(M)$.
But $\ker i_*=\im f_*$, and so it follows that
\[\chi(M)=\rank H_2(M)-\rank H_1(M)+\rank H_0(M)=2-2h,\]
where we use the fact that $\rank H_0(M)=1$.

Now notice that the same argument as in Example 8.7 implies that $f_*$ is the zero map.
Thus
\[H_2(M)=\ker f_*=H_1(S^1)=\ZZ.\]
For $H_1(M)$, since the flanking terms are torsion-free, it follows that $H_1(M)$ is also torsion-free.
Since it has rank $2h$, the result follows.

\item
The same argument as before shows that $\chi(M')=2-n$.
This time, however, the map $f_*$ is not the zero map.
In particular, by composing with projections, we find that $f_*:H_1(S^1)\to H_1(S^1\vee\dots\vee S^1)$ takes $x\mapsto(2x,\dots,2x)$, where we have identified $H_1(S^1\vee\dots\vee S^1)$ with $H_1(S^1)\oplus\dots\oplus H_1(S^1)$.

In particular, we have $\ker f_*=0$ and $\im f_*=(\ZZ/2\ZZ)^n$.
The argument before shows that $\ker f_*=H_2(M')$, and so $H_2(M')=0$.
Using the Euler characteristic (i.e., a rank argument), we can conclude that $\rank H_1(M')=n-1$.
(Note that, this time, the first homology group isn't torsion-free, thanks to the $\ZZ/2\ZZ$ terms.)

\item
We first consider $M$.
Note that it only has one vertex, say $v$.
Thus, with chains
\[E_2=\langle W\rangle,\quad E_1=\langle\alpha_1\rangle\oplus\langle\beta_1\rangle\oplus\dots\oplus\langle\alpha_n\rangle\oplus\langle\beta_n\rangle,\quad E_0=\langle v\rangle,\]
we have
\[\partial W=\alpha_1+\beta_1-\alpha_1-\beta_1+\dots=0,\quad\partial\alpha_i=\partial\beta_i=v-v=0,\quad\partial v=0.\]
Hence it follows that
\[
\begin{tabular}{l l l}
$Z_2=\langle P\rangle$, & $Z_1=\langle\alpha_1\rangle\oplus\langle \beta_1\rangle\oplus\dots$, & $Z_0=\langle v\rangle$, \\
$B_2=0$, & $B_1=0$, & $B_0=0$.
\end{tabular}
\]
Thus we have
\[H_2(M)=\ZZ,\quad H_1(M)=\ZZ^{2h},\quad H_0(M)=\ZZ.\]

Now, for $M'$, with the natural chains, we have
\[\partial P=2\alpha_1+\dots+2\alpha_n,\quad\partial\alpha_i=0,\quad\partial v=0.\]
Thus we conclude that
\[
\begin{tabular}{l l l}
$Z_2=0$& $Z_1=\langle\alpha_1\rangle\oplus\dots\oplus\langle\alpha_n\rangle$, & $Z_0=\langle v\rangle$, \\
$B_2=0$, & $B_1=\langle2(\alpha_1+\dots+\alpha_n)\rangle$, & $B_0=0$.
\end{tabular}
\]
This gives us that
\[H_2(M')=0,\quad H_1(M')=\ZZ^n/2\ZZ=\ZZ/2\ZZ\oplus\ZZ^{n-1},\quad H_0(M')=\ZZ,\]
which coincides with the previous parts.
\end{enumerate}
\end{exercise}

\subsection{CW Complexes}
\begin{exercise} \leavevmode
Note that $U\subseteq X$ is open if and only if $U^c\subseteq X$ is closed, which is in turn the case if and only if $U^c\cap A_j$ is closed in $A_j$ for all $j\in J$.
But $U\cap A_j=A_j-U^c\cap A_j$, so this last conditiono is true if and only if $U\cap A_j$ is open in $A_j$ for $j\in J$.
\end{exercise}

\begin{exercise} \leavevmode
It is obvious that $\{Y\cap A_j\}$ fits the conditions (i)--(iii).
Now note that if $F\subseteq Y$ is closed in the subspace topology, then $F=Y\cap F'$ for some closed $F'\subseteq X$.
Hence
\[F\cap(Y\cap A_j)=Y\cap F'\cap Y\cap F_j=(Y\cap A_j)\cap F'.\]
Of course, this is closed in $Y\cap A_j$, so $F$ must be closed in the weak topology.

Now suppose $F$ is closed in the weak topology.
Then $F\cap(Y\cap A_j)$ is closed in all $Y\cap A_j$.
Since $Y$ is closed, we know that $F\cap Y\cap A_j$ must also be closed in $A_j$.
This is true for all $j$, so $F\cap Y$ is closed in the weak topology on $X$, i.e., as a subset of $X$.
Thus $F=F\cap Y$ is closed as a subspace of $Y$.
\end{exercise}

\begin{exercise} \leavevmode
By Lemma 8.20, we know that $A$ closed in $X$ implies that $A\cap X'$ is closed in $X'$ for all finite subcomplexes $X'$.
Theorem 8.19 says that this implies that, for all compact $K$ of $X$, we must have $A\cap K$ closed in $K$.
Thus, by definition, it follows that $A$ is closed in the weak topology generated by compact subsets.

Now suppose $A\cap K$ is closed in $K$ for all compact $K$.
Since finite subcomplexes are compact, it follows that $A\cap X'$ is closed in $X'$ for all such $X'$.
Hence $A$ is closed in $X$.
\end{exercise}

\begin{exercise} \leavevmode
To see that $X^{(0)}$ is discrete, simply let $A\subseteq X^{(0)}$.
Then $A\cap\bar e$ is the finite union of 0-cells, and hence is closed.

To see that the 0-skeleton is closed, note that $X^{(0)}\cap\bar e$ is a finite union of 0-cells, and thus is closed in $\bar e$.
This is true for all $e$, so $X^{(0)}$ is closed.
\end{exercise}

\begin{exercise} \leavevmode
If $A$ is closed, then obviously $A\cap X^{(n)}$ is closed in $X^{(n)}$.
Now if $A\cap X^{(n)}$ is closed in $X^{(n)}$ for each $n$, pick $X'$ to be any finite complex.
Let $n$ be the highest dimension of any cell in $X'$.
Then $A\cap X'=(A\cap X^{(n)})\cap X'$ must be closed in $X'$.
Lemma 8.20 implies the result.

Finally, the corresponding statement for open sets follows from \Cref{8.19}.
\end{exercise}

\begin{exercise} \leavevmode
This is simply because each $n$-cell is just $D^n-S^{n-1}$;
the attachment is given by $\Phi_e(S^{n-1})$ according to \Cref{8.12}(iv).
\end{exercise}

\begin{exercise} \leavevmode
This is visually clear.
Alternatively, with $\alpha$ and $\beta$ as the edges, and $v$ as the vertex, we can notice that there is a relative homeomorphism
\[\Phi:(D^2,S^1)\to(T,\alpha\cup\beta\cup\{v\})\]
since $D^2\cong\II\times\II$.
Since $\alpha$ and $\beta$ are 1-cells, and $v$ is a 0-cell, it follows that this map gives $T$ as the union of two 1-cells, one 0-cell, and one 2-cell (namely $\im\Phi|(D^2-S^1)$).

The same argument can be done for the Klein bottle.
\end{exercise}

\begin{exercise} \leavevmode
\begin{enumerate}
\item 
They both violate closure finiteness since the closure of the base point intersects infinitely many cells.
\item
The set of all $\{1/n\}$ is closed in the weak topology, but not as a subspace.
\end{enumerate}
\end{exercise}

\begin{exercise} \leavevmode
The same proof as Theorem 7.1 holds, but with $D^n$ in place of $\Dlt^n$.
\end{exercise}

\begin{exercise} \leavevmode
First, observe that the 1-skeleton is always nonempty (as long as $X$ is nonempty).
In particular, suppose $e$ is an $n$-cell in $X$, where $n$ is the smallest dimension of a cell in $e$.
Then the relative homeomorphism $(D^n,S^{n-1})\to(e\cup X^{(n-1)},X^{(n-1)})$ implies that there is a map between $S^{n-1}$ and $X^{(n-1)}=\emptyset$, which is impossible.
Thus there must be some 0-cell, and so the 1-skeleton is nonempty.

In fact, there must be some part of the 1-skeleton in each path component.
Thus if $X$ is disconnected, then its 1-skeleton must be as well.

Now suppose that the 1-skeleton is disconnected.
We can easily show that $X^{(n)}$ disconnected implies that $X^{(n+1)}$ is disconnected.
Since $X$ is the union of all its skeletons, and since $X^{(n)}\subseteq X^{(n+1)}$, it follows that $X$ being connected would have to imply that there is some $n$ with $X^{(n)}$ connected.
Since $X^{(0)}$ is discrete, hence disconnected (unless it has one element only, in which case the 1-skeleton would be connected), it follows that $n\ge1$, and so this provides the desired contradiction.
\end{exercise}

\begin{exercise} \leavevmode
The forward direction is obvious.
Now suppose that $f\Phi_e$ is continuous for all $e$.
Let $K\subseteq Y$ be closed and let $e$ be a $k$-cell.
We want to show that
\[f^{-1}\cap\Phi_e(D^k)\]
is closed in $\bar e=\Phi_e(D^k)$.
But $\Phi_e$ is a relative homeomorphism and is, in particular, a closed map on $D^k-S^{k-1}$.
Now, because
\[\Phi_e^{-1}(f^{-1}(K)\cap\Phi_e(D^k))=(f\Phi_e)^{-1}(K)\cap D^k\]
is closed in $D^k$, we're done.
\end{exercise}

\begin{exercise} \leavevmode
\begin{enumerate}
\item 
Consider attaching the (closed) top half of the circle to the topologist's sine curve (which maps 0 to 0 and $x$ to $\sin(1/x)$ for $x\in(0,2\pi]$).
Then attach the (closed) bottom half of the circle to the same curve, but running backwards.
Obviously this is a CW complex.
But it is connected and not path-connected, violating \Cref{7.35}.
Hence this is not a polyhedron.

\item
If $n=0$, this is obvious.
Suppose this is true for $n$.
Say we attach $k$ total $(n+1)$-cells.
(Note that this kind of inductive creation of CW complexes is made possible byy Theorem 8.24.)
Note that an $(n+1)$-cell is homeomorphic to an open $(n+1)$-simplex.
Furthermore, the attachment map can be approximated by a simplicial map.
Since simplicial approximations are homotopic to the original maps, the result follows.
\end{enumerate}
\end{exercise}

\begin{exercise} \leavevmode
Here we can use the same cells and attaching maps, only with the basepoints all identified.
For any cell not equal to the basepoint, its closure is contained in whichever $X_\la$ the cell was originally in, and thus intersects only finitely many cells.
The closure of the basepoint is itself, and only intersects itself.
This proves closure finiteness.

To see that this has the weak topology, simply note that if $A$ is closed, then $A\cap X_\la$ is closed, where we identify $X_\la$ with its natural image in $\bigvee X_\la$.
Thus, since the closure of any cell is contained within $X_\la$ for some $\la$, it follows that $A\cap\bar e=A\cap X_\la\cap\bar e$ is closed in $\bar e$ for every $e$.
\end{exercise}

\begin{exercise} \leavevmode
The first and second conditions of a CW complex are clearly satisfied since $D^{i+j}$ is homeomorphic to $[0,1]^{i+j}$.

To see the third condition holds, use the equation in the hint.
Notice that all four expressions on the right side intersect finitely many cells in $X$ or $X'$.
In particular, it follows that $(\bar e-e)\times\bar e'$ intersects finitely many cells of $E''$, and similarly for the other term.
Thus $\overline{e\times e'}$ intersects $e\times e'$, plus these finitely many other cells.
This proves closure finiteness.

Finally, for the fourth condition, note that the weak topology is just the product topology when working with finitely many factors.
\end{exercise}

\begin{exercise} \leavevmode
With notation as suggested in the hint, suppose the intersection of $A$ and every cell in $E''$ is closed in the cell.
Now observe that
\[\overline{e\times a^0}=\bar e\times a^0,\]
and similarly for $b^0$.
Moreover, we know that
\[\overline{e\times c^1}=\big[(\bar e-e)\times\II\big]\cup\big(\bar e\times(a^0\cup b^0)\big)\cup(e\times c^1).\]
But now observe that $\bar e=e\cup(\bar e-e)$, so that the middle term can be rewritten as
\[\bar e\times(a^0\cup b^0)=\big(e\times(a^0\cup b^0)\big)\cup\big((\bar e-e)\times(a^0\cup b^0)\big).\]
Since $a^0\cup b^0\cup c^1=\II$, it then follows that
\[\overline{e\times c^1}=\big[(\bar e-e)\times\II\big]\cup(e\times\II)\cup\big((\bar e-e)\times(a^0\cup b^0)\big)=\bar e\times\II.\]

Now let $\pi_X$ and $\pi_\II$ be the projections to $X$ and $\II$, respectively.
We know that $\pi_X(A)\cap\bar e$ is closed in each $e$, since $\pi_X(A)\cap\bar e$ is closed in each $\bar e$, and similarly for $\pi_\II(A)\cap\bar{a^0}$ and $\pi_\II(A)\cap\bar{b^0}$.
Moreover, since $\pi_\II(A)\cap\bar{c^1}=\pi_\II(A)\cap\II$, and since $A\cap(\bar e\times\II)$ is closed in $\bar e\times\II$, it follows that the intersection $\pi_\II(A)\cap\bar{c^1}$ is also closed in $\II$.

Thus $\pi_X(A)$ and $\pi_\II(A)$ are closed, so $A$ is closed, as desired.
\end{exercise}

\begin{exercise} \leavevmode
Let $i:Z\to Y$ and $j:Y\to X$ be the injections.
We can now easily check the criteria for a strong deformation retraction.
In particular, note that
\[r_1r_2ji=r_11_Yi=r_1i=1_Z,\]
while
\[jir_1r_2=j(ir_1)r_2\simeq j_r2\rel{Z}\simeq1_X\rel{Z},\]
since $Y\subseteq Z$.
\end{exercise}

\subsection{Cellular Homology}
\begin{exercise} \leavevmode
\frownie
\end{exercise}

\begin{exercise} \leavevmode
\begin{enumerate}
\item
If $X$ is compact, then it is finite.
Thus $W_k(X,Y)=H_k(X^k_Y,X^{k-1}_Y)$ is free abelian of rank equal to the number of $k$-cells in $E-E'$.
Say this rank is $r_k$.
Then we know that
\[H_k(X,Y)\cong H_k(W_*(X,Y))=\ker d_k/\im d_{k+1},\]
but both $\ker d_k$ and $\im d_{k+1}$ are subsets of $\ZZ^{r_k}$.
Thus $H_k(X,Y)$ is finitely generated.

\item
This is the same proof, since $r_k$ is at most the number of cells of dimension $k$.
\end{enumerate}
\end{exercise}

\begin{exercise} \leavevmode
Using the cellular decomposition for $\RR P^\infty=\bigcup\RR P^n$, we find that
\[W_k(\RR P^\infty)=H_k(e^0\cup\dots\cup e^k,e^0\cup\dots\cup e^{k-1}),\]
which is obviously free abelian of rank 1.
It follows that the we get a chain $\dots\to\ZZ\to\ZZ\to\ZZ\to\dots$, so the kernels and images of each map must be 0 or $\ZZ$.
Hence $H_k(\RR P^\infty)=H_k(W_*(\RR P^\infty))$ is either 0 or $\ZZ$.
\end{exercise}

\begin{exercise} \leavevmode
Let $b_\la$ be the basepoint of $X_\la$.
Then we know that $\bigvee X_\la=\coprod X_\la/\{b_\la\}$.
Let $v$ be the natural map.
Then Theorem~8.41 implies that $v_*$ induces an isomorphism from \[H_k\left(\coprod X_\la,\{b_\la\}\right)\to\tilde H_k\left(\bigvee X_\la\right).\]
Of course, Theorems~5.13 and~5.17 also imply that the left side is equal to
\[\sum_\la\tilde H_k(X_\la),\]
which implies the result.
\end{exercise}

\begin{exercise} \leavevmode
To do this, we simply compute $d_2,d_1,d_0$.
In particular, since $W_2(T)$ is generated by $e^2$, we know that $d_2=\partial e^2=0$.
Similarly, we find that $d_1=d_0=0$.
This gives the result.
\end{exercise}

\begin{exercise} \leavevmode
Use \Cref{7.19}.
In particular, this implies that
\[\chi(S^m\times S^n)=\begin{cases}0&\text{if}~m~\text{or}~n~\text{odd}\\4\text{otherwise}\end{cases}.\]
\end{exercise}

\begin{exercise} \leavevmode
Use the cellular decomposition of $\CC P^n$.
In particular, we know that $\CC P^n=e^0\cup\dots\cup e^{2n}$, and so the only nonzero $\alpha_i$ are for even $i$.
Thus
\[\chi(\CC P^n)=\sum(-1)^i\alpha_i=1+1+\dots+1=n+1.\]
The same argument holds for $\mathbb H P^n$.
\end{exercise}

\begin{exercise} \leavevmode
This is again obvious:
\[\chi(\RR P^n)=1-1+1-1+\dots\]
is equal to 0 if $n$ is odd and 2 if $n$ is even.
This is exactly $\frac12\left(1+(-1)^n\right)$.
\end{exercise}

\begin{exercise} \leavevmode
This is simply the principle of inclusion-exclusion.
\end{exercise}

\begin{exercise} \leavevmode
It is sufficient to show that $\ecls$ is closed.
Notice that
\[\{(z_0,z_1,z_2,z_3):h^m(z_0,z_1)=(z_2,z_3)\}=\bigcup_{m=1}^p\{(z_0,z_1,z_2,z_3):h^m(z_0,z_1)=(z_2,z_3)\}=\bigcup_{m=1}^pS_m.\]
because $h^p=h$.
Thus it suffices to check that each $S_m$ is closed.
Suppose that $(z_0,z_1,z_2,z_3)\not\in S_m$.
Say that $z_2\ne\zeta^m z_0$; note that a similar argument can be given if $z_3\ne\zeta^{mq}z_1$.
Then there is an open neighborhood with coordinates $(x_0,x_1,x_2,x_3)$ on which $\zeta^m(x_0)\ne x_3$ since $\zeta^mx-y$ is continuous.
Thus $(S_m)^c$ is open, which proves that $S_m$ is closed, as desired.
\end{exercise}

\begin{exercise} \leavevmode
\begin{enumerate}
\item 
Note that $\zeta=1$, so $h$ is just the identity.
Thus $S^3/\ecls=S^3$
\item
Now we have $\zeta=-1$, so $h$ maps antipodal points to each other.
Thus $S^3/\ecls=\RR P^3$.
\item
In this case, we know that $\zeta^q=\zeta^{q'}$, so $h_q=h_{q'}$.
Thus $L(p,q)=L(p,q')$.
\end{enumerate}
\end{exercise}

\begin{exercise} \leavevmode
\begin{enumerate}
\item 
Since this is a finite decomposition, we only need to verify the first two conditions for a CW complex.
The first is clear by definition.
For the second condition, the maps are obvious for $e^0_r$ and $e^1_r$.
For $e^2_r$, we use the fact that $z_1=z_1(z_0)$ is determined by $z_0$.
Thus the map
\[z_0\mapsto(z_0,z_1(z_0))\]
works.
Finally, for $e^3_r$, take $(z_0,\theta)$ and map $\theta$ linearly onto $(2\pi r/p,2\pi(r+1)/p)$.

\item
It is easy to check that $e^i_r\sim e^i_{r'}$ for each $i$.
\end{enumerate}
\end{exercise}

\begin{exercise} \leavevmode
\begin{enumerate}
\item 
This is the cellular boundary formula, or just a generalization of the argument for Lemma~8.46.
\item
For $D(\ga_1)$, simply notice that
\[D(\ga_1)=v_\#d_1v_\#^{-1}(\ga_1)=v_\#d_1e^1_r=v_\#(e^0_r-e^0_{r+1})=0.\]
A similar argument holds for the other differentiations.
\end{enumerate}
\end{exercise}

\end{document}